\p
\begin{enumerate}
\item
پاسخ بزودی نوشته خواهد شد.
%$a_n$
%را تعداد حالات برای داشتن تعداد زوجی ۰ در 
%$n$
%بیت در نظر می‌گیریم.
%حال کم ارزش‌ترین بیت عدد را در نظر می‌گیریم؛ اگر ۰ باشد،
%$a_{n-1}$
%جواب می‌باشد چون باید مسئله را برای
%$n - 1$
%بیت دیگر حل کنیم واگر ۱ باشد، پاسخ
%$??$
%می‌شود. طبق اصل جمع پاسخ نهایی برابر عبارت زیر می‌شود:
%$$???$$
%با پایه‌های زیر:
%$$a_1 = 1$$
%$$a_2 = 2$$
%$$a_3 = 4$$
\item
پاسخ بزودی نوشته خواهد شد.
%کم‌ارزش‌ترین دو بیت را در نظر می‌گیریم؛ اگر هر دو ۰ یا هر دو ۱ بودند، حاصل
%$2a_{n-2}$
%می‌شود و اگر به صورت ۰۱ یا ۱۰ بودند، باید تعداد ۱ها و ۰ها هر کدام در ادامه ارقام این عدد فرد باشند. بنابرین حاصل
%$2_a{n-2}$
%می‌شود.
%اگر
%$n$
%فرد باشد حاصل
%$a_n$
%برابر ۰ می‌شود و اگر 
%$n$
%زوج باشد، حاصل
%$4a_{n-2}$
%می‌شود.

\end{enumerate}
\p
مساله را به صورت زیر مدل سازی می‌کنیم تا آن را با استفاده از بسط دوجمله‌ای حل کنیم.
اگر کیانوش قلکی را نشکند،
$1$،
اگر قلکی را شکسته و سکه 1 تومانی را انتخاب کند،
$x^1$،
و به همین ترتیب اگر قلک را شکسته و سکه 2 و 3 تومانی را انتخاب کند، به ترتیب،
$x^2$
و
$x^3$
را در عبارت زیر در نظر می گیریم
.
$$(1+x+x^2+x^3)^7$$
حال باید ضریب
$x^6$
در بسط چندجمله‌ای بالا را محاسبه کنیم.

\begin{enumerate}
    \item 
1 + 1 + 1 + 1 + 1 + 1 = 6
که 
$\binom{7}{6}$
حالت دارد.

    \item 
    1 + 1 + 1 + 1 + 2 = 6
که
$\binom{7}{4}\times\binom{3}{1}$
حالت دارد.

    \item 
    1 + 1 + 1 + 3 = 6
که
$\binom{7}{3}\times\binom{4}{1}$
حالت دارد.

    \item 
    1 + 1 + 2 + 2 = 6
که
$\binom{7}{2}\times\binom{5}{2}$
حالت دارد.

    \item 
    1 + 2 + 3 = 6
که
$\binom{7}{1}\times\binom{6}{1}\times\binom{5}{1}$
حالت دارد.

\item 
3 + 3 = 6
که
$\binom{7}{1}\times\binom{6}{1}$
حالت دارد.
\end{enumerate}

\p
جواب نهایی جمع 6 حالت بالا می‌باشد.
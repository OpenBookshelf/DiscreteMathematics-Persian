        \p
برای اثبات یک تساوی از طریق دوگانه‌شماری باید یک مجموعه را از دو راه مختلف حساب کنیم. این دو راه مختلف، دو طرف تساوی هستند. فرض می‌کنیم
$n$
شی داریم و می‌خواهیم
$k$
شی از آن را انتخاب کنیم. این انتخاب طرف راست تساوی است:
$$\binom n k$$
حال یک شی از
$n$
شی را انتخاب می‌کنیم و نام آن را
$x$
گذاشته و آن را کنار می‌گذاریم. سپس به سراغ
$n-1$
شی باقی‌مانده می‌رویم. دو حالت زیر به‌وجود می‌آیند:
\begin{enumerate}
\item
$x$
جزو
$k$
شی انتخابی ما باشد. در نتیجه باید از
$n-1$
شی،
$k-1$
شی را انتخاب کنیم:
$$\binom{n-1}{k-1}$$
\item
$x$
جزو
$k$
شی انتخابی ما نباشد. در نتیجه باید از
$n-1$
شی،
$k$
شی را انتخاب کنیم:
$$\binom{n-1}{k}$$
\end{enumerate}
با توجه به برابر بودن دو طرف تساوی، طبق دوگانه‌شماری مسئله اثبات شد.
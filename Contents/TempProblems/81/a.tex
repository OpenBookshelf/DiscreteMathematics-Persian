        \p
تابع مولد
$a_r$
را
$A(x)$
در نظر می‌گیریم. می‌دانیم که:
$$A(x) = A_1(x)A_2(x)A_3(x)A_4(x)$$
که
$A_i(x)$
تابع مولد
$x_i$
است. داریم:
$$A_1(x) = 1 + x^2 + x^4 + \cdots = \frac{1}{1-x^2}$$
$$A_2(x) = 1 + x^5 + x^{10} + \cdots = \frac{1}{1-x^5}$$
$$A_3(x) = 1 + x + x^2 + x^3 + x^4 = \frac{1-x^5}{1-x}$$
$$A_4(x) = 1 + x$$
پس:
$$A(x) = \frac{1}{(1-x)(1+x)}\frac{1}{1-x^5}\frac{1-x^5}{1-x}(1+x)$$
$$= \frac{1}{(1-x)^2} = 1 + 2x + 3x^2 + \cdots$$
پس مقدار
$a_7$
برابر است با
$8$
.
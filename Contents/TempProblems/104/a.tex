\p
جفت کردن خوب جفت‌کردنی است که در آن هر جفت شامل یک دانش‌آموز زرنگ و یک دانش‌آموز تنبل باشد. معلوم است که
$15!$
جفت کردن خوب وجود دارد. به ازای هر جفت کردن خوب،
$14! \times 2^{15}$
راه برای نشاندن دانش‌آموزان دور میز وجود دارد. هر چنین طرز نشاندنی را یک رابطه‌ی کاری خوب می‌نامیم. بنابرین،
$14! \times 15! \times 12^{15}$
رابطه‌ی کاری خوب وجود دارد. طرز نشاندن خوب طرز نشاندنی است که معلم بتواند دانش‌آموزان را در گروه‌های دونفره‌ی زرنگ/تنبل کنار هم طوری بنشاند که لازم نباشد از دانش‌آموزی بخواهد جایش را عوض کند. می‌خواهیم
$x$
، تعداد طرز نشستن‌های خوب را حساب کنیم. دو نوع طرز نشستن خوب وجود دارد:
\begin{enumerate}
\item
طرز نشاندن خوبی که دقیقا یک رابطه‌ی کاری خوب برقرار می‌کند. یعنی این‌که دست‌کم دو دانش‌آموز زرنگ کنار هم نشسته‌اند. این طرز نشاندن‌ها را نشاندن‌های خوب نوع اول می‌نامیم. فرض کنید
$x_1$
تعداد طرز نشاندن‌های خوب نوع اول باشد.
\item
طرز نشاندن خوبی که دقیقا دو رابطه‌ی کاری خوب برقرار می‌کند. یعنی این‌که دانش‌آموزان زرنگ و تنبل یکی در میان نشسته‌اند. این طرز نشاندن‌ها را طرز نشاندن‌های خوب نوع دوم می‌نامیم. فرض کنید
$x_2$
تعداد طرز نشاندن‌های خوب نوع دوم باشد. در این صورت
$x_2 = 14! \times 15!$
، زیرا
$14!$
راه برای نشاندن دانش‌آموزان زرنگ دور میز وجود دارد و
$15!$
راه هم برای نشاندن هر یک از دانش‌آموزان تنبل در میان دو دانش‌آموز زرنگ کنار هم وجود دارد.
\end{enumerate}
توجه کنید که
$x = x_1 + x_2$
، که در آن
$x_2 = 14! \times 15!$
و
$x_1 + 2x_2 = 14! \times 15! \times 2^{15}$
بنابرین:
$$x = 14! \times 15!(2^{15} - 1)$$
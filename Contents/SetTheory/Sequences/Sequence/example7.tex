\begin{PROBLEM}[جمله عمومی دنباله هندسی]
	\p
   اگر جمله‌ی نخست دنباله‌ای هندسی برابر با
    $a_1$
    و قدرنسبت آن برابر با
    $q$
    باشد، جمله‌ی عمومی این دنباله را حساب کنید.

    \SOLUTION{
    \p
دنباله به صورت زیر می‌باشد:
$$a_1, a_2, a_3, \cdots, a_{n-2}, a_{n-1}, a_n$$
$$a_1, a_1q, a_1q^2, \cdots, a_1q^{n - 3}, a_1q^{n - 2}, a_1q^{n - 1}$$
بنابراین جمله‌ی عمومی به صورت زیر است:
$$a_n = a_1q^{n - 1}$$
}
\end{PROBLEM}
    \begin{THEOREM}
        \p
جمله‌ی عمومی دنباله‌ی هندسی 
با جمله نخست $a_1$
و قدرنسبت $q$
برابر است با:
$$a_n = a_1q^{n - 1}$$
    \end{THEOREM}
    


\begin{PROBLEM}[مجموع جملات دنباله هندسی]
	\p
   اگر جمله‌ی نخست دنباله‌ای هندسی برابر با
$a_1$
 و قدرنسبت آن برابر با
$q$
باشد، مجموع
$n$
جمله‌ی نخست این دنباله را حساب کنید.

    \SOLUTION{
    \p
مجموع
$n$
جمله‌ی نخست دنباله را به صورت زیر به‌دست می‌آوریم.
ابتدا جمع کل را می‌نویسیم:
$$S_n = a_1 + a_2 + a_3 + \cdots + a_n$$
$$S_n = a_1 + a_1q + a_1q^2 + \cdots + a_1q^{n}$$
حال
$S_n$
را در
$q$
ضرب می‌کنیم:
$$S_nq = a_1q + a_1q^2 + a_1q^3 + \cdots + a_1q^{n - 1}$$
اکنون این دو عبارت را از هم کم کرده و با فاکتورگیری حاصل را به‌دست می‌آوریم:
$$S_n - S_nq = a_1 - a_1q^n$$
$$S_n(1 - q) = a_1(1 - q^{n})$$
$$S_n = \frac{a_1(1 - q^{n})}{1 - q}$$
    }
\end{PROBLEM}
    \begin{THEOREM}
        \p
جمله‌ی نخست دنباله‌ی هندسی 
با جمله نخست $a_1$
و قدرنسبت $q$
برابر است با:
$$S_n = a_1\frac{q^n - 1}{q - 1}$$
    \end{THEOREM}


\begin{PROBLEM}[اتحاد چاق و لاغر]
    اثبات کنید:
    $$a^n - 1 = (a-1)(a^{n-1} + a^{n-2} + \cdots + a + 1)$$

    \SOLUTION{
        دنباله‌ی هندسی $b_n$
        را با جمله‌ی نخست برابر $1$
        و قدرنسبت
        $a$
        درنظر بگیرید.
        مجموع $n$
        جمله‌ی اول این دنباله برابر است با:
        $$S_{n} = 1 + a + a^2 + \cdots + a^{n-1} = 1 \times \frac{a^n - 1}{a - 1}$$
        $$\Rightarrow a^n - 1 = (a-1)(a^{n-1} + a^{n-2} + \cdots + a + 1)$$
            }
\end{PROBLEM}
        \begin{THEOREM}
            \p
            \FOCUSEDON{اتحاد چاق و لاغر:}
            $$a^n - 1 = (a-1)(a^{n-1} + a^{n-2} + \cdots + a + 1)$$
        \end{THEOREM}
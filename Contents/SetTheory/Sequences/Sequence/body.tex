\begin{DEFINITION}
	\p
\FOCUSEDON{دنباله}
تابعی است که دامنه‌ی آن مجموعه‌ی اعداد حسابی یا طبیعی و برد آن مجموعه‌ای غیر تهی باشد.
اعداد واقع در برد یک دنباله را جملات آن دنباله می‌نامیم.
فرض کنید دنباله‌ی
$a$
با تابع
$f$
تعریف شود.
جمله
$i$ام
این دنباله را با
$a_i = f(i)$
نمایش می‌دهیم.
به 
$a_n = f(n)$
(که در آن $n$ مجهول است)
\FOCUSEDON{جمله}
\FOCUSEDON{عمومی}
دنباله نیز می‌گویند.
\end{DEFINITION}

\p
به زبان ساده‌تر، به اعدادی که به تعداد متناهی یا نامتناهی دارای ترتیب باشند، دنباله
گوییم. در دنباله‌ها تکرار مجاز است و ترتیب اهمیت دارد.
باید توجه داشت که در برخی منابع، دامنه‌ی دنباله‌ها تنها اعداد طبیعی در نظر گرفته شده است.

\begin{DEFINITION}
\p
 \FOCUSEDON{زیردنباله}
دنباله‌ای است که با حذف کردن یا نکردن اعضای دنباله‌ای دیگر، بدون به هم زدن ترتیب آن‌ها به‌وجود می‌آید.
\end{DEFINITION}
\p
به طور مثال
$\{A, B, F\}$
یک زیردنباله از دنباله‌ی
$\{A, B, C, D, E, F\}$
می‌باشد که با حذف سه عضو 
$C$
و
$D$
و
$E$
به‌دست آمده است.

\subfile{./example1.tex}
\subfile{./example2.tex}
\subfile{./example3.tex}



\SUBSECTION{یکنوایی و کران}
\begin{DEFINITION}
\begin{enumerate}
\item
 دنباله
$a_n$
\FOCUSEDON{اکیدا صعودی}
  نامیده می‌شود اگر به ازای
$n \in \mathbb{N}$
 داشته باشیم:
$$a_{n+1} > a_n$$

\item
 دنباله
$a_n$
\FOCUSEDON{اکیدا نزولی}
  نامیده می‌شود اگر به ازای
$n \in \mathbb{N}$
 داشته باشیم:
$$a_{n+1} < a_n$$

\item
 دنباله
$a_n$
\FOCUSEDON{نزولی}
  نامیده می‌شود اگر به ازای
$n \in \mathbb{N}$
 داشته باشیم:
$$a_{n+1} \leq a_n$$

\item
 دنباله
$a_n$
\FOCUSEDON{صعودی}
  نامیده می‌شود اگر به ازای
$n \in \mathbb{N}$
 داشته باشیم:
$$a_{n+1} \geq a_n$$

\item
 دنباله‌ای حقیقی  که دارای یکی از ویژگی‌های بالا است، دنباله 
 \FOCUSEDON{یکنوا}
 نامیده می‌شود.

\item
 دنباله
$a_n$
 را  
 \FOCUSEDON{از بالا کراندار}
  می‌نامند اگر عدد مثبت 
 $M$
 وجود داشته باشد که به ازای هر
$n \in \mathbb{N}$
 داشته باشیم:
$$a_{n} \leq M$$

\item
 دنباله
$a_n$
 را 
 \FOCUSEDON{از پایین کراندار}
  می‌نامند اگر عدد مثبت 
 $M$
 وجود داشته باشد که به ازای هر
$n \in \mathbb{N}$
 داشته باشیم:
$$a_{n} \geq M$$

\item
 دنباله
$a_n$
\FOCUSEDON{کراندار}
 نامیده می‌شود اگر هم از بالا و هم از پایین کراندار باشد.

\item
دنباله‌ای که کراندار نباشد، 
\FOCUSEDON{بی‌کران}
 است.

\item 
به زیردنباله‌هایی که خاصیت صعودی، نزولی، یکنوایی و ... داشته داشته باشند، 
به ترتیب
\FOCUSEDON{زیردنباله‌ی}
\FOCUSEDON{صعودی،}
\FOCUSEDON{نزولی،}
\FOCUSEDON{یکنوا}
\FOCUSEDON{و ...}
 می‌گویند.

\end{enumerate}
\end{DEFINITION}

% \p
% دنباله‌ی
%  $X = \{x_1, x_2, x_3, \cdots, x_n\}$
% را داریم. طولانی‌ترین زیردنباله‌ی صعودی
% آن را به روش‌های زیر می‌توان به‌دست آورد:
% \p
% \textbf{روش کورکورانه:} 
% تمام زیردنباله‌های
% $X$
% را به‌دست می‌آوریم. سپس، زیردنباله‌هایی که صعودی هستند را جدا کرده،
% و از بين آن‌ها دنباله‌ای که بيش‌ترین طول را دارد، به عنوان طولانی‌ترین زیر دنباله صعودی انتخاب می‌کنيم.
% برای محاسبه‌ی جواب باید تمام زیردنباله‌های
% $X$
%  را که
%  $2^n$
%  هستند، حساب کنیم.
% \p
% \textbf{روش حریصانه:} 
% از چپ به راست روی دنباله حرکت می‌کنيم. اولين عنصر دنباله را به عنوان عنصر اول زیردنباله در نظر می‌گيریم، سپس اولين عنصری از دنباله که بزرگتر از عنصری که انتخاب کردیم بود را به عنوان عنصر دوم زیردنباله در نظر می‌گيریم. حال عنصر دوم را مبنای مقایسه قرار داده و اولین عنصری که از آن بزرگ‌تر بود را به عنوان عنصر سوم در نظر می‌گیریم. به همين ترتيب تا پایان دنباله پيش می‌رویم. در این روش  باید تمام عناصر دنباله، یک‌بار ملاقات شوند و این روش هميشه جواب بهينه را نمی‌دهد. به طور مثال:
%  $$X = \{7, 3 ,2 ,6 ,9\}$$
% جواب به دست آمده از روش حریصانه:
%  $$\{7, 9\}$$
% جواب بهينه:
% $$\{3, 6, 9\}$$
% \p
% دلیل بهینه نبودن پاسخ‌ها طبق روش حریصانه، انتخاب قطعی عضو اول و به دست آوردن طولانی‌ترین زیردنباله‌ی صعودی به طوری که اولین عضو دنباله در آن باشد، است. در حالی که طولانی‌ترین زیردنباله صعودی لزوما اولین عضو دنباله را شامل نمی‌شود. پس اگر همین روش را با آغاز از هر یک از اعضای دنباله اجرا کنیم و از میان آن‌ها طولانی‌ترین زیردنباله را انتخاب کنیم، به جواب بهینه می‌رسیم. با این کار
% $n$
% بار
% $n$
% عضو دنباله را ملاقات می‌کنیم، پس این روش بسیار به‌صرفه‌تر از روش کورکورانه می‌باشد.


\SUBSECTION{همگرایی}

\begin{DEFINITION}
	\p
دنباله عددی
$a_n$
 به عدد 
$L$
\FOCUSEDON{همگرا}
است اگر به ازای هر
$\epsilon > 0$،
عدد طبیعی
$N$
وجود داشته باشد به طوری که:
$$n > N \Rightarrow |a_n - L| < \epsilon$$
به عبارت دیگر، دنباله فوق به عدد
$L$
همگراست اگر به ازای هر
$\epsilon > 0$
از مرحله‌ای به بعد، تمام جمله‌های آن در
$\epsilon$
 همسایگی
$L$
 قرار گیرند. دنباله‌ای که به عددی همگرا نباشد، 
\FOCUSEDON{واگرا}
  نامیده می‌شود.
\end{DEFINITION}

\begin{PROBLEM}[اثبات یکتایی نقطه‌ی همگرایی هر دنباله‌ی کراندار]
  \p
  نشان دهید نقطه‌ی همگرایی هر دنباله‌ی همگرا یکتاست.

  \SOLUTION{
    \p
    در حقیقت همگرایی دنباله 
    $a_n$
    به 
    $L$،
    هم‌ارز تعریف عدد
    $L$
    به عنوان حد در بی‌نهایت تابعی است که دنباله را تعریف می‌کند و چون حد تابع در هر نقطه منحصر به فرد است، پس
    $L$
    یکتاست.
    }
\end{PROBLEM}
    \begin{THEOREM}
      نقطه‌ی همگرایی هر دنباله‌ی همگرا یکتاست.
    \end{THEOREM}

	

% \begin{PROBLEM}[اثبات قضیه همگرایی دنباله‌های یکنوا و کراندار]
%   \p
%   اثبات کنید 
%   هر دنباله‌ی یکنوا و کراندار، همگرا است.
  
%   \SOLUTION{
%     \p
%     این پاسخ هنوز آماده نشده است.
%         }
% \end{PROBLEM}
    \begin{THEOREM}
      هر دنباله‌ی یکنوا و کراندار، همگرا است.
    \end{THEOREM}


\p
قضیه بالا یکی از مهم‌ترین ویژگی‌های دنباله‌های کراندار است.
به زبانی دیگر، دنباله‌های همگرا زیردسته‌ای از دسته دنباله‌های کراندار هستند.

% \begin{PROBLEM}[اثبات قضیه کراندار بودن دنباله‌های همگرا]
%   \p
%   اثبات کنید 
%   دنباله‌های همگرا کراندار هستند.
  
%   \SOLUTION{
%     \p
%     این پاسخ هنوز آماده نشده است.
%         }
% \end{PROBLEM}
    \begin{THEOREM}
      هر دنباله همگرا، کراندار است.
    \end{THEOREM}

\NOTE{
  عکس قضیه بالا صحیح نیست.
}

\begin{PROBLEM}
  \p
  نشان دهید عکس قضیه کراندار بودن دنباله‌های همگرا برقرار نیست.
  
  \SOLUTION{
    \p
    از مثال نقض استفاده می‌کنیم.
    کافیست نشان دهیم دنباله‌ای کراندار وجود دارد که همگرا نباشد.
    دنباله‌ی     
    $a_n = (-1)^n$
    دنباله‌ای کراندار و واگراست. بنابراین عکس قضیه مذکور برقرار نیست.
  }
\end{PROBLEM}

\NOTE{
دنباله‌های
$a_n, b_n$
را درنظر بگیرید که
 به ترتیب به
$A, B$
همگرا باشند.
تاثیر عملگر‌های زیر با توجه به خواص عملگر $lim$
به سادگی قابل اثبات است:
\begin{enumerate}
  \item 
  جمع:
  مجموع دو دنباله به
  $(A + B)$
  همگرا است.

  \item 
  ضرب:
  ضرب دو دنباله فوق در یکدیگر به 
  $(A.B)$
  همگراست.

  \item 
  تقسیم:
  حاصل تقسیم دو دنباله ذکر شده به 
  $(\frac{A}{B})$
  همگراست مشروط بر اینکه 
  $B \neq 0$
  و
  $b_n$
  هرگز صفر نباشد.

  \item 
  ضرب اسکالر:
  $k$
  را یک عدد ثابت و دلخواه درنظر بگیرید. در این صورت
  $ka_n$
  به
  $kA$
  همگراست.
\end{enumerate}

همچنین اگر $c_n$ یک دنباله‌ی واگرا باشد
و 
$C$
عددی مخالف صفر، می‌توان نشان داد که دنباله 
$Ca_n$
واگرا خواهد بود.
}


\begin{THEOREM}
\p
\FOCUSEDON{قضیه ساندویچ:}
هرگاه به ازای هر 
$n > N$،
برای سه دنباله‌ی
$a_n , b_n , c_n$
داشته باشیم:
$$a_n \leq b_n \leq c_n$$
$$\lim a_n = \lim c_n = L$$
آنگاه $\lim b_n = L$ خواهد بود.
\end{THEOREM}

\SUBSECTION{دنباله‌های کوشی}

\begin{DEFINITION}
	\p
دنباله 
$a_n$
را
\FOCUSEDON{کوشی}
گویند اگر به ازای هر
$\epsilon > 0$،
عدد طبیعی
$N$
وجود داشته باشد که:
$$m > n, n > N \Rightarrow |a_n - a_m| < \epsilon$$
\end{DEFINITION}


% \begin{PROBLEM}[اثبات هم‌ارزی دنباله‌های کوشی و همگرا]
%   اثبات کنید هر دنباله‌ای کوشی است، اگر و تنها اگر همگرا باشد.
  
%   \SOLUTION{
%     \p
%     این پاسخ هنوز آماده نشده است.
%         }
% \end{PROBLEM}
    \begin{THEOREM}
      هر دنباله‌ای کوشی است، اگر و تنها اگر همگرا باشد.
    \end{THEOREM}





\SUBSECTION{دنباله ثابت}

\begin{DEFINITION}
  \p
  \FOCUSEDON{دنباله ثابت}
  به دنباله‌ای گفته می‌شود که مقادیر تمام جملات آن یکسان باشد.
\end{DEFINITION}

\p
به زبان ساده‌تر، اگر
$k$
عدد ثابت دلخواهی باشد، آنگاه دنباله ثابت 
$k$
که به ازای هر 
$n$
با 
$a_n = k$
تعریف شده است، 
یک دنباله ثابت است.
مثال:
$$3, 3, 3, ...$$



\SUBSECTION{دنباله حسابی}

\begin{DEFINITION}
    \p
    \FOCUSEDON{دنباله حسابی}
    یا تصاعد حسابی به دنباله‌ای از اعداد گفته می‌شود که اختلاف هر دو جمله‌ی متوالی آن مقداری ثابت باشد. به این مقدار ثابت 
    \FOCUSEDON{قدر}
    \FOCUSEDON{نسبت}
    \FOCUSEDON{دنباله}
    \FOCUSEDON{حسابی}
      گفته می‌شود.
\end{DEFINITION}
  \p
به طور مثال دنباله‌ی زیر، یک دنباله‌ حسابی می‌باشد:
$$2, 5, 8, 11, 14, ...$$

\subfile{./example4.tex}
\subfile{./example5.tex}
\subfile{./example6.tex}


\SUBSECTION{دنباله هندسی}

\begin{DEFINITION}
    \p
    \FOCUSEDON{دنباله هندسی}
    دنباله‌ای از اعداد است که نسبت هر دو جمله متوالی آن مقداری ثابت باشد. به این مقدار ثابت 
    \FOCUSEDON{قدر نسبت دنباله هندسی}
  گفته می‌شود.    
\end{DEFINITION}
  \p
به طور مثال دنباله زیر، یک دنباله هندسی می‌باشد:
$$2, 6, 18, 54, ...$$

\subfile{./example7.tex}



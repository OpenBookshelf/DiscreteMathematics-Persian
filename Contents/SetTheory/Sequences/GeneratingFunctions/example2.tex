\begin{PROBLEM}
    \p
    اگر 
    $G(x)=\frac{1}{(1-x)^{2}}$
    باشد، ضرایب را در بسط 
    $G(x)=\sum\limits_{n=0}^{+\infty} a_{n}x^{n}$
    به دست بیاورید.( 
    $|x| < 1$
    )
     
    \SOLUTION{
        \p
        از آن جایی که 
        $|x| < 1$
        است، می‌توان گفت که داریم:
            $$ \frac{1}{1-x} = \sum\limits_{n=0}^{+\infty} x^{n} $$
        بنابراین:
            $$G(x)= \frac{1}{(1-x)^{2}} = (\sum\limits_{n=0}^{+\infty} x^{n})(\sum\limits_{n=0}^{+\infty} x^{n})$$ 
با توجه به عبارت
$(\sum\limits_{n=0}^{+\infty} x^{n})(\sum\limits_{n=0}^{+\infty} x^{n})$
، هر توان
$n$
 از
$x$
 را به طرق زیر می‌توان به‌دست آورد:\\ \\
* توان 
$n$
ام از پرانتز اول در توان صفرم از پرانتز دوم ضرب شود.\\
* توان 
$n - 1$ام از پرانتز اول در توان اول از پرانتز دوم ضرب شود.\\
* توان 
$n - 2$ام از پرانتز اول در توان دوم از پرانتز دوم ضرب شود.
$$\vdots$$
* توان صفرم از پرانتز اول در توان 
$n$
ام از پرانتز دوم ضرب شود.
\\ \p
هر کدام از عبارات بالا یک توان $n$ام از $x$ با ضریب $1$ حاصل می‌کنند که باید در جواب نهایی با هم جمع شوند. برای به‌دست آوردن تعداد آن‌ها از $0$ تا $n$ به شکل زیر عمل می‌کنیم:
            $$(\sum\limits_{n=0}^{+\infty} x^{n})(\sum\limits_{n=0}^{+\infty} x^{n}) = \sum\limits_{n=0}^{+\infty} (\sum\limits_{m=0}^{n} 1) x^{n} = \sum\limits_{n=0}^{+\infty} (n+1)x^{n}$$
        پس:
        $$a_n = n+1$$
    }
\end{PROBLEM}
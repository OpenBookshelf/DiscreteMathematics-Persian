\SECTION{اصل لانه کبوتری}

\begin{DEFINITION}
    \p
    \FOCUSEDON{اصل لانه کبوتری:}
    اگر 
    $k+1$
    کبوتر  بخواهند در 
    $k$
    لانه قرار گیرند، دست کم دو کبوتر در یک لانه قرار خواهند گرفت.
\end{DEFINITION}

\begin{THEOREM}
    \p
    مطلب بالا را می‌توان به شکل کلی‌تر، تحت عنوان
    \FOCUSEDON{تعمیم}
    \FOCUSEDON{اصل}
    \FOCUSEDON{لانه}
    \FOCUSEDON{کبوتری}
    اینگونه بیان کرد که
    اگر 
    $N$
    شیء را در 
    $k$
    جعبه قرار دهیم، آنگاه دست کم یکی از جعبه‌ها دارای
    $\left \lceil \frac{N}{k} \right \rceil$
    شیء خواهد بود.
\end{THEOREM}
\p
برای نمونه،
بین ۳۶۷ نفر، حداقل دو نفر با ماه و روز یکسان در تاریخ تولد وجود دارد؛ زیرا تنها ۳۶۶ تاریخ تولد متمایز در یک سال وجود دارد.
همچنین اگر نمره‌های ممکن برای درس ساختمان گسسته 
$A$
، 
$B$
، 
$C$
، 
$D$
و
$E$
باشد، برای اینکه دست کم ۶ تا از دانشجویان نمره‌ی یکسان بگیرند، این درس باید حداقل ۲۶ دانشجو داشته باشد.

\subfile{./example1.tex}
\subfile{./example2.tex}

\begin{PROBLEM}[قضیه‌ی زیردنباله]
    \subfile{./example3q.tex}

    \SOLUTION[اثبات \lr{Erdos-Szekeres}]{
        \subfile{./example3a.tex}
    }
\end{PROBLEM}

\SECTION{مقدمه‌ای بر روابط بازگشتی}

\begin{DEFINITION}
    \p
    می‌توان جمله‌ی عمومی برخی از دنباله‌ها را به شکل رابطه‌ای بر حسب جملات قبلی آن‌ها نوشت.
    به این نوع نوشتار،
    \FOCUSEDON{نمایش}
    \FOCUSEDON{بازگشی}
    \FOCUSEDON{دنباله}
    و به چنین رابطه‌ای،
    \FOCUSEDON{رابطه}
    \FOCUSEDON{بازگشتی}
    گفته می‌شود.
\end{DEFINITION}

\p
به طور مثال، روابط زیر، نمایش‌های بازگشتی از چند دنباله هستند:
$$a_n=a_{n-1}+a_{n-2}$$
$$a_n=2^{a_{n-1}}$$
$$a_n=a_{n-5}\times a_{n-3}$$

\p
همچنین می‌توان برای برخی از دنباله‌های بازگشتی، جمله عمومی به دست آورد. این موضوع در بخش \CROSSREF{حل روابط بازگشتی} بررسی خواهد شد.

\NOTE{
دقت کنید که برای این که بتوان اعداد یک دنباله را با استفاده از رابطه بازگشتی آن نوشت، لازم است که جملات ابتدایی این دنباله مشخص باشند.
}
\p
برای مثال اگر بدانیم که 
$a_1=1$،
می‌توان دنباله
$a_n=2^{a_{n-1}}$
را به این صورت به دست آورد:
\[1,2,4,16,65536,...\]
دقت کنید که اگر جمله ابتدایی این دنباله معلوم نبود، نمی‌شد آن را به صورت یکتا مشخص کرد و نهایتا می‌شد آن را به شکل زیر نوشت:
\[a_1,2^{a_1},2^{2^{a_1}},2^{2^{2^{a_1}}},...\]


\subfile{./ex1.tex}

\NOTE{
    اگر دنباله دارای رابطه بازگشتی به شکل
    \[a_{n}=f(a_{n-1},a_{n-2},...,a_{n-k})\]
    باشد که
    $a_{n}$
    به جمله
    $a_{n-k}$
    وابسته باشد،
    برای این که دنباله به صورت یکتا مشخص شود،
    کافی است که
    $k$
    جمله اول این دنباله را داشته باشیم.
}
\CROSSREF[اثبات]{اثبات یکتایی پاسخ رابطه بازگشتی} این موضوع را در فصل روابط بازگشتی خواهید دید.


\subfile{./ex3.tex}

\NOTE{برخی دنباله‌های معرفی شده در قسمت‌های قبل نیز نمایش بازگشتی دارند. مثلا نمایش بازگشتی دنباله حسابی با قدر نسبت 
$d$
به صورت 
$a_n=a_{n-1}+d$
و برای دنباله هندسی با قدرنسبت
$q$
به صورت
$a_n=q \times a_{n-1}$
است.}

\subfile{./ex2.tex}

\subfile{./ex4.tex}
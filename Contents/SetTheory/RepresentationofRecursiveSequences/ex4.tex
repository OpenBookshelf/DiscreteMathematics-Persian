\begin{PROBLEM}[برج هانوی]
    \p 
    در مسئله برج هانوی
    سه میله داریم که در اولین میله $n$ دیسک با قطرهای متفاوت وجود دارد که به ترتیب اندازه مرتب شده‌اند(دیسک‌های کوچکتر روی دیسک‌های بزرگتر هستند). ‌می‌خواهیم این دیسک‌ها را به دومین میله منتقل کنیم. در هر مرحله تنها یک دیسک را می‌توان حرکت داد و در هر میله، هیچ دیسک بزرگتری بر روی دیسک‌ کوچکتر قرار نمی‌گیرد. 
    فرض کنید 
    ${H_n}$
    تعداد حرکت‌های لازم برای
    $n$
    دیسک باشد.
    یک رابطه بازگشتی برای
    ${H_n}$
    بیابید.
    \SOLUTION{
        \p
        $n$ 
        دیسک را از ۱ تا 
        $n$ 
        شماره گذاری می‌کنیم. دیسک شماره یک کوچکترین و دیسک شماره
        $n$  
        بزرگترین دیسک است.
        دیسک 
        $n$
        را در میله اول ثابت در نظر می‌گیریم و سوال را برای
        $n-1$
        دیسک
        حل می‌کنیم. تعداد حرکت‌های لازم برای انتقال
        $n-1$
        دیسک به میله ۳ برابر
        ${H_{n-1}}$
        است.
        حال دیسک 
        n
        را با یک حرکت به میله ۲ می‌بریم.
        دوباره با 
        ${H_{n-1}}$
        حرکت دیسک‌های ۱ تا
        $n-1$
        را به میله ۲ بر روی دیسک 
        n
        می‌بریم.
        پس در نهایت داریم:
        $$H_n = 2 \times H_{n-1} + 1 $$
        حالت پایه نیز برابر 
        $H_1 = 1$
        است،
         زیرا یک دیسک را تنها با یک حرکت می‌توان از میله یک به سه انتقال داد.
       }
\end{PROBLEM}
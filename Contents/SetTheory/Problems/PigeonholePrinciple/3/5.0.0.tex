\p
خانه‌های صفحه 
$10\times10$
را به 25 مربع
$2\times2$
تقسیم می‌کنیم. در بین اعداد 1 تا 100، 25 عدد به صورت
$4k$،
25 عدد به صورت
$4k + 1$،
25 عدد به صورت
$4k + 2$
و 25 عدد به صورت 
$4k + 3$
هستند.
سه حالت ممکن است:
\begin{itemize}
    \item 
    در حداقل یک مربع
    $2\times2$
    هیچ عددی به صورت
    $4k$
    وجود نداشته باشد؛ در این صورت، در
    $24$
    مربع دیگر،
    $25$
    عدد به شکل
    $4k$
    وجود خواهد داشت و طبق اصل لانه کبوتری می‌دانیم حداقل در یکی از این مربع‌ها،
    حداقل
    $2$
    عدد به شکل
    $4k$
    خواهیم داشت که در این صورت حکم مسئله برقرار است.

    \item 
    در حداقل یک مربع
    $2\times2$
    هیچ عددی به صورت
    $4k+2$
    وجود نداشته باشد؛ در این حالت نیز مشابه حالت قبل،
    مربعی وجود دارد که دارای
    حداقل
    $2$
    عدد به شکل
    $4k$
    باشد و در این صورت نیز حکم مسئله برقرار است.

    \item 
    در هر یک از مربع‌های
    $2\times2$
    دقیقا یک عدد به شکل
    $4k$
    و یک عدد به شکل
    $4k+2$
    وجود داشته باشد؛
    در این صورت، با توجه به فرد($25$)
    بودن تعداد اعداد به شکل
    $4k+1$،
    حداقل یک مربع وجود دارد که حاوی تنها یک عدد به این شکل باشد.
    در این صورت عدد چهارم این مربع
    $4k+3$
    خواهد بود. از آنجایی که مجموع این دو عدد نیز بر
    $4$
    بخش‌پذیر است، در این صورت نیز حکم مسئله برقرار است.
\end{itemize}
\p
به طور دلخواه این 
$100$
 عدد را نام‌گذاری می‌کنیم:
$$x_1, x_2, x_3, \cdots, x_{99}, x_{100}\:$$
 حال زیرمجموعه‌های 
$\:y_1, y_2, \cdots, y_{99}, y_{100}$
 را طوری می‌سازیم که 
$y_i$
 شامل 
$x_1$
 تا
$x_i$
 باشد. اکنون مجموع اعضای هر کدام از آن زیرمجموعه‌ها را جداگانه حساب می‌کنیم.
 $100$
 تا حاصل جمع خواهیم داشت. اگر یکی از این حاصل جمع‌ها بر 
 $100$
 بخش‌پذیر باشد، حکم برقرار است. پس فرض می‌کنیم که هیچ‌کدام از حاصل جمع‌ها بر 
 $100$
 بخش‌پذیر نباشد. با توجه به این که دو رقم سمت راست هر یک از این حاصل جمع‌ها
 $99$
حالت متفاوت دارد، طبق اصل لانه کبوتری حداقل 
$2$
 تا از این  حاصل جمع‌ها دو رقم سمت راست یکسان دارند. اگر فرض کنیم
 $y_i$
 و
 $y_j$
 به طوری که
 $i < j$،
دو زیرمجموعه‌ای باشند که مجموع اعضای آن‌ها دو رقم سمت راست یکسانی داشته باشند، داریم:
  $$x_1 + x_2 + \cdots + x_i + \cdots + x_{j-1} + x_j\overset{100}{\equiv} x_1 + x_2 + \cdots + x_{i-1} + x_i$$
  $$\Rightarrow x_{i+1} + \cdots + x_{j-1} + x_j \overset{100}{\equiv} 0$$
  نوشتار
  $a\overset{m}{\equiv}b$
  به معنای یکسان بودن  باقی‌مانده تقسیم 
  $a$
  و
  $b$
  بر
  $m$
  است.
 \p
 بنابراین زیرمجموعه‌ای مانند
 $\{x_{i+1}, \cdots, x_{j-1}, x_j\}$
 وجود خواهد داشت که مجموع اعضایش بر
 $100$
 بخش‌پذیر باشد و حکم ثابت شود. توجه کنید که وجود یک عضو بخش‌پذیر بر
$100$ 
معادل زیرمجموعه‌ای تک عضوی به عنوان پاسخ سوال است.
 
 
\p	
از برهان خلف استفاده می‌کنیم. فرض کنید حکم درست نباشد و فاصله بین هر دو نقطه از نقاط داده شده برابر یکی از اعداد
$d_1, d_2, \ldots ,d_{10}$
باشد. یکی از نقاط مانند A را در نظر می‌گیریم.
به مرکز A ده دایره به شعاع های
$d_1$
تا
$d_{10}$
رسم می‌کنیم.
هر یک از 201 نقطه باقی‌مانده روی یکی از این دایره‌ها قرار دارند؛
پس طبق اصل لانه کبوتری، دایره‌ای وجود دارد که حداقل 21 نقطه روی آن قرار گرفته است.
این دایره را دایره C می‌نامیم.
نقاط روی دایره C را به ترتیب
$p_1, p_2, \ldots ,p_{21}$
می‌نامیم و نقطه B را نقطه ای غیر از A و این 21 نقطه فرض می‌کنیم.
به مرکز B و به شعاع های
$d_1, d_2, \ldots ,d_{10}$
ده دایره رسم می‌کنیم.
هر یک از 
$p_1$
تا
$p_{21}$
روی یکی از این دایره‌ها قرار دارند؛ پس طبق اصل لانه کبوتری دایره‌ای وجود دارد که شامل حداقل 3 نقطه از این 21 نقطه است. این دایره را 
$C'$
می‌نامیم.
دایره‌های 
$C$
و
$C'$
حداقل 3 نقطه مشترک دارند، اما می‌دانیم برخورد دو دایره با مراکز متفاوت نمی‌تواند بیش از 2 نقطه اشتراک داشته باشد. پس فرض خلف باطل و حکم برقرار است.
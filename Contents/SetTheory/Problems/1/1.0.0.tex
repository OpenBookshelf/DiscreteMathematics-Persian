\p
ابتدا ثابت می‌کنیم رابطه‌ی بین $A$ و 
$\:X = (0, \infty)\:$ یک رابطه پوشا و یک به یک است . سپس نشان می‌دهیم تعداد اعضای
 $X$
 با
 $\mathbb{R}$
 برابر است.
\p
		$f(x)$
		را به صورت
		$\: f(x) = x - 2 \:$
		تعریف می‌کنیم.
		در این صورت 
		$\: f(x):\; A \rightarrow X \:$. 
		$f(x)$
	     یک به یک است. زیرا:
		  $$if \: f(n) = f(m) \Rightarrow n = m$$
 		$f(x)$
		پوشا نیز هست. زیرا هر عضو
		$x$  
		در بازه
		 $\:(0, \infty)\:$	
		 را 	
		با
		$\:x + 2\:$
		از مجموعه
		 $A$
		متناظر می‌سازد.
 		بنابراین $|A| = |X|$.
		در نتیجه ثابت کردیم رابطه
		$\:f(x):\; A \rightarrow X\:$
		 پوشا و یک به یک است.
		 \p
 		حال می‌خواهیم برابری تعداد اعضای مجموعه‌های $A$ و $\mathbb{R}$ ثابت کنیم.
 	    مجموعه
		 $Y$
		 را طوری در نظر می‌گیریم که
	    به ازای هر عضو 
		$x$ 
		از مجموعه 
		$X$، 
		عضو 
		$-x$ 
		در  
		$Y$
		باشد($Y = -X$).
		در این صورت
		رابطه
		$\:g(x):\; A \rightarrow Y\:$
		نیز پوشا و یک به یک است.
		اکنون
 		تابع 
		 $h(x)$
		را به صورت زیر تعریف می‌کنیم: 
		$$h(2k) = f(k),\; h(2k+1) = g(k)$$
		برای اثبات  حکم مسئله،
		 ثابت می‌کنیم که رابطه 
		 $\:h(x): A \rightarrow X \cup Y\:$
	    یک به یک و پوشاست.
		با توجه به اینکه
        اگر 
        $\:h(n) = h(m)\:$
        آنگاه
		$\:n = m\:$،
		$h(x)$
		یک به یک است.
		همچنین به دلیل اینکه
		$h(x)$
		هر عضو 
		$\mathbb{R}$	
		 را 	
		با
		یک عضو
		از مجموعه
		 $A$
		متناظر می‌سازد، پوشاست.
		بنابراین:
		$$|A| = |X \cup Y| = |\mathbb{R}|$$
 		
 			
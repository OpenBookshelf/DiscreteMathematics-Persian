\p
ابتدا ثابت می‌کنیم رابطه‌ی بین A و 
$X = (0, \infty)$ یک رابطه پوشا و یک به یک است و سپس ثابت می‌کنیم تعداد اعضای
 $(0, \infty)$
 با $R$ برابر است.
 		\newline
 		تعریف می‌کنیم $f(x) = x - 2$ و می‌دانیم که این رابطه یک به یک است: ($if \: f(n) = f(m) \Rightarrow n = m$)
 		\newline
 		همچنین این رابطه پوشا نیز هست زیرا هر عضو بازه ۰ تا بی نهایت:$x$ با $x + 2$ از مجموعه A ارتباط دارد.
 		بنابراین $|A| = |X|$.
 		\bigbreak
 		حال می‌خواهیم برابری تعداد اعضای مجموعه‌های $A$ و $R$ ثابت کنیم.
 		\newline
 		تا اینجا ثابت کردیم رابطه زیر پوشا و یک به یک است:
 		\bigbreak
 		$f:\; A \rightarrow X$
 		\bigbreak
 		و به همین دلیل اگر $Y = -X$ (منظور: به ازای هر عضو x داخل مجموعه X عضو -x داخل Y باشد) نیز رابطه همچنان پوشا و یک به یک است:
 		\bigbreak
 		$g:\; A \rightarrow Y$
 		\bigbreak
 		تابع h را این گونه تعریف می‌کنیم: $h: A \rightarrow X \cup Y \:= \: R$ . اگر ثابت کنیم که رابطه بیان شده در تابع h یک رابطه یک به یک و پوشاست آنگاه حکم مسئله اثبات می‌شود:
 		\bigbreak
 		$h(2k) = f(k),\; h(2k+1) = g(k)$
 		\bigbreak
 		بنابراین ثابت شد که این رابطه پوشا و یک به یک است و بنابراین $|A| = |R|$
 		\bigbreak	
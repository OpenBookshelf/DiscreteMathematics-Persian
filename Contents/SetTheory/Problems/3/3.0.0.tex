\p
    تابع مولدی که فرم بسته‌ی آن را می‌دانیم می‌نویسیم و سعی می‌کنیم با تغییر آن به پاسخ برسیم:
\begin{itemize}
    \item 
    راه اول:
    $$ A(x) = 1 + x + x^2 + ... = \frac{1}{1 - x} $$ 
    $$ A^2(x) = 1 + 2x + 3x^2 + ... = \frac{1}{(1 - x)^2}$$ 
    $$ A^2(x) - 1 = 2x + 3x^2 + 4x^3 + ... = \frac{1}{(1 - x)^2} - 1$$
    $$ \frac{A^2(x) - 1}{x} = 2 + 3x + 4x^2 + ... = \frac{\frac{1}{(1 - x)^2} - 1}{x}$$
   \item
    راه دوم:
    می‌توانیم به جای این که
    $A(x)$
    را در خودش ضرب کنیم، از آن مشتق بگیریم:
    $$ A(x) = 1 + x + x^2 + ... = \frac{1}{1 - x} $$ 
    $$ A'(x) = 1 + 2x + 3x^2 + ... = \frac{1}{(1 - x)^2}$$
    $$ A'(x) - 1 = 2x + 3x^2 + 4x^3 + ... = \frac{1}{(1 - x)^2} - 1$$
    $$ \frac{A'(x) - 1}{x} = 2 + 3x + 4x^2 + ... = \frac{\frac{1}{(1 - x)^2} - 1}{x}$$
\end{itemize}
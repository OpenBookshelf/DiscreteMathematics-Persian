حال که با دو ابزار مهم استقرا آشنا شده‌اید،
برای بالا بردن آشنایی و تسلط،‌ به تعریف و حل مثال از گونه‌های دیگر استقرا می‌پردازیم.
از این دسته می‌توان به استقرای چندبعدی و استقرای قهقرایی اشاره کرد.

\SECTION{استقرای چندبعدی}

\begin{DEFINITION}
    فرض کنید که می‌خواهیم  گزاره
    $\mathnormal{P(m, n)}$
    را با استفاده از استقرای چندبعدی اثبات کنیم.
    برای استفاده از
    \FOCUSEDON{استقرای چندبعدی}
    مراحل زیر را انجام می‌دهیم.
    
    \begin{enumerate}
        \item [پایه:]
                
                نیاز است که گزاره 
                $\mathnormal{P(a, b)}$
                را به عنوان پایه استقرا اثبات کنیم.
                در اینجا
                $a$
                و
                $b$
                به ترتیب کوچک‌ترین مقادیری هستند که به ازای آنها
                $m$
                و
                $n$
                برقرار است.

        \item[استقرا روی :m]
        
                فرض می‌کنیم
                $\mathnormal{P(k, b)}$
                به ازای یک مقدار مثبت
                $k$
                برقرار است. باید ثابت کنیم
                $\mathnormal{P(k + 1, b)}$
                نیز برقرار خواهد بود.
        
        \item[استقرا روی :n]
                
                فرض می‌کنیم
                $\mathnormal{P(h, k)}$
                به ازای مقادیر مثبت
                $h, k$
                برقرار است. باید ثابت کنیم
                $\mathnormal{P(h, k + 1)}$
                نیز برقرار خواهد بود.


    \end{enumerate} 
    دقت کنید که تعریف بالا برای استقرای دوبعدی ارائه شد اما می‌توان تعاریف مشابهی برای استقرا با ابعاد
    بزرگتر نیز ارائه داد.
    
    به عنوان یک دید متفاوت، می‌توانید به استقرای چندبعدی به شکل استفاده از استقرا درون استقرا نگاه کنید!
\end{DEFINITION}

%i'm not sure about this, please check:
\NOTE{ممکن است استقرای چندبعدی را با نام استقرای چندگانه نیز ببینید. }
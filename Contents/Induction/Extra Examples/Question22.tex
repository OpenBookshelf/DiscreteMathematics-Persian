\begin{PROBLEM}
	\p
	% منبع اپدیا
	یک رشته به طول
	$n$
	شامل 
	$a$
	حرف 
	$A$
	، 
	$b$
	حرف
	$B$
	و 
	$c$
	حرف
	$C$
	داریم.
	
	در هر مرحله می‌توان دو حرف متوالی و متفاوت را در نظر گرفت و آن هارا با حرف سوم عوض کرد. منظور از حرف سوم،‌حرفی
	است که در دو حرف متوالی گفته شده نیامده است.

	ثابت کنید اگر حداقل باقی مانده تقسیم دو تا از اعداد
	$a, b, c$
	بر 
	$3$
	بخش پذیر باشد، می‌توان با تعدادی از عملیات های گفته شده تمام حروف رشته را یکسان کرد.
	\SOLUTION{
		\p

	}
\end{PROBLEM}
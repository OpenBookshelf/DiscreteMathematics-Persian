\begin{PROBLEM}
	\p
	ثابت کنید اعداد 
	$1, 2, \dots, n; n \in N$
	را می‌توان به صورتی روی یک خط چید به طوری که میانگین هیچ دو عددی در دنباله، میان آن دو عدد موجود نباشد.

	\SOLUTION{
		\p
		در این سوال نیز مانند سوال قبل ابتدا حکم را کمی تغییر می‌دهیم.

		$m$
		را کوچکترین عدد صحیحی در نظر بگیرید که 
		$n \leq 2^m$
		حال واضح است که به اضای اثبات حکم برای 
		$1, 2, \dots, 2^m$
		حکم برای 
		$1, 2, \dots, n$
		نیز اثبات می‌شود و فقط کافی است اعداد اضافه را از دنباله ی نهایی حذف کنیم تا به دنباله با 
		$n$
		عدد برسیم.

		پس حکم جدید به این صورت است: 

		ثابت کنید اعداد 
		$1, 2, \dots, 2^m; m \in W$
		را می‌توان به صورتی روی یک خط چید به طوری که میانگین هیچ دو عددی در دنباله، میان آن دو عدد موجود نباشد.

		این حکم را به کمک استقرا روی 
		$m$
		اثبات میکنیم.
		واضح است که حکم به ازای 
		$m = 0$
		صحیح است.

		حال فرض کنید حکم به ازای 
		$m = k; k \in W$
		صحیح باشد، ما اثبات می‌کنیم حکم به ازای 
		$m = k + 1$
		نیز صحیح است.

		اعداد را به دو مجموعه افراز می‌کنیم:
		$E$
		که شامل اعداد زوج است و 
		$O$ 
		که شامل اعداد فرد است.
		حال تمام اعداد 
		$E$
		را تقسیم بر
		$2$
		می‌کنیم و همه اعداد 
		$O$
		را ابتدا به اضافه 
		$1$
		کرده و سپس نصف می‌کنیم.

		واضح است که با این کار اعداد دو مجموعه برابر 
		$1, 2, \dots, 2^k$
		می‌شود که طبق فرض استقرا می‌توان آن ها را با ترتیبی مانند 
		$A$
		چید که میانگین هیچ دو عددی میانشان وجود نداشته باشد. حال اعداد 
		$O$
		را به این ترتیب می‌چینیم و سپس سمت راست آن اعداد
		$E$
		را با این ترتیب می‌چینیم و سپس اعداد این دو مجموعه را به حالت اولیه خود برمیگردانیم.
		با کمی دقت متوجه می‌شوید که در این صورت میانگین هیچ دو عددی میان آن ها قرار ندارد چون میانگین اعداد بین دو
		گروه که اصلا عدد صحیح نیستند و موجود نیستند و میانگین دو عدد داخل یک گروه نیز با توجه به ترتیب
		$A$
		و طبق فرض استقرا میان آن‌ها نیست.

		پس حکم به ازای 
		$m = k + 1$
		اثبات شد پس حکم به ازای تمام 
		$m \in W$
		صحیح است.
\end{PROBLEM}
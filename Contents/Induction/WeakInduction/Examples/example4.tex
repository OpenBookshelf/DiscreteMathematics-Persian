\begin{PROBLEM}
	\p
    دنباله‌ی 
    $(a_1, a_2, a_3, ..., a_m)$
    از اعداد طبیعی را
    n-zila
    می‌نامیم هرگاه بتوان هر جایگشتی از اعداد 
   $ 1, 2, 3, ..., n$
    را با حذف تعدادی از جملات آن به دست آورد. گربه ی والتر کچل ادعا کرده هر دنباله‌ی
    n-zila
    حداقل 
    $\binom{n + 1}{2}$
	جمله دارد ولی خود والتر کچل که دستی در ریاضیات گسسته ندارد با نظر گربه‌اش موافق نیست.

	به نظر شما کدام یک از این دو درست می‌گوید؟ ادعای خود را اثبات کنید.

	\SOLUTION{
		\p 

		هر دنباله‌ی
		n-zila
		حداقل 
		$\binom{n + 1}{2}$
	 .جمله دارد
	 
		پایه: 
		
		حکم به ازای 
		$ n= 1$ 
		صحیح است و دنباله ی ما باید حداقل یک عدد ۱ داشته باشد.
		
		گام استقرا:
		
		فرض کنید حکم برای
		$n = k-1$
		 ($k > 1, k \in N$)
		صحیح باشد. ما حکم را به ازای
		$n = k$
		اثبات می‌کنیم.
		
		یک دنباله‌ی 
		k-zila
		دلخواه را درنظر بگیرید.در ابتدا سمت چپ ترین ۱ و سمت چپ ترین ۲ و ... و سمت چپ ترین 
		$ k$
		را در نظر بگیرید.
		بدون از دست رفت کلیت مسئله فرض کنید میان این اعداد اولین 
		$ k$
		دیرتر از بقیه در دنباله ما آمده است و این عدد را پاشنه می‌نامیم(بقیه اعداد گفته شده در سمت چپ این عدد در دنباله قرار دارد)
		حال تمام جایگشت‌هایی از اعداد ۱ تا
		$k$
		که 
		$k$
		در سمت چپ آن‌ها قرار دارد را در نظر بگیرید چون یک دنباله‌ی
		k-zilla
		داریم پس باید با حذف تعدادی عنصر از دنباله به هر جایگشتی برسیم و چون 
		$ k$
		در سمت چپ این اعداد قرار دارد و سمت چپ ترین 
		$ k$
		ما در واقع همان پاشنه است پس همه ی اعداد این جایگشت‌ها به جز 
		$ k$
		سمت راست پایه قرار دارند.
		حال اگر 
		$ k$
		را در این جایگشت‌ها درنظر نگیریم می‌فهمیم که تمام جایگشت های اعداد ۱
		تا
		$ k-1$
		را می‌توان با حذف پاشنه و کل اعداد سمت چپش و بعضی اعداد سمت راستش ساخت پس دنباله ی سمت راست پاشنه یک دنباله‌ی
		k-1-zila
		است که طبق فرض استقرا حداقل
		$\binom{k}{2}$
		جمله دارد و همچنین توجه کنید که تعداد اعداد سمت راست پاشنه به اضافه خود پاشنه حداقل 
	 $  k$
		است(چون اعداد
	  $ 1, 2, ..., k-1$
		قبل پاشنه آمده‌اند)
		در نتیجه حداقل تعداد جملات دنباله‌ی ما برابر است با:
	   
		$\binom{k}{2} + k = \binom{k+1}{2}$
	   
		درنتیجه حکم به ازای 
		$n = k$
		اثبات می‌شود و در نتیجه حکم به ازای تمام 
		n
		ها
		($n \in N$)
		صحیح است.
	}
\end{PROBLEM}
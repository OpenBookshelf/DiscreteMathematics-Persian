\begin{PROBLEM}
	\p 
	اثبات کنید تورنومنت 
	$2 \leq n$
	راسی(گراف کامل جهت دار با 
	$n$
	راس)
	شامل یک مسیر همیلتونی(مسیر به طول
	$n$)
	است.

	\SOLUTION{
		\p
		حکم را با استقرا روی 
		$n$
		اثبات می‌کنیم.

		حکم به ازای 
		$n = 2$
		واضح است.

		حال فرض کنید حکم به ازای 
		$n = k; 2 \leq k$
		برقرار باشد. ما حکم را به ازای 
		$n = k + 1$
		اثبات می‌کنیم.

		گراف 
		$G$
		با 
		$k + 1$
		راس را در نظر بگیرید. یک راس دلخواه مانند 
		$v$
		را از آن حذف می‌کنیم. طبق فرض استقرا در گراف
		$G - v$
		یک مسیر همیلتونی موجود است.
		رئوس این مسیر را به ترتیب 
		$u_1, u_2, ..., u_k$
		بنامید به طوری که راس
		$u_i$
		به راس 
		$u_{i+1}$
		به ازای 
		$1 \leq i < k$
		یال خروجی داشته باشد.

		حال اگر از 
		$u_k$
		به 
		$v$
		یال خروجی داشته باشیم می‌توان این راس را به انتهای این مسیر همیلتونی افزود و حکم اثبات می‌شود.

		در غیر این صورت اگر از
		$v$
		به 
		$u_1$
		یال خروجی داشته باشیم می‌توان 
		$v$
		را به ابتدای مسیر افزود و در نتیجه در این حالت نیز حکم اثبات می‌شود.

		حال اگر هیج یک از دو حالت بالا برقرار نباشد یعنی از
		$u_1$
		به 
		$v$
		و از 
		$v$
		به 
		$u_k$
		یال خروجی داریم.

		در این حالت کوچکترین 
		$x$
		را در نظر بگیرید که از
		$v$
		به 
		$u_x$
		یال خروجی داریم. در این صورت یک مسیر همیلتونی به شکل زیر داریم:
		$$u_1, u_2, ..., u_{x-1}, u_x, u_{x + 1}, ..., u_{k-1}, u_k$$
		توجه کنید چون 
		$x$
		کوچکتریم عددی بود که از 
		$v$
		به 
		$u_x$
		یال خروجی داشتیم پس قطعا از 
		$u_{x-1}$
		به 
		$v$
		یال داریم.

		همچنین چون از 
		$u_1$
		به 
		$v$
		و از 
		$v$
		به 
		$u_k$
		یال خروجی داریم قطعا همجین
		$x$
		ای یافت میشود و مقدار آن بیش از ۱ است.

		پس در هر حالت حکم به ازای 
		$n = k+1$
		اثبات شد پس حکم به ازای تمام 
		$ n \in N, n > 1$
		صحیح است.
	}
\end{PROBLEM}
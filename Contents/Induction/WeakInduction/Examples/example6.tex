\begin{PROBLEM}
	\p
	ثابت کنید اگر در یک صفحه تعدادی دایره بکشیم، می‌توان ناحیه‌های به وجود آمده در صفحه را به دو رنگ سیاه و سفید رنگ آمیزی
	کرد به طوری که هیچ دو ناحیه‌ای که مرز مشترک دارند همرنگ نباشند.(به این نوع رنگ‌آمیزی، رنگ‌آمیزی مناسب می‌گوییم)

	\SOLUTION{
		\p
		فرض کنید تعداد دایره‌های کشیده شده برابر 
		$n$
		باشد.
		حکم سوال را با استقرا روی 
		$n$
		اثبات می‌کنیم.

		حکم به ازای
		$n = 1$
		صحیح است و می‌توان داخل دایره را سیاه و بیرون آن را سفید کرد.

		حال فرض کنید حکم به ازای 
		$n = k, k\in N$
		صحیح باشد. ما حکم را به ازای 
		$n = k + 1$
		اثبات می‌کنیم.

		یک دایره مانند 
		$C$
		را در نظر بگیرید و آن را حذف کنید. طبق فرض استقرا می‌توان ناحیه‌هایی که با 
		$k$
		دایره به وجود آمده را رنگ‌آمیزی مناسب کرد. حال
		$C$
		را اضافه کنید و رنگ‌های داخل آن را برعکس کنید، یعنی ناحیه‌هایی که داخل آن به رنگ سفید است را سیاه و نواحی داخل آن که 
		به رنگ سیاه است را سفید کنید.

		حال به وضوح می‌بینید که این رنگ‌آمیزی یک رنگ‌آمیزی مناسب است چون نواحی خارج آن که رنگ خود را حفظ کردند طبق فرض استقرا اگر
		مرز مشترک داشته باشند رنگ مخالف دارند، این موضوع برای نواحی داخلی دایره نیز برقرار است چون رنگ همه‌ی آن‌ها برعکس شده و نواحی که
		مرز مشترک دارند و یکی از آن ها داخل
		$C$
		و دیگری خارج آن است و مرز مشترک آن‌ها قسمتی از محیط دایره است نیز رنگ مخالف دارند چون قبل از ترسیم این دایره این دو ناحیه‌
		یکی و همرنگ بودند و با رسم دایره به دو ناحیه تبدیل شدند و رنگ ناحیه داخلی برعکس شده پس این دو ناحیه حالا رنگ مخالف دارند.

		پس حکم به ازای 
		$n = k + 1$
		صحیح است پس حکم به ازای تمام
		$n \in N$
		ها صحیح است و حکم مسئله اثبات می‌شود.
	}
	%منبع: استراتژی‌های حل مسئله صفحه 269
\end{PROBLEM}
حال که با تعریف اصل استقرای ریاضی آشنا شدیم، به حل چند مثال ساده می‌پردازیم.

\NOTE{
    ممکن است اصل استقرای ریاضی را با نام‌های
    \textit {استقرای ضعیف}
    یا
    \textit {استقرای ساده}
    نیز شنیده باشید. 
}

\begin{PROBLEM}[کلاسیک‌ترین سوال استقرا]
    به ازای هر
    $n \in \mathbb{N}$
    ثابت کنید:

    $$\sum_{i=0}^{n} i = \frac{n(n+1)}{2}$$

\end{PROBLEM}

\SOLUTION[پاسخ]{
    از استقرای ضعیف برای حل مساله استفاده می‌کنیم.
    همانطور که گفته شد، استقرای ضعیف سه مرحله اصلی دارد.
    
     فرض استقرا: فرض می‌کنیم برای
    $k \in \mathbb{N}$
    رابطه
    $\sum_{i=0}^{k} i = \frac{k(k+1)}{2}$
    برقرار است.

    حکم استقرا: باید رابطه بالا را برای
    $k+1$ 
    اثبات کنیم یعنی ثابت کنیم:
    $\sum_{i=0}^{k+1} i = \frac{(k+1)(k+2)}{2}$

    گام استقرا: قسمت اصلی مسائل استقرا معمولا مربوط به این بخش می‌شوند.
    یک مرحله بعد از فرض را در نظر بگیرید. 
    برای 
    $k+1$
    داریم:
    $\sum_{i=0}^{k+1} i = (k+1) + \sum_{i=0}^{k} i$

    حال از فرض استقرا می‌دانیم که
    $\sum_{i=0}^{k} i = \frac{k(k+1)}{2}$

    حال با یک جاگذاری ساده داریم:

    $\Rightarrow \sum_{i=0}^{k+1} i = (k+1) + \sum_{i=0}^{k} i = (k+1) + \frac{k(k+1)}{2}
    = \frac{2(k+1) + k(k+1)}{2} = \frac{(k+1)(k+2)}{2}$

    معادله بالا در واقع همان حکم استقرا است؛ حال می‌توانیم بگوییم که از فرض استقرا،
    با گام استقرا، به حکم استقرا رسیده‌ایم؛ پس حکم استقرا ثابت و مساله حل شد.
}

ممکن است در نگاه اول، راه حل بالا درست بنظر برسد، اما اگر دقیق‌تر به آن نگاه کنید، یکی از شروط
استقرا در آن رعایت نشده است.

\NOTE{
    توجه: در مسائل استقرا، نوشتن پایه اهمیت بسیار زیادی دارد.
    اثبات بدون نوشتن پایه استقرا، ناقص به شمار می‌آید.
}

در راه حل بالا، با آنکه پایه بسیار ساده است، اما به آن اشاره‌ای نشده و همین موضوع باعث
ناقص بودن راه حل شده است.

\SOLUTION[ادامه پاسخ]{
    پاسخ را با نوشتن پایه استقرا، کامل می‌کنیم.

    \begin{flushleft}
        
        $n = 0 \Rightarrow \sum_{i=0}^{0} i = \frac{0(0+1)}{2} = 0$
        
        $n = 1 \Rightarrow \sum_{i=0}^{1} i = \frac{1(1+1)}{2} = 1$

    \end{flushleft}
    به صورت کلی، در این مساله هر یک از
    $n = 0$
    یا
    $n = 1$
    می‌توانند به عنوان پایه در نظر گرفته شوند.

}
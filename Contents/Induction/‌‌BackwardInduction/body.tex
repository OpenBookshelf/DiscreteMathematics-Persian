\SECTION{استقرای قهقرایی}


در نهایت مبحث انواع استقرا را با استقرای قهقرایی یا استقرای برعکس 
\LRE{(Backward Induction)}
خاتمه می‌دهیم.
اگرچه کاربردهای استقرای قهقرایی نسبت به سایر انوع استقرا، کمتر است، اما فهم نحوه عملکرد آن
می‌تواند درک شما را از استقرا بسیار کامل کند.


\begin{DEFINITION}
    فرض کنید دامنه گزاره
    $\mathnormal{P(x)}$
    اعداد طبیعی باشد.
    اگر دو شرط زیر برقرار باشد، با استفاده از
    \FOCUSEDON{استقرای قهقرایی}
    اثبات می‌شود که
    $\mathnormal{P(n)}$
    در تمام اعداد طبیعی برقرار است.
    
    \begin{enumerate}
        \item [پایه:]
                برای هر
                $n \in \mathbb{N}$
                وجود دارد یک
                $N > n$
                به طوری که
                $\mathnormal{P(N)}$
                برقرار است.

        \item[گام:]
            برای هر
            $n \in \mathbb{N}$
            اگر
            $\mathnormal{P(n + 1)}$
            برقرار باشد، آنگاه
            $\mathnormal{P(n)}$
            نیز برقرار است؛ به عبارتی دیگر داریم:
            $\mathnormal{P(n + 1)} \implies \mathnormal{P(n)}$


    \end{enumerate} 
    
\end{DEFINITION}

\begin{EXTRA}{اثبات درستی استقرای قهقرایی}
    
    مجموعه
    $\mathnormal{S}$
    را، مجموعه تمامی اعداد طبیعی
    $n$
    در نظر بگیرید که
    $\mathnormal{P(n)}$
    برقرار نباشد.
    با استفاده از برهان خلف اثبات می‌کنیم
    $\mathnormal{S}$
    تهی است.
    فرض کنید
    $\mathnormal{S}$
    تهی نباشد؛ پیش‌تر طبق اصل خوش‌ترتیبی خواندیم که 
    $\mathnormal{S}$
    کوچک‌ترین عضوی مانند
    $n_0$
    دارد. حال طبق شرط اول
    (پایه)
    در تعریف استقرای قهقرایی، یک عضو
    $N > n$
    وجود دارد
    به طوری که
    $\mathnormal{P(N)}$
    برقرار است.
    حال با استفاده از استقرا و همچنین شرط دوم 
    (گام)
    می‌توان نشان داد که
    $\mathnormal{P(N)}$
    برای تمام اعداد طبیعی کوچک‌تر از
    $N$
    برقرار است.
    به طور دقیق‌تر، فرض کنید:

    \begin{enumerate}
        \item به ازای تمام اعداد صحیح منفی،
            $\mathnormal{P(N)}$
            را یک گزاره درست 
            (برای مثال
            $1 = 1$
            )
            در نظر بگیرید.

        \item 
            برای تمامی
            $n \in \mathbb{N}$
            تعریف می‌کنیم:
            $\mathnormal{Q(n)} = \mathnormal{P(N - n)}$

    \end{enumerate}
        حال 
        $\mathnormal{Q(0)}$
        برقرار است
        (پایه)؛
        همچنین طبق شرط دوم
        داریم:
        
        $$\mathnormal{P(N - n)} \implies \mathnormal{P(N - n - 1)}$$
        $$\rightarrow \mathnormal{Q(n)} \implies \mathnormal{Q(n + 1)}$$

        یعنی گام استقرا نیز برقرار می‌باشد.
        پس
        $\mathnormal{Q(n)}$
        بر روی تمام اعداد طبیعی برقرار بوده که نتیجه می‌دهد
        $\mathnormal{P(n)}$
        نیز برای تمامی
        $n < N$
        برقرار است.

        پس ثابت شد
        $\mathnormal{P(n_0)}$
        برقرار بوده و به تناقض برخوردیم.

\end{EXTRA}


\begin{PROBLEM}[نامساوی میانگین حسابی-هندسی]
    اعداد حقیقی 
    $a_1, a_2, \ldots a_n$
    را در نظر بگیرید.
    ثابت کنید:

    $$A_n = \frac{a_1 + a_2 + \ldots + a_n}{n} \geq \sqrt[n]{a_1a_2\ldots a_n} = G_n$$

\end{PROBLEM}

\SOLUTION[حل با استقرا قهقرایی]{
    مساله را در دو بخش حل می‌کنیم. ابتدا در بخش یک با استفاده از استقرا معمولی، مساله را برای 
    $n$
    هایی که توانی از 2 هستند حل می‌کنیم
    (در این مرحله پایه های استقرا قهقرایی ثابت می‌شوند)؛
    سپس در بخش دو، گام استقرا قهقرایی
    (اثبات صحت مرحله 
    $n$
    ام با استفاده از فرض درست بودن مرحله 
    $n+1$
    ام)
    را ثابت می‌کنیم تا مساله حل شود.

    \begin{enumerate}
        \item[بخش یک:]
            اگر 
            $n = 2^k$،
            آنگاه
            $A_n \geq G_n$؛
            حل با استفاده از استقرا:  

            پایه:

            $$k = 1: \sqrt{a_1a_2} \leq \frac{a_1 + a_2}{2}\Longleftrightarrow4a_1a_2 \leq (a_1+a_2)^2$$
            $$\Longleftrightarrow2a_1a_2\leq a_1^2+a_2^2\Longleftrightarrow(a_1-a_2)^2\geq 0$$

            فرض: مساله برای
            $A_{2^k}$
            برقرار باشد.

            حکم: مساله را برای
            $A_{2^{k+1}}$
            ثابت می‌کنیم.

            $$A_{2^{k+1}} = \frac{a_1+a_2+\ldots+a_{2^{k+1}}}{2^{k+1}}$$

            اثبات:
            تعریف می‌کنیم:

            $$B_1 = \frac{a_1+a_2+\ldots+a_{2^k}}{2^k}$$
            $$C_1 = \frac{a_{{2^k}+1}+a_{{2^k}+2}+\ldots+a_{2^{k+1}}}{2^k}$$
            $$B_2 = \sqrt[2^k]{a_1a_2\ldots a_{2^n}}$$
            $$C_2 = \sqrt[2^k]{a_{2^n+1}a_{2^n+2}\ldots a_{2^{n+1}}}$$

            حال طبق فرض استقرا می‌دانیم 
            $B_1\geq B_2, C_1\geq C_2$ 
            پس:
            $$A_{2^{k+1}} = \frac{B_1 + C_1}{2}\geq\sqrt{B_1C_1}\geq\sqrt{B_2C_2} = G_{2^{k+1}}$$

            \item[بخش دو:]

                اگر حکم برای
                $A_{k+1}$
                برقرار باشد، آنگاه برای
                $A_{k}$
                نیز برقرار است.

                اثبات: فرض مساله به ازای تمام 
                a
                های حقیقی ثابت شده است، پس با جاگذاری
                $\frac{a_1+a_2+\ldots+a_k}{k}$
                به جای
                $a_{k+1}$
                در فرض، داریم:

                $$\sqrt[k+1]{a_1a_2\ldots a_k a_{k+1}}\leq\frac{a_1+a_2+\ldots+ a_k+a_{k+1}}{k+1}$$
                $$\Longrightarrow \sqrt[k+1]{a_1a_2\ldots a_k \frac{a_1+a_2+\ldots+a_k}{k}}\leq\frac{a_1+a_2+\ldots+a_k+ \frac{a_1+a_2+\ldots+a_k}{k}}{k+1}$$
                $$\Longrightarrow \sqrt[k+1]{a_1a_2\ldots a_k \frac{a_1+a_2+\ldots+a_k}{k}}\leq\frac{a_1+a_2+\ldots+a_k}{k}$$
                $$\Longrightarrow a_1a_2\ldots a_k \frac{a_1+a_2+\ldots+a_k}{k}\leq(\frac{a_1+a_2+\ldots+a_k}{k})^{k+1}$$
                $$\Longrightarrow a_1a_2\ldots a_k \leq(\frac{a_1+a_2+\ldots+a_k}{k})^k$$
                $$\Longrightarrow\sqrt[k]{a_1a_2\ldots a_k}\leq\frac{a_1+a_2+\ldots+a_k}{k}$$
    \end{enumerate}

}
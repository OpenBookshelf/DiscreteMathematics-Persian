\begin{PROBLEM}
	\p 
	به ازای هر
	$ n \in N$
	اثبات کنید:
	$\sqrt{2 \sqrt{3 \sqrt{\dots \sqrt{n}}}} < 3$
	(TT1987)
	%استراتژي
	\SOLUTION{
		\p
		چه سوال به ظاهر ساده اما ایده‌دار و خاص!

		برای حل این سوال از استقرای معکوس نیز
		(reverse induction)
		استفاده می‌کنیم.

		خوب به حل سوال دقت کنید. حکم سوال را به این صورت تغییر می‌دهیم:
		به ازای هر دو عدد مانند
		$m <= n; n, m \in N$
		اثبات کنید داریم:
		$\sqrt{m \sqrt{(m + 1) \sqrt{\dots \sqrt{n}}}} < m + 1$

		برای اثبات این حکم جدید که آن را حکم آلفا می‌نامیم
		$n$
		را ثابت فرض می‌کنیم و حکم را برای مقادیر مختلف
		$m$
		اثبات می‌کنیم، به این صورت:
		یک 
		$n, n \in N$
		در نظر بگیرید به ازای هر 
		$m; m \leq n, m \in N$
		اثبات می‌کنیم:
		$\sqrt{m \sqrt{(m + 1) \sqrt{\dots \sqrt{n}}}} < m + 1$

		حکم را با استقرای معکوس روی
		$m$
		اثبات می‌کنیم.
		حکم بدیهتا به ازای 
		$m = n$
		صحیح است:
		$\sqrt{m} < m + 1$
		حال فرض کنید حکم آلفا به ازای 
		$ m = k; 2 < k \leq n, k \in N$
		صحیح است. ما حکم آلفا را به ازای 
		$ m = k - 1$
		اثبات می‌کنیم:

		طبق فرض استقرا داریم:
		$\sqrt{k \sqrt{(k + 1) \sqrt{\dots \sqrt{n}}}} < k + 1$
		
		پس داریم:
		$\sqrt{(k - 1) \sqrt{k \sqrt{\dots \sqrt{n}}}} < \sqrt{(k - 1)(k + 1)} = \sqrt{k^2 - 1} < k$

		پس حکم به ازای 
		$m = k - 1$
		نیز صحیح است. پس حکم آلفا به ازای هر 
		$ 2 \leq k \leq n ; n, k \in N$
		صحیح است.

		با فرض کردن 
		$ k = 2$
		حکم اصلی سوال نیز اثبات می‌شود.
	}
\end{PROBLEM}
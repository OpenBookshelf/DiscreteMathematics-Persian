\CHAPTER[../cover.jpg]{استقرا}{
    استقرا ریاضی، یک روش ریاضی برای اثبات یک گزاره یا یک نتیجه برای حالات مختلف قابل شمارش است.
    در این روش، یک گزاره منطقی یا فرمول را برای تمامی اعداد طبیعی ثابت می‌کنیم.
    در این فصل ابتدا با استقرای ریاضی و انواع آن آشنا می‌شویم و سپس با حل مسائل متنوع، دانش خود را محک می‌زنیم.
}

\subfile{./intro/body.tex}

\subfile{./wellOrderingPrinciple/body.tex}

\subfile{./wellOrderingPrinciple/samples.tex}

\subfile{./WeakInduction/body.tex}

%add Weak Induction examples here after review. They are in /WeakInduction/Examples

\subfile{./StrongInduction/body.tex}

%add Strong Induction examples here after review. They are in /StrongInduction/Examples

\subfile{./MultipleInduction/body.tex}

% add Multiple Induction examples here after review. They are in /MultipleInduction/Examples
% example no.3 f(n, m) is very useful, be sure to mention it (even in the text).

\subfile{./‌‌BackwardInduction/body.tex}


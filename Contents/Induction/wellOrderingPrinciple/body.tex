\SECTION{تعاریف و اثبات}
    پیش از اینکه به بررسی استقرای ریاضی و خواص آن بپردازیم، باید با دو اصل مهم و کلیدی آشنا شویم!
    \begin{DEFINITION}
        \FOCUSEDON{اصل خوش ترتیبی}
        بیان می‌کند که هر زیرمجموعه ناتهی از مجموعه اعداد طبیعی، عضو کوچک‌ترین دارد.
        
        به عبارت دقیق‌تر،
        برای هر مجموعه ناتهی از اعداد طبیعی مانند
        $\mathnormal{S}$،
         وجود دارد عضو
        $\mathnormal{a}$
        متعلق به 
        $\mathnormal{S}$،
        به طوری که به ازای تمامی اعضای دیگر
        $\mathnormal{S}$
        همانند
        $\mathnormal{b}$،
        داریم:
        $b \leq a$

    \end{DEFINITION}

    \begin{DEFINITION}
        برای گزاره دلخواه
        $\mathnormal{P(n)}$
        بر روی اعداد طبیعی،
        \FOCUSEDON{اصل استقرا ریاضی}
        بیان دارد که اگر دو شرط زیر برقرار باشند، آنگاه
        $\mathnormal{P(n)}$
        برای تمام 
        $\mathnormal{n}$
        های طبیعی، برقرار است.
        
        \begin{enumerate}
            \item گزاره برای 
                    $\mathnormal{n} = 1$
                     برقرار است. به عبارتی داریم:
                    \begin{flushleft}
                        $\mathnormal{P(1)} \equiv T$
                    \end{flushleft}
                    
            \item اگر گزاره برای
                    $\mathnormal{n} = k \in \mathbb{N}$
                    برقرار باشد، آنگاه برای
                    $\mathnormal{n} = k + 1$
                    نیز برقرار است. به عبارتی دیگر:
                    \begin{flushleft} 
                        $\mathnormal{P(k)} \implies \mathnormal{P(k + 1)}$
                    \end{flushleft}

        \end{enumerate} 

        به شرط اول، اصطلاحاً پایه استقرا، به 
        $\mathnormal{P(n)}$
        فرض استقرا،
        و به
        $\mathnormal{P(n + 1)}$
        حکم استقرا می‌گویند.
        به فرآیند نتیجه گرفتن حکم استقرا از فرض آن، گام استقرا گفته می‌شود.

    \end{DEFINITION}

    برای درک بهتر اصل استقرا، دنباله ای از دومینوهای پشت سر هم چیده شده را در نظر بگیرید، 
    هنگامی که اولین دومینو را فشار می‌دهید، دومینو اول، به دومینو دوم برخورد می‌کند؛ سپس، دومینو
    دوم به دومینو سوم برخورد می‌کند و... تا در نهایت، آخرین دومینو نیز به زمین می‌افتد.
    برای اینکه مطمئن باشیم که تمام دومینوها زمین می‌خورند، تنها کافی است بررسی کنیم که 
    \begin{enumerate}
        \item دومینو اول زمین می‌خورد
                
        \item هر دومینویی هنگام زمین خوردن، دومینو بعدی را نیز زمین می‌زند
    \end{enumerate} 

\NOTE{توجه کنید که در ریاضیات مجموعه ها و جبر، در برخی دیدگاه ها، عدد صفر را نیز متعلق به مجموعه
اعداد طبیعی $\mathbb{N}$
می‌دانند. پس دقت کنید که در طول این فصل، هرکجا از شما خواسته شده بود حکمی را برای مجموعه‌ای اثبات کنید، حکم را برای
تک تک اعضای آن مجموعه ثابت کنید تا اثبات ناقص نماند!
البته اکثر اوقات می‌توان حالت
$\mathnormal{n = 0}$
را از بقیه حالات جدا کرد و حکم را برای آن ثابت کرد.
یک راه دیگر آن است که پایه استقرا را برابر با
$\mathnormal{n = 0}$
قرار داد.}

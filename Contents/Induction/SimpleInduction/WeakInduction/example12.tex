\begin{PROBLEM}
	\p

	اسمشو نبر
	b
	جعبه و 
	n
	نوع توپ دارد
	$(b\ge n)$
	. در هر جعبه، تعدادی توپ وجود دارد. توجه کنید یک نوع توپ می‌تواند در چند جعبه وجود داشته باشد. 
	
	هر 
	n
	جعبه‌ای را که در نظر بگیریم، می‌توان از هر یک، ۱ توپ انتخاب کرد؛ به طوری که هیچ دو توپی از 
	n
	توپ انتخاب شده، هم‌نوع نباشند. فرض کنید مجموع تعداد توپ‌های جعبه‌ها 
	s
	باشد. کمینه‌ی ممکن s را بیابید (در واقع شما باید یک s را پیدا کنید که حالتی با 
	s
	توپ داشته باشیم؛ ولی هیچ حالتی با 
	s−1
	توپ وجود نداشته باشد).


	\SOLUTION{
		ثابت می‌کنیم پاسخ برابر
		n×(b−n+1)
		است.

		ابتدا ثابت می‌کنیم در هر آرایش، حداقل این تعداد توپ داریم. هر نوع توپی که در نظر بگیرید، باید در حداقل
		b−n+1
		جعبه آمده باشد. برهان خلف می‌زنیم. فرض کنید این طور نباشد. یعنی یک نوع توپ وجود 
		دارد که در حداقل
		n
		جعبه نیامده است. آن 
		n
		جعبه را در نظر بگیرید. نمی‌توان از آن‌ها توپ‌هایی انتخاب کرد که تمام انواع توپ‌ها انتخاب شوند و با فرض مسئله به تناقض می‌رسیم. پس 
		s≥n×(b−n+1)
		است.

		حال ثابت می‌کنیم آرایشی با 
		n×(b−n+1)
		توپ وجود دارد. حکم را با استقرا روی b ثابت می‌کنیم. برای پایه، حالت 
		b=n
		را در نظر می‌گیریم. برای هر نوع توپ، یک جعبه‌ی جدا در نظر می‌گیریم و یک توپ از آن نوع در جعبه‌ی مذکور می‌گذاریم. به این ترتیب آرایشی با 
		n×(b−n+1)=n
		توپ ارائه می‌شود. حال فرض کنید حکم برای 
		b=k
		برقرار باشد. ثابت می‌کنیم حکم برای 
		b=k+1
		نیز برقرار است. یک جعبه را کنار می‌گذاریم و در k جعبه‌ی باقی‌مانده، آرایشی با n×(k−n+1)
		توپ، مطابق فرض استقرا ارائه می‌کنیم. حال در جعبه‌ی کنار گذاشته شده از هر نوع توپ، یکی می‌گذاریم. به این ترتیب یک آرایش مطلوب به دست می‌آید که 
		n×(k−n+2)
		توپ دارد و حکم ثابت می‌شود.

			در نتیجه چون 
			n
			دلخواه بود حکم برای تمام
			$1 \leq n \leq b; n, b \in N$
			صحیح است.

	}


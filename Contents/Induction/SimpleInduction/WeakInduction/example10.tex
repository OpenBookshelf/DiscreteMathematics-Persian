\begin{PROBLEM}
	\p
	ثابت کنید به ازای هر 
	$3 \leq n$
	می‌توان 
	$n$
	عدد طبیعی و متمایز به صورت
	$a_1, a_2, \dots, a_n$
	یافت به صورتی که داشته باشیم:
	
	منبع:‌ روش های ترکیبیات ۱
	$$\frac{1}{a_1} + \frac{1}{a_2} + \dots + \frac{1}{a_n} = 1$$

	\SOLUTION{
		حکم سوال را با استقرا روی
		$n$
		اثبات می‌کنیم.

		حکم به ازای 
		$n = 3$
		صحیح است. مثال:

		$$a_1 = 2, a_2 = 3, a_3 = 6$$

		حال فرض کنید حکم به ازای 
		$n = k; k \in N, 3 \leq k$
		صحیح باشد. ما حکم را به ازای 
		$ n = k + 1$
		اثبات می‌کنیم.

		طبق فرض استقرا 
		$k$
		عدد طبیعی و متمایز داریم که :
		$\frac{1}{a_1} + \frac{1}{a_2} + \dots + \frac{1}{a_k} = 1$
		در نتیجه داریم:

		$$\frac{1}{2 \times a_1} + \frac{1}{2 \times a_2} + \dots + \frac{1}{2 \times a_k} = \frac{1}{2}$$

		پس می توانیم بنویسیم:
		$$\frac{1}{2 \times a_1} + \frac{1}{2 \times a_2} + \dots + \frac{1}{2 \times a_k} + \frac{1}{2} = 1$$

		پس 
		$k + 1$
		عدد طبیعی یافتیم که شرایط خواسته شده را دارند.
		توجه کنید که این اعداد متمایز نیز هستند چون واضح است که کوچک ترین عدد ممکن میان اعداد برابر با 
		$2$
		پس با دوبرابر کردم آن‌ها و اضافه کردن یک 
		$2$
		به این مجموعه عدد تکراری به وجود نمی‌آید.

		از درستی حکم به ازای 
		$n = k + 1$
		نتیجه می‌گیریم حکم به ازای تمام 
		$n \in N, 3 \leq n$
		ها صحیح است.
	}


\begin{PROBLEM}
	\p

	اثبات کنید در هر تورنومنت (گراف کامل جهت دار)، راسی مانند 
     $v$
     وجود دارد که با شروع از بقیه رئوس با طی کردن حداکثر دو یال می‌توان به 
     $v$
     رسید
     (برای سادگی به این راس، راس ضعیف می‌گوییم).

	\SOLUTION{
		نام گراف را 
		$G$
		بگذارید و فرض کنید 
		$n$
		راس دارد.
		
		حکم را با استقرا روی
		$n$
		اثبات میکنیم.
		
		حکم به ازای 
		$n = 1$
		و
		$n = 2$
		بدیهی است.
		
		حال فرض کنید حکم به ازای
		$n = 1$
		تا
		$n = k( k \in N, k >1)$
		صحیح باشد.
		
		در این صورت ما اثبات می‌کنیم حکم به ازای 
		$n = k + 1$
		نیز صحیح است.
		
		اثبات:
		یک راس مانند 
		$v$
		از گراف را در نظر گرفته و آن را از گراف حذف می‌کنیم.
		طبق فرض استقرا در باقی مانده‌ی گراف یک راس ضعیف مانند
		$u$
		داریم.دوباره راس 
		$v$
		را به گراف اضافه می‌کنیم.
		حال اگر از
		$v$
		به
		$u$
		یال داشته باشیم راس 
		$u$
		راس ضعیف کل گراف است و حکم اثبات می‌شود. در غیر این :
	
		مجموعه 
		$A$
		را مجموعه ای از رئوس در نظر بگیرید که به 
		$u$
		یال دارند و مجموعه 
		$B$
		را مجموعه بقیه رئوس به جز 
		$A, u, v$
		در نظر بگیرید.
	
		حال دو حالت داریم :
	
		حالت ۱) راس 
		$v$
		به یکی از رئوس داخل 
		$A$
		یال داشته باشد که در این صورت می‌توان با شروع از 
		$v$
		و طی کردن دو یال به 
		$u$
		رسید پس 
		$u$
		راس ضعیف کل گراف است.
	
		حالت ۲) تمام رئوس
		$A$
		به 
		$v$
		یال داشته باشند :
		حال در اینجا اگر دقت کنید در میابید که هر راس در 
		$B$
		باید به حداقل یک راس در 
		$A$
		یال داشته باشد چون در غیر این صورت نمی‌تواند با دو یال به 
		$u$
		برسد که تناقض است.
	
		حال از رئوس 
		$u, A$
		می‌توان مستقیم به 
		$v$
		رفت و از رئوس 
		$B$
		می‌توان با یک واسطه از 
		$A$
		به 
		$v$
		رفت. پس 
		$v$
	راس ضعیف گراف است و حکم اثبات می‌شود
	
	پس در هر صورت حکم به ازای 
	$n = k + 1$
	هم صحیح است.
	پس حکم به ازای هر 
	$n \in N$
	صحیح است و حکم سوال اثبات می‌شود.
	}


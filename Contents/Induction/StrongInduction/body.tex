پس از آشنا شدن با استقرای ساده، با ابزاری قدرت‌مند‌ و مشابه، به نام استقرای قوی آشنا می‌شویم.

استقرای قوی گونه (Variant)
ای از استقرا است که در آن فرض می‌کنیم استقرا برای تمامی مقادیر قبل از
$k$
برقرار است.
چنین فرضی اطلاعات بیشتری به ما می‌دهد و حل مساله را برای ما آسان‌تر می‌کند.

ابتدا به شیوه گذشته، استقرای قوی را تعریف می‌کنیم.

\SECTION{تعریف}
    \begin{DEFINITION}
        برای گزاره دلخواه
        $\mathnormal{P(n)}$
        بر روی اعداد طبیعی،
        \FOCUSEDON{استقرای قوی}
        بیان دارد که اگر دو شرط زیر برقرار باشند، آنگاه
        $\mathnormal{P(n)}$
        برای تمام 
        $\mathnormal{n}$
        های طبیعی، برقرار است.

        \begin{enumerate}
            \item گزاره برای 
                    $\mathnormal{n} = 1$
                     برقرار است. به عبارتی داریم:
                    \begin{flushleft}
                        $\mathnormal{P(1)} \equiv T$
                    \end{flushleft}
                    
            \item اگر گزاره برای تمامی
                    $\mathnormal{n} \leq k \in \mathbb{N}$
                    برقرار باشد، آنگاه برای
                    $\mathnormal{n} = k + 1$
                    نیز برقرار است. به عبارتی دیگر:
                    \begin{flushleft} 
                        $\mathnormal{P(1)}, \mathnormal{P(2)}, \ldots, \mathnormal{P(k)} \implies \mathnormal{P(k + 1)}$
                    \end{flushleft}

        \end{enumerate} 

        مشابه تعاریف ارائه شده برای استقرای ساده،
        به شرط اول، اصطلاحاً پایه استقرا، به 
        $\mathnormal{P(1)}, \mathnormal{P(2)}, \ldots, \mathnormal{P(k)}$
        فرض استقرا،
        و به
        $\mathnormal{P(n + 1)}$
        حکم استقرا می‌گویند.
        به فرآیند نتیجه گرفتن حکم استقرا از فرض آن، گام استقرا گفته می‌شود.


    \end{DEFINITION}

    همانطور که دیدید، فرق اصلی استقرای ساده و قوی در بخش
    \textit{فرض استقرا}
    می‌باشد.

    \NOTE{دقت کنید که استقرای ساده و استقرای قوی از لحاظ قدرت با یکدیگر برابر می‌باشند؛ در واقع هر
    مساله‌ای که با استقرای ضعیف حل شود، با استقرای قوی نیز حل می‌شود و برعکس.}

    \begin{EXTRA}{اثبات برابر بودن قدرت استقرای ساده و قوی}
        برای اثبات نکته بالا نیاز است که دو حالت زیر را بررسی کنیم:

        \begin{enumerate}
            \item[1.] اگر شرایط استقرای قوی برقرار باشد، آنگاه استقرای ساده نیز برقرار است:
            
                اگر استقرای قوی برقرار باشد، چون شرط اول
                (پایه)
                هر دو یکسان است، پس شرط اول استقرای ساده نیز برقرار خواهد بود.
                
                شرط دوم استقرای قوی نیز اظهار می‌کند که
                اگر گزاره برای تمامی
                $\mathnormal{n} \leq k$
                برقرار باشد، آنگاه برای
                $\mathnormal{n} = k + 1$
                نیز برقرار است؛ پس اگر 
                گزاره برای
                $\mathnormal{n} = k$
                برقرار باشد، آنگاه برای
                $\mathnormal{n} = k + 1$
                نیز برقرار است که این همان شرط دوم استقرای ساده می‌باشد.
                    
                بنابراین استقرای ساده نیز برقرار خواهد بود.
                
            \item[2.] اگر شرایط استقرای ضعیف برقرار باشد، آنگاه استقرای قوی نیز برقرار است:
        
                اثبات این بخش کمی از بخش پیشین دشوارتر است.
                $\mathnormal{Q(k)}$
                را به شکل گزاره
                "$\mathnormal{P(n)}$
                برای هر
                $n \leq k$
                برقرار است"،
                تعریف می‌کنیم.

                در ادامه با استفاده از استقرای ساده اثبات می‌کنیم که
                $\mathnormal{Q(n)}$
                برای تمامی
                $n$ های
                مثبت برقرار است که باعث می‌شود نتیجه بگیریم
                $\mathnormal{P(n)}$
                نیز برای تمامی
                $n$ های
                مثبت برقرار است
                (که این، همان حکم استقرای قوی می‌باشد).

                همانطور که گفتیم برای اثبات حکم بالا از استقرای ساده کمک می‌گیریم. فرض استقرای ساده
                برقرار بودن شروط استقرای قوی بر روی
                $\mathnormal{P(1)}$
                و
                $\mathnormal{P(n)}$
                می‌باشد
                (دوباره این جمله را بخوانید! در واقع استقرای ساده استفاده شده سعی در اثبات شروط استقرای قوی دارد).

                دقت کنید هنگامی که شرط پایه استقرا قوی بر روی
                $\mathnormal{P(1)}$ 
                برقرار باشد یعنی گزاره
                "$\mathnormal{P(n)}$
                برای هر
                $n \leq 1$
                برقرار است" نیز صحیح می‌باشد.
                این گزاره همان تعریف
                $\mathnormal{Q(k = 1)}$
                می‌باشد.
                پس
                $\mathnormal{Q(1)}$
                نیز برقرار است.


                از طرف دیگر هنگامی که شرط دوم استقرا قوی بر روی
                $\mathnormal{P(n)}$ 
                برقرار است یعنی از گزاره
                "$\mathnormal{P(n)}$
                برای هر
                $n \leq k$
                برقرار است" نتیجه می‌گیریم که
                $\mathnormal{P(k + 1)}$
                نیز برقرار است.
                (به طور ساده تر یعنی: اگر
                $\mathnormal{Q(k)}$
                برقرار باشد، آنگاه
                $\mathnormal{P(k + 1)}$
                برقرار است).
                
                حال از برقرار بودن
                $\mathnormal{P(k + 1)}$
                و 
                $\mathnormal{Q(k)}$
                می‌توان نتیجه گرفت که
                $\mathnormal{Q(k + 1)}$
                برقرار خواهد بود.

                پس می‌توان نتیجه گرفت:
                $\mathnormal{Q(k)} \implies \mathnormal{Q(k + 1)}$
                از طرفی نیز ثابت کردیم
                $\mathnormal{Q(1)}$
                نیز برقرار است.
                پس استقرای ساده به ما نتیجه می‌دهد که
                $\mathnormal{Q(n)}$
                به ازای تمامی
                $n$
                های مثبت برقرار است. با استفاده از این گزاره نیز نتیجه می‌گیریم که
                $\mathnormal{P(n)}$
                به ازای تمامی
                $n$
                های مثبت برقرار است.
                پس درنهایت مساله ثابت شد.
                
            \end{enumerate} 
        
    \end{EXTRA}








% \begin{PROBLEM}[کلاسیک‌ترین سوال استقرا]
%     به ازای هر
%     $n \in \mathbb{N}$
%     ثابت کنید:

%     $$\sum_{i=0}^{n} i = \frac{n(n+1)}{2}$$

% \end{PROBLEM}

% \SOLUTION[پاسخ]{
%     از استقرای ضعیف برای حل مساله استفاده می‌کنیم.
%     همانطور که گفته شد، استقرای ضعیف سه مرحله اصلی دارد.
    
%      فرض استقرا: فرض می‌کنیم برای
%     $k \in \mathbb{N}$
%     رابطه
%     $\sum_{i=0}^{k} i = \frac{k(k+1)}{2}$
%     برقرار است.

%     حکم استقرا: باید رابطه بالا را برای
%     $k+1$ 
%     اثبات کنیم یعنی ثابت کنیم:
%     $\sum_{i=0}^{k+1} i = \frac{(k+1)(k+2)}{2}$

%     گام استقرا: قسمت اصلی مسائل استقرا معمولا مربوط به این بخش می‌شوند.
%     یک مرحله بعد از فرض را در نظر بگیرید. 
%     برای 
%     $k+1$
%     داریم:
%     $\sum_{i=0}^{k+1} i = (k+1) + \sum_{i=0}^{k} i$

%     حال از فرض استقرا می‌دانیم که
%     $\sum_{i=0}^{k} i = \frac{k(k+1)}{2}$

%     حال با یک جاگذاری ساده داریم:

%     $\Rightarrow \sum_{i=0}^{k+1} i = (k+1) + \sum_{i=0}^{k} i = (k+1) + \frac{k(k+1)}{2}
%     = \frac{2(k+1) + k(k+1)}{2} = \frac{(k+1)(k+2)}{2}$

%     معادله بالا در واقع همان حکم استقرا است؛ حال می‌توانیم بگوییم که از فرض استقرا،
%     با گام استقرا، به حکم استقرا رسیده‌ایم؛ پس حکم استقرا ثابت و مساله حل شد.
% }

% ممکن است در نگاه اول، راه حل بالا درست بنظر برسد، اما اگر دقیق‌تر به آن نگاه کنید، یکی از شروط
% استقرا در آن رعایت نشده است.

% \NOTE{
%     توجه: در مسائل استقرا، نوشتن پایه اهمیت بسیار زیادی دارد.
%     اثبات بدون نوشتن پایه استقرا، ناقص به شمار می‌آید.
% }

% در راه حل بالا، با آنکه پایه بسیار ساده است، اما به آن اشاره‌ای نشده و همین موضوع باعث
% ناقص بودن راه حل شده است.

% \SOLUTION[ادامه پاسخ]{
%     پاسخ را با نوشتن پایه استقرا، کامل می‌کنیم.

%     \begin{flushleft}
        
%         $n = 0 \Rightarrow \sum_{i=0}^{0} i = \frac{0(0+1)}{2} = 0$
        
%         $n = 1 \Rightarrow \sum_{i=0}^{1} i = \frac{1(1+1)}{2} = 1$

%     \end{flushleft}
%     به صورت کلی، در این مساله هر یک از
%     $n = 0$
%     یا
%     $n = 1$
%     می‌توانند به عنوان پایه در نظر گرفته شوند.

% }
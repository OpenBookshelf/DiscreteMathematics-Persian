پس از آشنا شدن با استقرای ساده، با ابزاری قدرت‌مند‌ و مشابه، به نام استقرای قوی آشنا می‌شویم.

\SECTION{استقرای قوی}

استقرای قوی گونه (Variant)
ای از استقرا است که در آن فرض می‌کنیم استقرا برای تمامی مقادیر قبل از
$k$
برقرار است.
چنین فرضی اطلاعات بیشتری به ما می‌دهد و حل مساله را برای ما آسان‌تر می‌کند.

ابتدا به شیوه گذشته، استقرای قوی را تعریف می‌کنیم.

    \begin{DEFINITION}
        برای گزاره دلخواه
        $\mathnormal{P(n)}$
        بر روی اعداد طبیعی،
        \FOCUSEDON{استقرای قوی}
        بیان دارد که اگر دو شرط زیر برقرار باشند، آنگاه
        $\mathnormal{P(n)}$
        برای تمام 
        $\mathnormal{n}$
        های طبیعی، برقرار است.

        \begin{enumerate}
            \item گزاره برای 
                    $\mathnormal{n} = 1$
                     برقرار است. به عبارتی داریم:
                    \begin{flushleft}
                        $\mathnormal{P(1)} \equiv T$
                    \end{flushleft}
                    
            \item اگر گزاره برای تمامی
                    $\mathnormal{n} \leq k \in \mathbb{N}$
                    برقرار باشد، آنگاه برای
                    $\mathnormal{n} = k + 1$
                    نیز برقرار است. به عبارتی دیگر:
                    \begin{flushleft} 
                        $\mathnormal{P(1)}, \mathnormal{P(2)}, \ldots, \mathnormal{P(k)} \implies \mathnormal{P(k + 1)}$
                    \end{flushleft}

        \end{enumerate} 

        مشابه تعاریف ارائه شده برای استقرای ساده،
        به شرط اول، اصطلاحاً پایه استقرا، به 
        $\mathnormal{P(1)}, \mathnormal{P(2)}, \ldots, \mathnormal{P(k)}$
        فرض استقرا،
        و به
        $\mathnormal{P(n + 1)}$
        حکم استقرا می‌گویند.
        به فرآیند نتیجه گرفتن حکم استقرا از فرض آن، گام استقرا گفته می‌شود.


    \end{DEFINITION}

    همانطور که دیدید، فرق اصلی استقرای ساده و قوی در بخش
    \textit{فرض استقرا}
    می‌باشد.

    \NOTE{دقت کنید که استقرای ساده و استقرای قوی از لحاظ قدرت با یکدیگر برابر می‌باشند؛ در واقع هر
    مساله‌ای که با استقرای ضعیف حل شود، با استقرای قوی نیز حل می‌شود و برعکس.}

    \begin{EXTRA}{اثبات برابر بودن قدرت استقرای ساده و قوی}
        برای اثبات نکته بالا نیاز است که دو حالت زیر را بررسی کنیم:

        \begin{enumerate}
            \item[1.] اگر شرایط استقرای قوی برقرار باشد، آنگاه استقرای ساده نیز برقرار است:
            
                اگر استقرای قوی برقرار باشد، چون شرط اول
                (پایه)
                هر دو یکسان است، پس شرط اول استقرای ساده نیز برقرار خواهد بود.
                
                شرط دوم استقرای قوی نیز اظهار می‌کند که
                اگر گزاره برای تمامی
                $\mathnormal{n} \leq k$
                برقرار باشد، آنگاه برای
                $\mathnormal{n} = k + 1$
                نیز برقرار است؛ پس اگر 
                گزاره برای
                $\mathnormal{n} = k$
                برقرار باشد، آنگاه برای
                $\mathnormal{n} = k + 1$
                نیز برقرار است که این همان شرط دوم استقرای ساده می‌باشد.
                    
                بنابراین استقرای ساده نیز برقرار خواهد بود.
                
            \item[2.] اگر شرایط استقرای ضعیف برقرار باشد، آنگاه استقرای قوی نیز برقرار است:
        
                اثبات این بخش کمی از بخش پیشین دشوارتر است.
                $\mathnormal{Q(k)}$
                را به شکل گزاره
                "$\mathnormal{P(n)}$
                برای هر
                $n \leq k$
                برقرار است"،
                تعریف می‌کنیم.

                در ادامه با استفاده از استقرای ساده اثبات می‌کنیم که
                $\mathnormal{Q(n)}$
                برای تمامی
                $n$ های
                مثبت برقرار است که باعث می‌شود نتیجه بگیریم
                $\mathnormal{P(n)}$
                نیز برای تمامی
                $n$ های
                مثبت برقرار است
                (که این، همان حکم استقرای قوی می‌باشد).

                همانطور که گفتیم برای اثبات حکم بالا از استقرای ساده کمک می‌گیریم. فرض استقرای ساده
                برقرار بودن شروط استقرای قوی بر روی
                $\mathnormal{P(1)}$
                و
                $\mathnormal{P(n)}$
                می‌باشد
                (دوباره این جمله را بخوانید! در واقع استقرای ساده استفاده شده سعی در اثبات شروط استقرای قوی دارد).

                دقت کنید هنگامی که شرط پایه استقرا قوی بر روی
                $\mathnormal{P(1)}$ 
                برقرار باشد یعنی گزاره
                "$\mathnormal{P(n)}$
                برای هر
                $n \leq 1$
                برقرار است" نیز صحیح می‌باشد.
                این گزاره همان تعریف
                $\mathnormal{Q(k = 1)}$
                می‌باشد.
                پس
                $\mathnormal{Q(1)}$
                نیز برقرار است.


                از طرف دیگر هنگامی که شرط دوم استقرا قوی بر روی
                $\mathnormal{P(n)}$ 
                برقرار است یعنی از گزاره
                "$\mathnormal{P(n)}$
                برای هر
                $n \leq k$
                برقرار است" نتیجه می‌گیریم که
                $\mathnormal{P(k + 1)}$
                نیز برقرار است.
                (به طور ساده تر یعنی: اگر
                $\mathnormal{Q(k)}$
                برقرار باشد، آنگاه
                $\mathnormal{P(k + 1)}$
                برقرار است).
                
                حال از برقرار بودن
                $\mathnormal{P(k + 1)}$
                و 
                $\mathnormal{Q(k)}$
                می‌توان نتیجه گرفت که
                $\mathnormal{Q(k + 1)}$
                برقرار خواهد بود.

                پس می‌توان نتیجه گرفت:
                $\mathnormal{Q(k)} \implies \mathnormal{Q(k + 1)}$
                از طرفی نیز ثابت کردیم
                $\mathnormal{Q(1)}$
                نیز برقرار است.
                پس استقرای ساده به ما نتیجه می‌دهد که
                $\mathnormal{Q(n)}$
                به ازای تمامی
                $n$
                های مثبت برقرار است. با استفاده از این گزاره نیز نتیجه می‌گیریم که
                $\mathnormal{P(n)}$
                به ازای تمامی
                $n$
                های مثبت برقرار است.
                پس درنهایت مساله ثابت شد.
                
            \end{enumerate} 
        
    \end{EXTRA}

    در ادامه مثالی برایتان آورده‌ایم تا نشان دهیم با آنکه قدرت این دو نوع استقرا با یکدیگر برابر است،
    اما لازم است تا هر دو ابزار را یاد بگیریم تا مسائل را راحت‌تر حل کنیم.

\begin{PROBLEM}[]
    به ازای هر
    $n \in \mathbb{N}$
    ثابت کنید:

    $$12 | (n^4 - n^2)$$

\end{PROBLEM}

\EWSOLUTION[حل با استقرا ضعیف]{
    از استقرای ضعیف برای حل مساله استفاده می‌کنیم.
    
    پایه: به راحتی قابل بررسی است که
    $$12 | 1^4 - 1^2$$
    
    فرض می‌کنیم برای
    $k \in \mathbb{N}$
    رابطه
    $12 | (k^4 - k^2)$
    برقرار است. پس داریم:

    $$12 | (k^4 - k^2) \Rightarrow (k^4 - k^2) = 12a$$

    حال باید رابطه بالا را برای
    $k+1$ 
    اثبات کنیم، سعی می‌کنیم حکم را بسط دهیم تا به فرض برسیم

    $$ (k+1)^4 - (k+1)^2 = k^4 + 4k^3 + 6k^2 + 4k + 1 - (k^2 + 2k + 1) = $$
    $$ (k^4 - k^2) + 4k^3 + 6k^2 + 2k = 12a + 4k^3 + 6k^2 + 2k \ldots$$

    همانطور که می‌بینید حل بسیار طولانی شد و رسیدن به جواب آسان نیست.

}

حال به روش دیگر توجه کنید.

\SOLUTION[حل با استقرا قوی]{
    فرض کنید
    $k \geq 6 \in \mathbb{N}$
    و شرط استقرا برای هر 
    $m \leq k$
    برقرار باشد. ثابت می‌کنیم استقرا برای
    $m = k$
    نیز برقرار است.

    برای راحتی در نوشتار
    $l = k - 5$
    تعریف می‌کنیم.

    حال مانند راه قبل، سعی می‌کنیم حکم استقرا را بسط دهیم تا به فرض برسیم.
    
    $$ (k+1)^4 - (k+1)^2 = (l+6)^4 - (l+6)^2 = $$
    $$ (l^4 +24l^3 + 180l^2 + 864l + 1296) - (l^2 + 12l + 36) = $$
    $$ (l^4 - l^2) +24l^3 + 180l^2 + 852l + 1260 = 12a +12(2l^3 + 15l^2 + 71l + 105)$$

    پس حکم برقرار می‌باشد. در نهایت پایه استقرا را نیز می‌نویسیم.

    \NOTE{از آنجایی که در فرض
    $k \geq 6$
    ذکر شده است. پایه باید برای تمامی مقادیر کوچک‌تر از 
    $6$
    نوشته شود.
    }

        \begin{enumerate}
            \item $n = 1$:
                
                $12|(14 – 12) = 12|(1- 1) = 0 = 0 \times 12$
            \item $n = 2$:
            
                $12|(24 – 22) = 12|16-12 = 12 = 1 \times 12$
            \item $n = 3$:
            
                $12|(34 – 32) = 12|(81 – 9) = 72  = 6 \times 12$
            \item $n = 4$:
            
                $12|(44 – 42) = 12|(256- 16) = 240 = 20 \times 12$
            \item $n = 5$:
            
                $12|(54 – 52) = 12|(625- 25) = 600 = 50 \times 12$
        \end{enumerate}


    واضح است که برای
    $n \geq 6$
    پایه‌های بالا کفایت می‌کنند.
    برای مثال برای اثبات حکم برای
    $n = 13$
    از فرض برقرار بودن حکم برای
    $n = 8$
    استفاده می‌کنیم که خود این فرض نیز برای اینکه برقرار باشد از فرض برقرار بودن حکم برای
    $n = 3$
    استفاده می‌کند که برقرار است.

    یک نگرش جالب در رابطه با این نوع از مسايل استقرا این است که مساله به صورت دسته‌دسته حل می‌شود.
    یعنی ابتدا پایه برای
    $n = 1, 2, 3, 4, 5$
    ثابت شده است؛ سپس از روی پایه مسئله برای
    $n = 6, 7, 8, 9, 10$
    ثابت می‌شود. به همین ترتیب
    $n = 11, 12, 13, 14, 15$
    ثابت می‌شوند و...

}

در ادامه به حل مثال‌های بیشتر می‌پردازیم تا مفهوم استقرای قوی را بهتر درک کنید.

\begin{PROBLEM}[]
    فرض کنید یک شکلات تخته‌ای به شکل یک صفحه مستطیلی
    $n \times m \in \mathbb{N}^2$
    در اختیار داریم.
    اگر بتوانیم شکلات را تنها از روی خطوط آن برش دهیم، به چند برش نیاز داریم؟

\end{PROBLEM}


\SOLUTION[حل با استقرا قوی]{
    با کمی تفکر می‌توان فهمید که جواب
    $nm - 1$
    می‌باشد. راه‌های متفاوتی برای حل این مساله وجود دارند اما میخواهیم آن را با استفاده از
    استقرای قوی حل کنیم.

    پایه استقرا:
    برای یک مستطیل
    $1 \times 1$
    هیچ عملی نیاز نیست. پس پایه برقرار می‌باشد.
    $1 \times 1 - 1 = 0$

    فرض استقرا: به ازای هر
    $x, y \in \mathbb{N}$
    اگر
    $x \times y < nm$
    باشد، نیاز به
    $xy - 1$
    برش داریم. 

    حکم استقرا: به ازای
    $x, y \in \mathbb{N}$
    اگر
    $x \times y = nm$
    باشد، نیاز به
    $xy - 1$
    برش داریم.

    برای حل مساله، اولین برش را در نظر بگیرید. بدون کم شدن از کلیت    
    فرض می‌کنیم اولین برش عمودی است و مستطیل
    به دو بخش
    $x_1 \times y$
    و
    $x_2 \times y$
    تقسیم می‌شود.

    حال چون
    $x_1 \times y < xy$
    و
    $x_2 \times y < xy$
    می‌باشد، طبق فرض استقرا هر کدام به ترتیب
    $x_1 \times y - 1$
    و
    $x_1 \times y - 1$
    برش نیاز دارند. پس تعداد کل برش‌ها برابر است با:

    $$1 + x_1 \times y - 1 + x_2 \times y - 1 = (x_1 + x_2) \times y = x \times y - 1$$

    پس حکم استقرا ثابت شد و جواب برای مستطیل
    $n \times m$
    برابر با 
    $nm - 1$
    است.
    
    \NOTE{دقت کنید که می‌توانستیم فرض استقرا را به گونه‌های دیگری
    نیز تعریف کنیم. برای مثال بجای اینکه فرض را برای مساحت کمتر از
    $nm$
    برقرار بدانیم، می‌توانستیم فرض را برای مجموع طول و عرض کمتر از
    $n + m$
    صحیح تصور کنیم و به حل مساله بپردازیم. آنچه در اینجا حائز اهمیت است، آن است که 
    مقدار عنصری که درنهایت انتخاب می‌شود، پس از شکسته شدن شکلات، کوچک‌تر شود
    تا بتوانیم از فرض استقرا کمک بگیریم.
    
    }
}

یکی از کاربردهای استقرای قوی حل روابط بازگشتی است.
بدین صورت که می‌توان یک جواب برای رابطه داده شده، حدس زد و سپس صحت آن را با استقرای قوی بررسی کرد.
به مثال بعد توجه کنید.


\begin{PROBLEM}[]
    ثابت کنید
    $a_n = \frac{7}{3}2^n - \frac{1}{3}5^n$
    جواب معادله بازگشتی زیر می‌باشد.

    $$a_n = 7a_{n - 1} - 10a_{n - 2}$$
    $$a_0 = 2, a_1 = 3$$

\end{PROBLEM}

\SOLUTION[]{
    مساله را با استفاده از استقرای قوی حل می‌کنیم.
    
    پایه:

    $$a_0 = \frac{7}{3} \times 1 - \frac{1}{3} \times 1 = 2$$
    $$a_1 = \frac{7}{3} \times 2 - \frac{1}{3} \times 5 = 3$$

    فرض می‌کنیم که به ازای هر 
    $n \leq k \in \mathbb{N}$
    داشته باشیم:
    $a_n = \frac{7}{3}2^n - \frac{1}{3}5^n$
    حال باید ثابت کنیم چنین رابطه‌ای برای
    $n = k +‌ 1$
    نیز برقرار است. طبق تعریف رابطه داریم:

    $$a_{k + 1} = 7a_{k} - 10a_{k - 1}$$
    $$    = 7 \times (\frac{7}{3}2^k - \frac{1}{3}5^k) - 10 \times (\frac{7}{3}2^{k-1} - \frac{1}{3}5^{k-1}) $$
    $$    = (14 - 10) \times \frac{7}{3}2^{k-1} - (35 - 10) \times \frac{1}{3}5^{k-1} $$
    $$    = 4 \times \frac{7}{3}2^{k-1} - 25 \times \frac{1}{3}5^{k-1} $$
    $$    = \frac{7}{3}2^{k+1} - \frac{1}{3}5^{k+1} $$

    پس حکم استقرا ثابت شد. دقت کنید که در اثبات بالا، در واقع با استفاده از فرض استقرا توانستیم مقادیر 
    $a_k$
    و
    $a_{k-1}$
    را جای‌گذاری کنیم.
}


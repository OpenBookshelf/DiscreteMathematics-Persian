\begin{PROBLEM}
	\p
	ثابت کنید در هر گراف ساده با 
	$n$
	راس و 
	$m$
	یال که زیر گراف 
	$K_4$
	ندارد نامساوی زیر برقرار است:

	$$k \leq \floor*{\frac{n^2}{3}}$$

	\SOLUTION{
		\p
		حکم را با استقرا روی
		$n$
		اثبات می‌کنیم.
		
		مشخص است که حکم به ازای
		$n \leq 3$
		بر قرار است.

		حال فرض کنید حکم مسئله به ازای
		$n = 1$
		تا 
		$n = k; k \in N, 3 \leq k$
		صحیح باشد. ما حکم را به ازای 
		$n = k + 1$
		اثبات می‌کنیم:

		در ابتدا ۲ حالت داریم:

		حالت ۱) هیچ مثلثی نداشته باشیم:
		

		در این صورت دو راس مانند
		$u, v$
		را در نظر بگیرید که به هم یال دارند.(اگر همچین دو راسی وجود نداشته باشد گراف تهی است و در نتیجه حکم استقرا اثبات می‌شود.)
		حال چون مثلث نداریم حداکثر یکی از این دو راس می‌تواند به یک راس دیگر مانند
		$x$
		یال داشته باشد.
		در نتیجه به این دو راس حداکثر
		$(k - 1) + 1 = k$
		یال متصل است. طبق فرض استقرا میان 
		$k - 1$
		راس دیگر نیر حداکثر 
		$\frac{(k - 1)^2}{3}$
		یال وجود دارد. در نتیجه این گراف حداکثر
		$$\frac{(k - 1)^2}{3} + k - 1 = \frac{k^2 + k - 2}{3} \leq \frac{(k+1)^2}{3}$$
		یال دارد و حکم استقرا به ازای 
		$n = k + 1$
		اثبات می‌شود.

		حالت ۲) حداقل یک مثلث داشته باشیم:

		سه راس یک مثلث را در نظر بگیرید و آن‌ها را 
		$u, v, x$
		بنامید.
		حال چون 
		$K_4$
		نداریم حداکثر دو راس از این ۳ می‌توانند به یک راس دیگر در گراف یال داشته باشند پس در کل این سه راس حداکثر
		$3 + 2 *(k-2) = 2k - 1$
		یال دارند.
		از طرف دیگر طبق فرض استقرا 
		$k - 2$
		راس دیگر حداکثر 
		$\frac{(k - 2)^2}{3}$
		میان خود دارند.

		در نتیجه گراف ما حداکثر
		$$2k - 1 + \frac{(k - 2)^2}{3} = \frac{k^2 + 2k + 1}{3} = \frac{(k + 1)^2}{3}$$
		یال خواهد داشت. پس در این حالت نیز حکم استقرا اثبات می‌شود.

		دیدیم که در هر دو حالت حکم استقرا به ازای 
		$n = k + 1$
		صحیح است پس حکم به ازای هر 
		$n \in N$
		صحیح است.
	}
	% منبع(استراتژی حل مسئله)
\end{PROBLEM}
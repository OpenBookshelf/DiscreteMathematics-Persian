\p
طبق رابطه‌ی تعداد جایگشت‌های خطی با اعضای تکراری،
به
\underline{$\frac{10!}{3!2!}$ حالت}
 می‌توانیم حروف غیر از 
$t$ را در یک ردیف قرار دهیم.
بار دیگر، دو حرف
$n$ 
را یک حرف فرض می‌کنیم. پس یکی از تعداد حروف کم شده و جایگشت ۹ حرف که ۳ تا از آن‌ها تکراری هستند را 
محاسبه می‌کنیم.
در نتیجه در 
\underline{$\frac{9!}{3!}$ حالت} 
 حروف 
$n$ مجاورند.
\begin{itemize}
\item 
اگر دو حرف
 $n$ 
 مجاور باشند، می‌توانیم حروف 
 $t$
 را در 8 جایگاه‌ از 11 جایگاه‌ِ ایجاد شده، قبل و بعد سایر حروف، قرار دهیم. پس داریم:
$$x_1 + x_2 + \ldots + x_8 = 3$$
طبق تعداد پاسخ‌های معادله سیاله خطی با ضرایب واحد،
پاسخ برابر
 \underline{$\binom{10}{7}$}
است.

\item 
اگر دو حرف
$n$ 
غیرمجاور باشند، می‌توانیم حروف
$t$ 
را در 7 جایگاه‌ از جایگاه‌های ایجاد شده، قبل و بعد سایر حروف، قرار دهیم. پس داریم:
$$x_1 + x_2 + \ldots + x_7 = 3$$
طبق تعداد پاسخ‌های معادله سیاله خطی با ضرایب واحد،
پاسخ برابر 
\underline{$\binom{9}{6}$}
است.
\end{itemize}

\p
طبق اصول جمع و ضرب، جواب نهایی برابر است با:
$$\frac{9!}{3!}\times\binom{10}{7} + (\frac{10!}{3!2!} - \frac{9!}{3!})\times\frac{9}{6}$$
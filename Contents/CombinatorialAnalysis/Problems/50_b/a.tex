\p
به
\underline{$\frac{10!}{3!2!}$ حالت}
 می‌توانیم حروف غیر از 
$t$ را در یک ردیف قرار دهیم.
حال دو حرف
$n$ 
را یک حرف فرض می‌کنیم. پس یکی از تعداد حروف کم شده و جایگشت ۹ حرف که ۳ تا از آن‌ها تکراری هستند را 
محاسبه می‌کنیم.
در نتیجه در 
\underline{$\frac{9!}{3!}$ حالت} 
 حروف 
$n$ مجاورند.
\begin{itemize}
\item 
اگر دو حرف
 $n$ 
 مجاور باشند حروف 
 $t$
  را در 8 جایگاه‌ از 11 جایگاه‌ِ ایجاد شده قبل و بعد سایر حروف، می‌توانیم قرار دهیم. پس داریم:
$$x_1 + x_2 + \ldots + x_8 = 3$$
 معادله بالا 
 \underline{$\binom{10}{7}$}
جواب در مجموعه اعداد صحیح و نامنفی دارد.

\item 
اگر دو حرف
 $n$ 
 غیرمجاور باشند حروف
  $t$ را در 7 جایگاه‌ از جایگاه‌های ایجاد شده قبل و بعد سایر حروف، می‌توانیم قرار دهیم. پس داریم:
$$x_1 + x_2 + \ldots + x_7 = 3$$
 معادله بالا 
 \underline{$\binom{9}{6}$}
جواب در مجموعه اعداد صحیح و نامنفی دارد. 
\end{itemize}
در نتیجه جواب نهایی برابر است با:

$$\frac{9!}{3!}\times\binom{10}{7} + (\frac{10!}{3!2!} - \frac{9!}{3!})\times\frac{9}{6}$$
\p
به
$\frac{10!}{3!2!}$
طریق می‌توانیم حروف غیر از t را در یک ردیف قرار دهیم که در 
$\frac{9!}{3!}$
حالت حروف n مجاورند.
(یادآوری: در این حالت انگار دو حرف n را با یکدیگر یک حرف فرض می‌کنیم. پس یکی از تعداد حروف کم شده و جایگشت ۹ حرف که ۳ تا از آن‌ها تکراری هستند را محاسبه می‌کنیم).
اگر دو حرف n مجاور باشند حروف t را در 8 فضا از 11 فضای خالی ایجاد شده می‌توانیم قرار دهیم. پس داریم:

$$x_1 + x_2 \ldots + x_8 = 3$$

که معادله بالا 
$\binom{10}{7}$
جواب در مجموعه اعداد صحیح و نامنفی دارد.
اگر دو حرف n غیرمجاور باشند حروف t را در 7 فضا از فضاهای خالی می‌توانیم قرار دهیم پس داریم:

$$x_1 + x_2 + \ldots + x_7 = 3$$

که معادله بالا 
$\binom{9}{6}$
جواب در مجموعه اعداد صحیح و نامنفی دارد. 
پس جواب نهایی برابر است با:

$$\frac{9!}{3!}\times\binom{10}{7} + (\frac{10!}{3!2!} - \frac{9!}{3!})\times\frac{9}{6}$$
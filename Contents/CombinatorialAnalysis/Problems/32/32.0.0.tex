\p
فرض می‌کنیم هر روز یک وعده شام خورده شده باشد. فضای حالات ملاقات جادوگر با دوستانش را به 
$5$
فضای
$S_1, S_2, \cdots, S_5$
تقسیم می‌کنیم، به طوری که 
$S_1$
نشان‌دهنده‌ی مجموعه‌ی دیدارهایش با هر یک از دوستانش به صورت انفرادی و
$S_2$
نشان‌دهنده مجموعه‌ی دیدارهایش با دو تا از دوستانش و
...
باشد.اگر
$s$
را تعداد شب‌هایی در نظر بگیریم که جادوگر حداقل با یکی از دوستانش شام خورده،
طبق اصل شمول و عدم شمول داریم:
$$s = |S_1 \cup S_2 \cup S_3 \cup \cdots \cup S_{n-1} \cup S_n| =$$
$$|S_1| - |S_2| + |S_3| - |S_4| + \cdots + (-1)^n|S_{n-1}| + (-1)^{n+1}|S_n| $$
$$|S_1| = \binom{5}{1} \times 10$$
$$|S_2| = \binom{5}{2} \times 5$$
$$|S_3| = \binom{5}{3} \times 3$$
$$|S_4| = \binom{5}{4} \times 2$$
$$|S_5| = \binom{5}{5} \times 1$$
\p
$s$
تعداد شب‌هایی است که جادوگر حداقل با یکی از دوستانش شام خورده است اما نباید
$6$
مرتبه‌ای که به تنهایی غذا خورده فراموش شود. پس تعداد روزهایی که کنفرانس در آن برگزار شده است، برابر است با:
$$s + 6 = |S_1| - |S_2| + |S_3| - |S_4| + |S_5| + 6$$
$$= \binom{5}{1} \times 10 - \binom{5}{2} \times 5 + \binom{5}{3} \times 3 - \binom{5}{4} \times 2 + \binom{5}{5} \times 1 + 6$$
$$= 5 \times 10 - 10 \times 5 + 10 \times 3 - 5 \times 2 + 1 \times 1 + 6$$
$$= 50 - 50 + 30 -10 + 1 + 6$$
$$= 27$$
\p
برای حل از دوگانه‌شماری استفاده می‌کنیم. 
 فرض کنید می‌خواهیم دو گروه از بین
$n$
نفر تشکیل دهیم، به طوری که می‌توانیم برخی از افراد را برای هیچ یک از دو گروه انتخاب نکنیم. 
این کار را به دو صورت انجام می‌دهیم:

\begin{itemize}
\item
روش اول:
       برای هر فرد بین عضو هیچ گروهی نبودن، عضو گروه 
$1$ 
        بودن و عضو گروه 
$2$ 
        بودن،
$3$ 
        انتخاب وجود دارد:
        $$3^n$$

\item     
روش دوم:
        ابتدا
$k$
         نفر از
$n$
          را انتخاب کرده، سپس تصمیم می‌گیریم که هر کدام در گروه 
$1$
         یا در گروه
$2$ 
    باشند. این کار را برای
$k$
         ‌های 
$0$ 
          تا
$n$ 
         انجام می‌دهیم:
        $$\sum_{k=0}^{n} {\binom{n}{k}\times2^k}$$

\end{itemize}
    با توجه به معادل بودن دو روش فوق، طبق دوگانه‌شماری داریم:
    $$\sum_{k=0}^{n} {\binom{n}{k}\times2^k} = 3^n$$
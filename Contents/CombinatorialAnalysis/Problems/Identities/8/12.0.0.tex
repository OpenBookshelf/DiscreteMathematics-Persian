       \p
       از دوگانه‌شماری استفاده می‌کنیم.
         می‌خواهیم تعداد حالات انتخاب
        $2n$
        نفر از بین
        $2n$
        زوج را به دست آوریم.
        
       \begin{itemize}
        \item 
       روش اول:
         $2n$
          نفر از بین 
          $2n$ 
          زوج یا همان 
          $4n$ 
          نفر انتخاب می‌کنیم. 
          $$\binom{4n}{2n}$$
          \item 
       روش دوم:
       متغیر
       $i$
       را
       تعداد بانوانی در نظر بگیرید که همراه همسرشان انتخاب می‌شوند. در این صورت پاسخ مسئله به صورت زیر می‌باشد:
        $$\underbrace{\sum\limits_{i=0}^{n}}_{\text{د}} \underbrace{\binom{2n}{2n-2i}}_{\text{آ}} \underbrace{2^{2n-2i}}_{\text{ب}} \underbrace{\binom{2i}{i}}_{\text{ج}}$$
         
        \begin{enumerate}
         \item 
        ابتدا
       $2n-2i$ 
       زوج از بین
       $2n$ 
       زوج انتخاب می‌کنیم. 
        \item 
       از هر کدام از 
       $2n-2i$
       زوج برداشته‌ شده، زن یا مرد را انتخاب می‌کنیم (هر زوج دو حالت دارد).
       \item 
       در آخر از
       $2i$ 
       زوج باقی‌مانده 
       $i$ 
       زوج را انتخاب می‌کنیم.
       \item 
       طبق اصل جمع،
       پاسخ مسئله برابر است با عبارت فوق 
        به ازای همه‌ی 
        $i$
        های ممکن. 
       \end{enumerate}
       \end{itemize}
        \p
        با توجه به معادل بودن دو روش فوق، طبق دوگانه‌شماری داریم:
        $$\sum\limits_{i=0}^{n} 2^{2n-2i} \binom{2n}{2n-2i} \binom{2i}{i} = \binom{4n}{2n}$$
 
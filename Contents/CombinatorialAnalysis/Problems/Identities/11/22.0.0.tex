\p
برای حل این سوال از دوگانه‌شماری استفاده می‌کنیم.
فرض کنید یک گروه 
$n$ 
نفره به یک رستوران می‌روند. منوی این رستوران چهار نوع پیتزا دارد و یک نفر انتخاب می‌شود تا غذای همه را حساب کند.
می‌دانیم این فرد برای خودش پیتزای سبزیجات سفارش خواهد داد.
 این کار به دو روش می‌تواند انجام شود:
\begin{itemize}
\item 
روش اول: 
یک نفر به 
\underline{$n$ حالت}
انتخاب می‌شود تا غذای همه را حساب کند
و به 
\underline{1 حالت}
برای خودش پیتزای سبزیجات انتخاب می‌کند.  
بقیه‌ی 
$n-1$ 
نفر، 
\underline{هرکدام چهار انتخاب}
 دارند. 
$$n\times 1 \times 4^{n-1}$$

\item 
روش دوم:
ابتدا $k$ 
نفری که قرار است پیتزایی غیر از سبزیجات سفارش بدهند را به
\underline{$\binom{n}{k}$ حالت}
انتخاب می‌کنیم. 
 بنابراین هرکدام از این افراد
 \underline{3 انتخاب}
   برای پیتزا دارند و
سایر افراد به
\underline{1 حالت}
  پیتزای سبزیجات سفارش می‌دهند.
در نهایت فردی که قرار است غذاها را حساب کند انتخاب می‌کنیم.
 از آنجایی که این فرد پیتزای سبزیجات سفارش خواهد داد، به
 \underline{$\binom{n-k}{1}$ حالت}
 از بین 
 $n-k$ 
 نفر انتخاب می‌شود. با جمع عبارت برای
 \underline{تمام مقادیر ممکن 
 $k$}
همه‌ی حالت‌ها در نظر گرفته می‌شود.
$$\sum\limits_{k=0}^{n} \binom{n}{k}3^k (n-k)$$
\end{itemize}

با ‌توجه به معادل بودن هر دو روش فوق، طبق دوگانه‌شماری داریم:
$$ n\times 4^{n-1} = \sum\limits_{k=0}^{n} \binom{n}{k}3^k (n-k) $$
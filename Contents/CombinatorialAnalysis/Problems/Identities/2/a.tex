\p
از دوگانه‌شماری استفاده می‌کنیم.
فرض کنید می‌خواهیم به دو روش  
 $n$ 
 دانشجو را در سه کلاس دسته‌بندی کنیم. 
 به صورتي كه از بین آن‌ها یک نماینده انتخاب شود که به هیچ کلاسی نمي‌رود و 
 ساير دانشجویان به يكي از سه كلاس رفته يا به هيچ کلاسی نمي‌روند.
\begin{itemize}
\item 
  روش اول:
  \[\underbrace{n}_{\text{آ}}\times\underbrace{4^{n-1}}_{\text{ب}}\] 
  \begin{enumerate}
    \item  
    \p
    به 
    $n$ 
    روش نماینده را انتخاب می‌کنیم.
    \p
    \item 
    \p
    سپس نماینده، هر یک از 
    $n-1$ 
    نفر دیگر را به 4 طریق در کلاس‌ها دسته‌بندی می‌کند 
    (1 حالت برای کلاس نداشتن و 3 حالت برای کلاس داشتنِ هر دانشجو وجود دارد).
  \end{enumerate}

\item 
 روش دوم:
   \[\underbrace{\sum\limits_{k=0}^{n-1}}_{\text{د}}\underbrace{\binom{n}{k+1}}_{\text{آ}}\underbrace{(k+1)}_{\text{ب}}\times\underbrace{3^{k}}_{\text{ج}}\] 
    تعداد دانشجویانی که به کلاس می‌روند را با 
    $k$ نشان می‌دهیم.
    \begin{enumerate}
    \item
    \p
    $k+1$
    نفر را انتخاب می‌کنیم.
    \item
    \p
    سپس به 
     $k+1$
    حالت، یکی از آن‌ها را به عنوان نماینده بر می‌گزینیم.
    \item
    \p
    $k$
    نفر دیگر را به
     $3^{k}$ 
    حالت به یکی از سه کلاس می‌فرستیم.
    \item
    \p
     طبق اصل جمع به ازای همه‌ی 
     $k$
     های ممکن داریم:    
    
     $$\sum\limits_{k=0}^{n-1}\binom{n}{k+1}(k+1)\times3^{k}$$
     $$= \binom{n}{1} + \binom{n}{2} 2\times3 + \binom{n}{3} 3\times3^2 +\binom{n}{4} 4\times3^3 +\ldots +\binom{n}{n} n\times3^{n-1}$$ 
    \end{enumerate}
\end{itemize}
\p
با توجه به اینکه هر دو روش مسئله یکسانی را پاسخ داده‌اند، طبق دوگانه‌شماری داریم:

$$\sum\limits_{k=0}^{n-1}\binom{n}{k+1}(k+1)\times3^{k} = n\times4^{n-1}$$

        \p
        از دوگانه‌شماری استفاده می‌کنیم.
        فرض کنید
        $P$
        بیانگر تعداد راه‌های انتخاب یک انجمن علمی
        $n$
        نفره از بین
        $n$
        دانشجوی کامپیوتر و 
        $n$
        دانشجوی برق است؛ 
        به طوری که یک دانشجوی کامپیوتر مدیر انجمن ‌باشد.
        به دو روش این کار را انجام می‌دهیم:
        \begin{itemize}
        \item 
        روش اول:
        $$P = \underbrace{n}_{\text{آ}} \underbrace{\binom{2n-1}{n-1}}_{\text{ب}}$$
        \begin{enumerate}
        \item 
        به 
        $n$
        حالت 
        یکی از دانشجو‌های کامپیوتر را به عنوان مدیر انجمن انتخاب می‌کنیم.
        \item  
        $n-1$
        عضو دیگر انجمن را از
        $2n-1$
        دانشجوی باقی‌مانده انتخاب می‌کنیم.
       
        \end{enumerate}
        \item
        روش دوم:
        $$\underbrace{\binom{n}{i}}_{\text{آ}} \underbrace{\binom{n}{i}}_{\text{ب}} \times \underbrace{i}_{\text{ج}}  =\sum\limits_{i=0}^{n} i {\binom{n}{i}}^2$$
        \begin{enumerate}             
        \item 
        ابتدا 
        $i$
        عضو انجمن را از دانشجو‌های کامپیوتر 
        انتخاب می‌کنیم.
        \item 
        سپس
        $n-i$
        عضو دیگر را از دانشجو‌های برق انتخاب می‌کنیم.
       توجه کنید که 
    $\binom{n}{n-i} = \binom{n}{i}$.
        \item 
        در نهایت یکی از
        $i$
        دانشجوی کامپیوتر عضو انجمن را به عنوان مدیر انجمن انتخاب خواهیم کرد.
        \item
        طبق اصل جمع،
        پاسخ مسئله برابر است با عبارت فوق 
         به ازای همه‌ی 
         $i$
         های ممکن.   
         لازم به ذکر است که از آنجاییکه  قرار است یکی از دانشجو‌های کامپیوتر به عنوان مدیر انتخاب شود، همواره شرط
         $i\geq 1$
         برقرار است. 
        \end{enumerate}
        \end{itemize} 
       
        \p
        با توجه به معادل بودن دو روش فوق، طبق دوگانه‌شماری داریم:
        $$ \sum\limits_{i=0}^{n} i {\binom{n}{i}}^2 = n\binom{2n-1}{n-1}$$
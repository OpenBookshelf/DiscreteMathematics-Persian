        \p
        از دوگانه‌شماری استفاده می‌کنیم.
        فرض کنید
        $P$
        بیانگر تعداد راه‌های انتخاب یک انجمن علمی
        $n$
        نفره از بین
        $n$
        دانشجوی کامپیوتر و 
        $n$
        دانشجوی برق است؛ 
        به طوری که یک دانشجوی کامپیوتر مدیر انجمن ‌باشد.
        به دو روش این کار را انجام می‌دهیم:
        \begin{itemize}
        \item 
        روش اول:
        $$P = \underbrace{n}_{\text{آ}} \underbrace{\binom{2n-1}{n-1}}_{\text{ب}}$$
        \begin{enumerate}
        \item 
        به 
        $n$
        حالت 
        یکی از دانشجو‌های کامپیوتر را به عنوان مدیر انجمن انتخاب می‌کنیم.
        \item  
        $n-1$
        عضو دیگر انجمن را از
        $2n-1$
        دانشجوی باقی‌مانده انتخاب می‌کنیم.
       
        \end{enumerate}
        روش دوم:
        $$\sum\limits_{i=0}^{n} i {\binom{n}{i}}^2$$\\
        $$\underbrace{\binom{n}{k}}_{\text{آ}} \underbrace{\binom{n}{n-k}}_{\text{ب}} \times \underbrace{k}_{\text{ج}}  = k{\binom{n}{k}}^2$$
        \begin{enumerate}             
        \item 
        ابتدا 
        $k$
        عضو انجمن را از دانشجو‌های کامپیوتر 
        انتخاب می‌کنیم.
        \item 
        سپس
        $n-k$
        عضو دیگر را از دانشجو‌های برق انتخاب می‌کنیم.
        \item 
        در نهایت یکی از
        $k$
        دانشجوی کامپیوتر عضو انجمن را به عنوان مدیر انجمن انتخاب خواهیم کرد.
       
        حال طبق اصل جمع، به ازای تمام مقادیر    
        $k$
        داریم: 
        \begin{align*}
        P &= 1{\binom{n}{1}}^2 + 2{\binom{n}{2}}^2 + \ldots + n{\binom{n}{n}}^2\\
        &= \sum\limits_{i=0}^{n} i {\binom{n}{i}}^2
        \end{align*}
        \end{enumerate}
        \end{itemize} 
        لازم به ذکر است که از آنجاییکه که قرار است یکی از دانشجو‌های کامپیوتر به عنوان مدیر انتخاب شود، همواره شرط 
        $1\leq k$
        برقرار است.
        \p
        با توجه به معادل بودن دو روش فوق، طبق دوگانه‌شماری داریم:
        $$ \sum\limits_{i=0}^{n} i {\binom{n}{i}}^2 = n\binom{2n-1}{n-1}$$
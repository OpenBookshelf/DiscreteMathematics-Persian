    \p
    دنباله‌ی معروف 
    $c_i = 1$
    را در نظر بگیرید. طبق جدول توابع مولد پر کاربرد، تابع مولد این دنباله
    $\frac{1}{1-x}$
    است.
حال که این موضوع را می‌دانیم به حل سوال می‌پردازیم.
\begin{enumerate}
    \item 
            برای حل این قسمت باید به سادگی ضرایب جمله‌های 5 و 6 و 7 را از دنباله عادی
            $${1,1,1,1,1,\ldots}$$
            کم کنیم. داریم:
            $$S_{1}(n) = \frac{1}{1-x} - x^5 - x^6 - x^7$$
    \item
            برای دنباله‌ی 
            $b_i$
              نیز مانند قسمت «آ» عمل می‌کنیم با این تفاوت که در این قسمت جمله‌های 3 و 4 و 5 و 6 و 7 با دنباله عادی متفاوت هستند. پس برای اینکه ضرایب را با دنباله مد نظر سوال یکی کنیم به صورت زیر عمل می‌کنیم:
            $$S_{2}(n) = \frac{1}{1-x} + 3x^3 + 2x^4 + 4x^5 + 5x^6 + x^7$$
      
\end{enumerate}
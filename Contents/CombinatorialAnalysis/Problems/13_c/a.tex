\p
 فرض کنید خانه‌های آبی نشان دهنده خانه‌های مجاز برای کتاب فیزیک و عبارت‌های 
 $M$ و $CH$
 به ترتیب نشان دهنده‌ی کتاب‌های ریاضی و شیمی باشند.
  توجه کنید قرار گرفتن 
 $M$ و $CH$
در یک خانه به معنی امکان بودن کتاب شیمی یا ریاضی در آن خانه است. 
 در این صورت حالت قرار گرفتن کتاب‌ها به صورت زیر می‌باشد:
\centerimage{1.0}{./0.png}  
ابتدا برای قرار دادن کتاب‌های شیمی، به
\underline{${7\choose 2}$ حالت}
دو جایگاه انتخاب کرده و به 
 \underline{$2!$ حالت}
 کتاب‌ها را در دو جایگاه انتخاب شده قرار می‌دهیم.
 سپس کتاب‌‌های ریاضی را به
 \underline{$7!$ حالت}
 در جایگاه‌های مناسب
 می‌چینیم.
 برای قرار دادن کتاب‌های فیزیک نیز ابتدا به
 \underline{${8\choose 3}$ حالت}
 سه جایگاه از 8 جایگاه آبی انتخاب کرده و 
 به 
 \underline{$3!$ حالت}
 کتاب‌ها را در سه جایگاه انتخاب شده می‌چینیم(ترتیب).
  طبق اصل ضرب جواب نهایی برابر است با:
$${7\choose 2}\times 2!\times7! \times {8\choose 3}\times 3!$$
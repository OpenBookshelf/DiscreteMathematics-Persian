\p
با قرار دادن تعداد مناسبی صفر در سمت چپ اعدادی که کمتر از ۹ رقم دارند، می‌توان آن‌ها را به اعدادی با دقیقا ۹ رقم تبدیل کرد. اگر 
$\overline{x_1x_2...x_9}$
عدد مورد نظر باشد، داریم: 
$$x_1 + x_2 + ... + x_9 = 32$$
$$0 \leq x_i \leq 9$$
\p
اگر تعداد کل جواب‌های این معادله را بشماریم، در بعضی جواب‌ها به ازای مقادیر $i$
 خواهیم داشت
  $x_i \geq 10$
 که نامطلوب هستند و باید از کل حالات کم شوند. اما نکته‌ای که وجود دارد این است که تعداد چنین 
$x_i$
هایی  
حداکثر ۳ تا می‌تواند باشد(در غیر این صورت مجموع ارقام بزرگتر یا مساوی ۴۰ می‌شود که خلاف فرض سوال است).
\p
فرض کنید دقیقا
$k$
رقم داشته باشیم که مقدار بزرگتر یا مساوی ۱۰ گرفته‌اند.
همچنین فرض کنید
$r(k)$
برابر تعداد جواب‌های معادله فوق باشد؛ به شرطی که 
$k$
تا از 
$x_i$ها مقدار بزرگتر یا مساوی ۱۰ گرفته باشند. 
می‌دانیم تعداد جواب‌های چنین معادله‌ای برابر است با تعداد جواب‌های معادله زیر که 
$\binom{32 - 10k + 9 - 1}{9 - 1}$
می‌باشد:
$$y_1 + y_2 + ... + y_9 = 32 - 10k$$ 
$$y_i \geq 0$$
با توجه به اینکه
برای انتخاب این 
$k$
رقم،
$\binom{9}{k}$
حالت 
داریم:
$$r(k) = \binom{9}{k}\binom{32 - 10k + 9 - 1}{9 - 1}$$
\p
طبق اصل شمول و عدم شمول، پاسخ مسئله به صورت زیر محاسبه می‌شود:
$$r(0) - r(1) + r(2) - r(3)$$
$$= \binom{9}{0}\binom{40}{8} - \binom{9}{1}\binom{30}{8} + \binom{9}{2}\binom{20}{8} - \binom{9}{3}\binom{10}{8}$$
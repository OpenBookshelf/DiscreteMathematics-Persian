\p
این مسئله از سه بخش مجزا تشکیل شده است که جواب نهایی آن طبق اصل ضرب، برابر حاصل‌ضرب جواب هر بخش می‌باشد.
\begin{itemize}
\item 
به
$\underline{\binom{m}{r}}$
روش $r$ حرف متمایز را انتخاب می‌کنیم.

\item 
برای محاسبه‌ی حالت‌های قرارگیری 
$n$
حرف در
$t$
سطر، معادله سیاله زیر را در مجموعه اعداد طبیعی حل می‌کنیم: 
$$r_1 + r_2 + \ldots + r_t = n$$
این معادله
$\underline{\binom{n - 1}{t - 1}}$
جواب دارد.

\item
برای محاسبه‌ی حالت‌های تعدادِ تکرارِ حروف، از تابع مولد کمک می‌گیریم.
اگر یک حرف، زوج تکرار داشته باشد تابع مولد آن به شکل زیر است:
\[ \sum_{n=0}^{+\infty} \frac{x^{2n}}{(2n)!} = \frac{e^x + e^{-x}}{2} \]
اگر یک حرف، فرد تکرار داشته باشد، تابع مولد آن به شکل زیر است:
\[ \sum_{n=0}^{+\infty} \frac{x^{2n + 1}}{(2n + 1)!} = \frac{e^x - e^{-x}}{2} \]
\end{itemize}
\p
 جواب نهایی این بخش، ضریب
$\frac{x^n}{n!}$
در عبارت زیر می‌باشد.

\[ \binom{r}{s}\times\left({\frac{e^x + e^{-x}}{2}}\right)^s\times\left({\frac{e^x - e^{-x}}{2}}\right)^{r - s} \]
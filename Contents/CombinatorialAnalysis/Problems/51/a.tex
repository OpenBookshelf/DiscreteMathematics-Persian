\begin{enumerate}

    \item
    -
    اگر در هر مرحله 3 نفر را انتخاب و وارد یک اتاق کنیم، تعداد کل حالات برابر می‌شود با: 
    $$\binom{15}{3}\binom{12}{3}\binom{9}{3}\binom{6}{3}$$

    -
    تعداد حالاتی که دکترها یا مهندس‌ها یا معلم‌ها به 5 اتاق متفاوت بروند برابر است با
    $$3\times5!\times\binom{10}{2}\binom{8}{2}\binom{6}{2}\binom{4}{2}$$
    (یکی از شغل‌ها را انتخاب و با !5 حالت 5 نفر دارای آن شغل را جایگشت می‌دهیم سپس باقی افراد را دو‌تا دو‌تا وارد اتاق‌ها می‌کنیم).

    -
    تعداد حالاتی که دقیقا 2 دسته از افراد به 5 اتاق متفاوت بروند برابر است با
    $$3\times5!\times5!\times5!$$
    (دو تا از شغل‌ها را انتخاب و به افراد هر دسته به !5 روش اتاق اختصاص می‌دهیم سپس 5 نفر باقی‌مانده را به !5 روش وارد اتاق‌ها می‌کنیم).

    -
    و در نهایت تعداد حالاتی که هر 3 دسته از افراد به 5 اتاق متفاوت بروند برابر
    است با
    $$5!\times5!\times5!$$
    (به هر دسته به !5 روش اتاق‌های متفاوت اختصاص داده‌ایم).

    با استفاده از اصل شمول و عدم شمول جواب نهایی برابر می‌شود با:
    $$\binom{15}{3}\binom{12}{3}\binom{9}{3}\binom{6}{3}-3\times5!\times\binom{10}{2}\binom{8}{2}\binom{6}{2}\binom{4}{2}+$$
    $$3\times5!\times5!\times5!-5!\times5!\times5!$$

    \item
    -
    به
    $$3\times\binom{5}{1}\binom{5}{3}\binom{12}{3}\binom{9}{3}\binom{6}{3}$$
    حالت یک دسته از 3 شغل را انتخاب و 3 نفر از آن دسته از افراد را در یک اتاق مشخص قرار می‌دهیم و سپس بقیه افراد را سه تا سه تا در اتاق‌ها افراز می‌کنیم.

    -
    در 
    $$3\times\binom{5}{1}\binom{5}{3}\binom{4}{1}\binom{5}{3}\binom{9}{3}\binom{6}{3}$$
    حالت حداقل در 2 اتاق مشخص شغل همه افراد یکسان است که باید آن را کم کنیم (به
    $\binom{3}{2}$
    روش دو اتاق مشخص را انتخاب کرده و سپس به 
    $\binom{5}{1}\times\binom{5}{3}$
    روش اتاق اول را انتخاب و 3 نفر با شغل یکسان را داخل آن قرار می‌دهیم. مشابه چنین کاری را برای اتاق دوم تکرار کرده و مابقی افراد را بصورت سه تا سه تا وارد اتاق‌های باقی مانده می‌کنیم).

    -
    و در نهایت حالتی که شغل افراد در 3 اتاق یکسان است (مشابه حالت پیشین اما این دفعه باید 3 مرتبه یک اتاق را انتخاب و افراد با شغل یکسان را داخل آن قرار دهیم) بیشتر از حد لازم کم شده است که آن را به جواب نهایی اضافه می‌کنیم.
    
    پس پاسخ نهایی برابر است با
    $$3\times\binom{5}{1}\binom{5}{3}\binom{12}{3}\binom{9}{3}\binom{6}{3}-$$
    $$3\times\binom{5}{1}\binom{5}{3}\binom{4}{1}\binom{5}{3}\binom{9}{3}\binom{6}{3}+$$
    $$\binom{5}{1}\binom{5}{3}\binom{4}{1}\binom{5}{3}\binom{3}{1}\binom{5}{3}\binom{6}{3}$$

    \item
    -
    به
    $$3\binom{5}{1}^3\binom{12}{3}\binom{9}{3}\binom{6}{3}$$
    حالت می‌توان در یک اتاق مشخص 3 نفر با شغل‌های متفاوت قرار داد (به 3 حالت یکی از 3 اتاق  ،A B و C را انتخاب می‌کنیم؛ از هر شغل به 5 روش یک نفر را وارد آن اتاق می‌کنیم و در‌نهایت بقیه افراد را سه تا سه تا وارد سایر اتاق‌ها می‌کنیم).

    -
    در
    $$3\binom{5}{1}^3\binom{4}{1}^3\binom{9}{3}\binom{6}{3}$$
    حالت در حداقل 2 اتاق 5 نفر با شغل‌های متفاوت قرار دارند (مانند توضیحات پیشین ابتدا دو اتاق مشخص را انتخاب کرده و افراد با شغل متفاوت را داخل آن‌ها قرار می‌دهیم سپس سایر افراد را سه تا سه تا وارد اتاق‌های خالی می‌کنیم). که باید از جواب اصلی کم شود.
    
    -
    در نهایت حالتی که در 3 اتاق شغل تمامی افراد متفاوت است (برای اتاق اول از هر شغل 5 انتخاب وجود دارد، برای اتاق دوم از هر شغل 4 انتخاب وجود دارد و برای اتاق سوم از هر شغل 3 انتخاب وجود دارد؛ در نهایت 3 نفر از 6 نفر باقی‌مانده را وارد اتاق D می‌کنیم و 3 نفر آخر به ناچار وارد اتاق E می‌شوند) بیشتر از حد لازم کم شده است که آن را به جواب نهایی اضافه می‌کنیم.
    
    پس پاسخ نهایی برابر است با:
    $$3\binom{5}{1}^3\binom{12}{3}\binom{9}{3}\binom{6}{3}-$$
    $$3\binom{5}{1}^3\binom{4}{1}^3\binom{9}{3}\binom{6}{3}+$$
    $$\binom{5}{1}^3\binom{4}{1}^3\binom{3}{1}^3\binom{6}{3}$$
\end{enumerate}
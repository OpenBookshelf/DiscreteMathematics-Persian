\begin{enumerate}
    \item
  سه تا 1 را به 
	\underline{$3!$ حالت}
	 می‌توان در سطر و ستون‌های مختلف چید
	(سطر اول 3 حالت، سطر دوم 2 حالت و سطر سوم 1 حالت).
	حال عدد 2 را به 
   \underline{2 حالت}
  می‌توان در سطر اول قرار داد:
  \vspace*{+0.4cm}
	 \begin{center}
	\begin{tabular}{ |c|c|c| } 
     \hline
       & 1 & 2  \\ 
     \hline
       &   & 1 \\ 
     \hline
     1 &  &   \\ 
     \hline
    \end{tabular}
    \end{center}
    \vspace*{+0.4cm}
     ‌دقت کنید از اینجا به بعد تنها یک انتخاب برای هر یک از خانه‌های باقی‌مانده داریم.
      مثلا جدول بالا تنها به صورت زیر پر می‌شود:
    
    \begin{center}
	\begin{tabular}{ |c|c|c| } 
     \hline
      3 & 1 & 2 \\ 
     \hline
      2 & 3 & 1 \\ 
     \hline
      1 & 2 & 3 \\ 
     \hline
    \end{tabular}
    \end{center}
    \vspace*{+0.4cm}
     در نتیجه جواب این بخش برابر
    $3! \times 2$ 
    است.
    
    \item
    اگر خانه وسط جدول 1 باشد، دو 1 دیگر را به 
    \underline{$2!$ حالت}
    می‌توان در سطر اول و سوم قرار داد. بقیه‌ی خانه‌های جدول مانند حالت قبل به صورت یکتا پر می‌شود. پس جواب این بخش 
    برابر
    $2!$
    است.
    \end{enumerate}
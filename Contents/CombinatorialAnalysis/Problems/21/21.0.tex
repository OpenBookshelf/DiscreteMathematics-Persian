\p
مدیرعاملی را در نظر بگیرید که دو کارمند دارد و قصد دارد زمانی که این دو کارمند در اختیار شرکت قرار می‌دهند را بین کار‌های شرکت تقسیم کند. شرکت دارای 5 کار می‌باشد:

1- رسیدگی به امور مخاطبین که برای پایداری شرکت باید حداقل دو ساعت برای آن وقت صرف شود و کارمند دوم قادر به انجام آن نیست.

2- طراحی و پخش تبلیغات که طراحی را فقط کارمند اول و پخش را فقط کارمند دوم می‌تواند انجام دهد و زمان مورد نیاز برای پخش تبلیغات،دو برابر زمان طراحی آن‌ها است.

3و4و5 - دیگر امور شرکت که کارمند اول حاضر به انجام آن‌ها نیست.
این کارها می‌توانند توسط هيچ يك از دو كارمند انجام نشوند.
\p
مشخص کنید مدیر عامل به چند طریق می‌تواند کارمندانش را کنترل کند اگر کارمند‌ها به ترتیب 6 و 9 ساعت در اختیار باشند و تقسیم زمان با واحد ساعت انجام شود.
توجه کنید که هر کارمند باید ساعات کاری خود را کامل با این ۵ کار پوشش دهد و نمی‌تواند کمتر از ساعت مذکور مشغول باشد.

\p
  دنباله 
  $a_n$ 
  را تعداد حالت‌های ممکن برای انتخاب این
  $n$
  توپ
  در نظر می‌گیریم.
  ضریب جمله
  $x^n$
  در تابع مولد
   دنباله‌ی
   $a_n$،
  ‌جمله‌ی 
  $n$ام
  این دنباله و پاسخ مسئله است.
  برای به دست آوردن تابع مولد دنباله‌ی
  $a_n$،
  ابتدا برای هر رنگ، تابع مولد می‌نویسیم که تعداد حالت‌های انتخاب‌ توپ‌های از یک رنگ را می‌شمارد. سپس طبق اصل ضرب،
  تابع مولد نهایی برابر حاصل‌ضرب تمامی توابع مولد به دست‌آمده خواهد بود.        
\begin{itemize}
    \item 
      تعداد توپ‌های آبی باید زوج باشد:
        $$\text{تابع مولد توپ‌های آبی} = G_a(x) = 1 + x^2 + x^4 + ... = \frac{1}{1 - x^2}$$
    \item 
    تعداد توپ‌‌های قرمز باید مضربی از ۷ باشد:
            $$\text{تابع مولد توپ‌های قرمز} = G_b(x) = 1 + x^7 + x^{14} + ... = \frac{1}{1 - x^7}$$

    \item 
    تعداد توپ‌های سبز باید ۰  یا ۱ باشد:
            $$\text{تابع مولد توپ‌های سبز} = G_c(x) = 1 + x$$

    \item 
    تعداد توپ‌های زرد باید حداکثر ۶ باشد:
            $$\text{تابع مولد توپ‌های زرد} = G_d(x) = 1 + x + x^2 + x^3 + x^4 + x^5 + x^6 = \frac{1 - x^7}{1 - x}$$
\end{itemize}

  \p
  همان‌طور که بیان شد، تابع مولد 
  $a_n$
  برابر حاصل‌ضرب توابع مولد فوق می‌باشد. در نتیجه داریم:
  \begin{align*}
    \text{تابع مولد $a_n$} = G(x) &= G_a(x)G_b(x)G_c(x)G_d(x)\\
    &= (\frac{1}{1 - x^2})(\frac{1}{1 - x^7})(\frac{1 + x}{1})(\frac{1 - x^7}{1 - x})\\
    &= (\frac{1}{(1 - x)(1 + x)})(\frac{1 + x}{1})(\frac{1}{1 - x})\\
    &= \frac{1}{(1 - x)^2} = 1 + 2x + 3x^2 + ...  
  \end{align*}
  \p
  بنابراین پاسخ مسئله برابر 
  $n + 1$
  است
 که ضریب جمله 
 $x^n$ 
  می‌باشد.
\p
این مسئله از سه بخش مجزا تشکیل شده است که جواب نهایی آن طبق اصل ضرب، برابر حاصل‌ضرب جواب هر بخش می‌باشد.
\begin{itemize}
\item 
به
$\underline{\binom{m}{r}}$
روش $r$ حرف متمایز را انتخاب می‌کنیم.

\item 
برای محاسبه‌ی حالت‌های قرارگیری 
$n$
حرف در
$t$
سطر، معادله سیاله زیر را در مجموعه اعداد طبیعی در نظر می‌گیریم: 
$$r_1 + r_2 + \ldots + r_t = n$$
طبق 
\CROSSREF{محدودیت حداقلی متغیرها در معادله سیاله خطی با ضرایب واحد}،
این معادله
$\underline{\binom{n - 1}{t - 1}}$
جواب دارد.

\item
برای محاسبه‌ی حالت‌های تعدادِ تکرارِ حروف، از تابع مولد کمک می‌گیریم.
اگر یک حرف زوج تکرار داشته باشد، تابع مولد آن به شکل زیر است:
\[ \sum_{n=0}^{+\infty} \frac{x^{2n}}{(2n)!} = \frac{e^x + e^{-x}}{2} \quad (1) \]
اگر یک حرف فرد تکرار داشته باشد، تابع مولد آن به شکل زیر است:
\[ \sum_{n=0}^{+\infty} \frac{x^{2n + 1}}{(2n + 1)!} = \frac{e^x - e^{-x}}{2} \quad (2) \]
نیاز است به 
$\binom{r}{s}$
حالت، 
$s$
حرف را انتخاب کرده تا تعدادِ تکرارِ زوج داشته باشند.
در این صورت تابع مولد
$(1)$
به توان 
$s$
و تابع مولد
$(2)$
به توان 
$r-s$
می‌رسند.
 جواب نهایی این بخش، ضریب
$\frac{x^n}{n!}$
در عبارت زیر می‌باشد:
\[ \binom{r}{s}\times\left({\frac{e^x + e^{-x}}{2}}\right)^s\times\left({\frac{e^x - e^{-x}}{2}}\right)^{r - s} \]
\end{itemize} 
\p
طبق اصل ضرب پاسخ مسئله به صورت زیر است:
$$\binom{m}{r} \times \binom{n - 1}{t - 1} \times \binom{r}{s}\times\left({\frac{e^x + e^{-x}}{2}}\right)^s\times\left({\frac{e^x - e^{-x}}{2}}\right)^{r - s}  $$

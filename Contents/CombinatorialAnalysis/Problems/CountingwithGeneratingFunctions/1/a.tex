\p
 ابتدا برای هر کدام از شروط، تابع مولد می‌نویسیم. طبق اصل ضرب،
  تابع مولد نهایی برابر حاصل‌ضرب تمامی  توابع مولد به دست‌آمده خواهد بود؛
    زیرا هر یک از توابع مولد به دست‌آمده برای شروط، تعداد حالت‌های انتخاب‌ تیله‌های از یک رنگ را می‌شمارد.  
   طبق این توضیحات به بررسی هر یک از شروط موجود در صورت سوال می‌پردازیم.
\begin{itemize}
    \item 
    تیله‌های آبی باید زوج باشند:
    $$\text{تابع مولد تیله‌های آبی} = G_a(x) = 1 + x^2 + x^4 + \ldots = \frac{1}{1-x^2}$$
    \item
    تیله‌های زرشکی باید مضرب 13 باشند:
    $$\text{تابع مولد تیله‌های زرشکی} = G_b(x) = 1 + x^{13} + x^{26} + \ldots = \frac{1}{1-x^{13}}$$
    \item
    تیله‌های سبز باید کوچک‌تر مساوی 7 باشند:
    $$\text{تابع مولد تیله‌های سبز} = G_c(x) = 1 + x + x^2 + \ldots + x^7 = \frac{1-x^8}{1-x}$$
    \item
    تیله‌های بنفش باید از سه بیشتر باشند:
    \begin{align*}
    \text{تابع مولد تیله‌های بنفش} &= G_d(x) = x^4 + x^5 + x^6 + \ldots\\
    &= (1+x+x^2+x^3+\ldots) - (1 + x + x^2 + x^3)\\
    &= (\frac{1}{1-x}) - (\frac{1-x^4}{1-x}) = \frac{x^4}{1-x}
    \end{align*}
\end{itemize}
\p
  همان‌گونه که بیان شد، تابع مولد
$S_n$
حاصل‌ضرب توابع مولد فوق می‌باشد. پس در نهایت خواهیم داشت:
$$\text{تابع مولد $S_n$} = G(x) = \frac{x^4(1-x^8)}{(1-x)^2(1-x^{13})(1-x^{2})}$$
\p
ابتدا 
$4$
مهره سبز، بنفش، نقره‌ای و آبی را بدون اعمال هیچ شرط خاصی، درون یک حلقه می‌چینیم. تعداد حالات مربوط به آن با توجه به تعداد جایگشت‌ها‌ی دوری، برابر با
\underline{3!}
است.
حال از
$4$
  جای بین آن‌ها 
$3$
   جا را برای 
$3$
    مهره باقی‌مانده که نباید مجاور هم باشند، به
    \underline{$\frac{4!}{1!}$ حالت}
    انتخاب می‌کنیم
     (ترتیب).
      همچنین نگاه به دست‌بند از
$2$
    سوی متفاوت حالت جدیدی را به وجود نمی‌آورد، پس حاصل را بر 
$2$
    تقسیم می‌کنیم:
    $$\frac{3!\times \frac{4!}{1!}}{2}$$
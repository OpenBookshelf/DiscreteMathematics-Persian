\p
برای پاسخ به این سوال از اصل شمول و عدم شمول استفاده می‌کنیم.
\begin{itemize}
    \item 
    فرض کنید تعداد کل حالاتِ تقسیم کردن این افراد در اتاق‌ها
   $U$
   باشد.
   در هر مرحله 3 نفر را انتخاب و وارد یک اتاق می‌کنیم:
    $$U = \binom{15}{3}\binom{12}{3}\binom{9}{3}\binom{6}{3}\binom{3}{3}$$

    \item 
    تعداد حالاتی که دکترها به 5 اتاق متفاوت بروند را 
    $n_1$،
    مهندس‌ها به 5 اتاق متفاوت بروند را
    $n_2$
    و معلم‌ها به 5 اتاق متفاوت بروند را 
    $n_3$
    در نظر می‌گیریم. 
    اکنون تعداد حالاتی که دکترها یا مهندس‌ها یا معلم‌ها به 5 اتاق متفاوت بروند را به صورت زیر به دست می‌آوریم:
    $$|n_1| + |n_2| + |n_3| = \underbrace{\binom{3}{1}\times5!}_{\text{آ}}\times\underbrace{\binom{10}{2}\binom{8}{2}\binom{6}{2}\binom{4}{2}\binom{2}{2}}_{\text{ب}}$$
    \begin{enumerate}
     \item 
    یکی از شغل‌ها را انتخاب و با !5 حالت 5 نفر دارای آن شغل را جایگشت می‌دهیم.
    \item  
    باقی افراد را دو‌تا دو‌تا وارد اتاق‌ها می‌کنیم.
    \end{enumerate}
    \item 
    سپس تعداد حالاتی که دقیقا 2 دسته از افراد به 5 اتاق متفاوت بروند را محاسبه می‌کنیم:
    $$|n_1 \cap n_2| + |n_1 \cap n_3| + |n_2 \cap n_3| = \underbrace{\binom{3}{2}\times5!\times5!}_{\text{آ}}\times \underbrace{5!}_{\text{ب}}$$
    \begin{enumerate}
     \item
    دو تا از شغل‌ها را انتخاب و به افراد هر دسته به !5 روش اتاق اختصاص می‌دهیم.
    \item
      5 نفر باقی‌مانده را به !5 روش وارد اتاق‌ها می‌کنیم.
    \end{enumerate}
    \item 
     در نهایت تعداد حالاتی که هر 3 دسته از افراد به 5 اتاق متفاوت بروند برابر
    است با:
    $$|n_1 \cap n_2 \cap n_3| = 5!\times5!\times5!$$
    به هر دسته به !5 روش اتاق‌های متفاوت اختصاص داده‌ایم.
\end{itemize}
در این صورت جواب نهایی مسئله با استفاده از اصل شمول و عدم شمول برابر می‌شود با:
  \begin{align*}
    |\overline{n_1 \cup n_2 \cup n_3}| = U - |n_1 \cup n_2 \cup n_3|
    &= \binom{15}{3}\binom{12}{3}\binom{9}{3}\binom{6}{3}\binom{3}{3}\\ 
     &- 3\times5!\times\binom{10}{2}\binom{8}{2}\binom{6}{2}\binom{4}{2}\binom{2}{2}\\
     &+ 3\times5!\times5!\times5!-5!\times5!\times5! 
  \end{align*}

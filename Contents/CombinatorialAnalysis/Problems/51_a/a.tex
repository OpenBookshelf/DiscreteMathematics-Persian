\p
\begin{itemize}
    \item 
    اگر در هر مرحله 3 نفر را انتخاب و وارد یک اتاق کنیم، تعداد کل حالات برابر می‌شود با: 
    $$\binom{15}{3}\binom{12}{3}\binom{9}{3}\binom{6}{3}$$

    \item 
    تعداد حالاتی که دکترها یا مهندس‌ها یا معلم‌ها به 5 اتاق متفاوت بروند برابر است با:
    $$\underbrace{\binom{3}{1}\times5!}_{\text{آ}}\times\underbrace{\binom{10}{2}\binom{8}{2}\binom{6}{2}\binom{4}{2}}_{\text{ب}}$$
    \begin{enumerate}
     \item 
    یکی از شغل‌ها را انتخاب و با !5 حالت 5 نفر دارای آن شغل را جایگشت می‌دهیم.
    \item  
    باقی افراد را دو‌تا دو‌تا وارد اتاق‌ها می‌کنیم.
    \end{enumerate}
    \item 
    تعداد حالاتی که دقیقا 2 دسته از افراد به 5 اتاق متفاوت بروند برابر است با:
    $$\underbrace{\binom{3}{2}\times5!\times5!}_{\text{آ}}\times \underbrace{5!}_{\text{ب}}$$
    \begin{enumerate}
     \item
    دو تا از شغل‌ها را انتخاب و به افراد هر دسته به !5 روش اتاق اختصاص می‌دهیم.
    \item
      5 نفر باقی‌مانده را به !5 روش وارد اتاق‌ها می‌کنیم.
    \end{enumerate}
    \item 
     در نهایت تعداد حالاتی که هر 3 دسته از افراد به 5 اتاق متفاوت بروند برابر
    است با:
    $$5!\times5!\times5!$$
    به هر دسته به !5 روش اتاق‌های متفاوت اختصاص داده‌ایم.
\end{itemize}
    با استفاده از اصل شمول و عدم شمول جواب نهایی برابر می‌شود با:
    $$\binom{15}{3}\binom{12}{3}\binom{9}{3}\binom{6}{3}-3\times5!\times\binom{10}{2}\binom{8}{2}\binom{6}{2}\binom{4}{2}+$$
    $$3\times5!\times5!\times5!-5!\times5!\times5!$$

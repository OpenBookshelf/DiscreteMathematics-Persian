\p
برای پاسخ به این سوال از اصل شمول و عدم شمول استفاده می‌کنیم.
\begin{itemize}
    \item 
    تعداد حالاتی که دکترها در یک اتاق باشند را
    $n_1$،
    تعداد حالاتی که مهندس‌ها
     در یک اتاق باشند را
     $n_2$
    و تعداد حالاتی که معلم‌ها در یک اتاق باشند را
    $n_3$
    می‌نامیم.
    در این صورت داریم:
    $$|n_1| + |n_2| + |n_3| = \underbrace{\binom{3}{1}\binom{5}{1}}_{\text{آ}} \underbrace{\binom{5}{3}\binom{12}{3}\binom{9}{3}\binom{6}{3}\binom{3}{3}}_{\text{ب}}$$
    \begin{enumerate}
        \item  
         یک دسته از 3 شغل را انتخاب و 3 نفر از آن دسته از افراد را در یک اتاق مشخص قرار می‌دهیم.
        \item  
          بقیه افراد را سه تا سه تا در اتاق‌ها افراز می‌کنیم.
    \end{enumerate}

    \item 
   تعدادِ حالاتی که شغل افرادِ حداقل 2 اتاق مشخص، یکسان باشد برابر است با: 
   $$|n_1 \cap n_2| + |n_1 \cap n_3| + |n_2 \cap n_3| = \underbrace{\binom{3}{2}}_{\text{آ}}\underbrace{\binom{5}{1}\binom{5}{3}\binom{4}{1}\binom{5}{3}}_{\text{ب}}\underbrace{\binom{9}{3}\binom{6}{3}\binom{3}{3}}_{\text{ج}}$$
   \begin{enumerate}
       \item
       دو شغل را انتخاب می‌کنیم. 
       \item
       اتاق اول را انتخاب و 3 نفر با شغل یکسان را داخل آن قرار می‌دهیم. مشابه چنین کاری را برای اتاق دوم انجام می‌دهیم.
       \item 
       مابقی افراد را سه تا سه تا وارد اتاق‌های دیگر می‌کنیم.
\end{enumerate}

    
        \item 
        در نهایت حالتی که شغل افراد در 3 اتاق یکسان باشد 
        برابر است با:
        $$|n_1 \cap n_2 \cap n_3| = \underbrace{\binom{5}{1}\binom{5}{3}}_{\text{آ}}\underbrace{\binom{4}{1}\binom{5}{3}}_{\text{ب}}\underbrace{\binom{3}{1}\binom{5}{3}}_{\text{ج}}\underbrace{\binom{6}{3}\binom{3}{3}}_{\text{د}}$$
        \begin{enumerate}
        \item 
          اتاق اول را انتخاب و 3 نفر با شغل یکسان را داخل آن قرار می‌دهیم.
          \item
          اتاق دوم را انتخاب و 3 نفر با شغل یکسان را داخل آن قرار می‌دهیم.
          \item
          اتاق سوم را انتخاب و 3 نفر با شغل یکسان را داخل آن قرار می‌دهیم.
          \item
          باقی افراد را وارد اتاق‌های دیگر می‌کنیم.
    \end{enumerate}
\end{itemize} 
    در نتیجه جواب نهایی مسئله با استفاده از اصل شمول و عدم شمول برابر است با:
    \begin{align*}
     |n_1 \cup n_2 \cup n_3| &= \binom{3}{1}\binom{5}{1}\binom{5}{3}\binom{12}{3}\binom{9}{3}\binom{6}{3}\binom{3}{3}\\
    &- \binom{3}{2}\binom{5}{1}\binom{5}{3}\binom{4}{1}\binom{5}{3}\binom{9}{3}\binom{6}{3}\binom{3}{3}\\
    &+ \binom{5}{1}\binom{5}{3}\binom{4}{1}\binom{5}{3}\binom{3}{1}\binom{5}{3}\binom{6}{3}\binom{3}{3}
    \end{align*}  
\p
از دوگانه‌شماری استفاده می‌کنیم.
فرض کنید می‌خواهیم به دو روش  
 $n$
 دانشجو را در سه کلاس دسته بندی کنیم.

\p
 - روش اول:\[\underbrace{n}_{1}\times\underbrace{4^{n-1}}_{2}\] 
 به $n$ 
 روش نماینده را انتخاب می‌کنیم.(1)

  سپس نماینده، هر یک از 
 $n-1$ 
 نفر دیگر را به 4 طریق در کلاس‌ها دسته‌بندی می‌کند 
(1 حالت، برای کلاس نداشتن و 3 حالت، برای کلاس داشتنِ هر دانشجو وجود دارد).
(2)

\p
- روش دوم:\[\underbrace{\sum\limits_{k=0}^{n}}_{4}\underbrace{\binom{n}{k+1}}_{1}\underbrace{(k+1)}_{2}\times\underbrace{3^{k}}_{3}\] 
 تعداد دانشجویانی که به کلاس می‌روند را با 
 $k$ نشان می‌دهیم.  
  $k+1$
  نفر را انتخاب می‌کنیم.(1)
  
   سپس به 
   $k+1$
  حالت، یکی از آن‌ها را به عنوان نماینده بر می‌گزینیم.
  (2)
  
   $k$
   نفر دیگر را به
    $3^{k}$ 
    حالت به یکی از سه کلاس می‌فرستیم.
    (3)
    
     طبق اصل جمع به ازای همه‌ی 
     $k$
     های ممکن داریم(4):    
 
  $$\sum\limits_{k=0}^{n}\binom{n}{k+1}(k+1)\times3^{k}$$
  $$= \binom{n}{1} + \binom{n}{2} 2\times3 + \binom{n}{3} 3\times3^2 +\binom{n}{4} 4\times3^3 +\ldots +\binom{n}{n} n\times3^{n-1}$$ 
 
\p
با توجه به اینکه هر دو روش مسئله یکسانی را پاسخ داده‌اند طبق دوگانه‌شماری داریم:

$$\sum\limits_{k=0}^{n}\binom{n}{k+1}(k+1)\times3^{k} = n\times4^{n-1}$$

\p
از اثبات به روش ترکیبیاتی کمک می‌گیریم.
فرض کنید n دانش آموز داریم. می‌خواهیم از بین آن‌ها یک نماینده انتخاب کنیم. سپس نماینده برخی از دانشجویان را در سه کلاس دسته بندی می‌کند و به سایر دانش آموزان کلاسی را اختصاص نمی‌دهد.

\p
سمت راست: به n روش نماینده را انتخاب می‌کنیم، سپس نماینده هر نفر را به 4 طریق در کلاس‌ها دسته‌بندی می‌کند(در 3 حالت به دانش آموز کلاس می‌رسد و در 1 حالت دانش آموز بدون کلاس می ماند)

\p
سمت چپ: اگر تعداد افرادی که به کلاس می‌روند را x بنامیم. روی x حالت‌بندی می‌کنیم.
اگر $x = 1$ باشد
آنگاه تنها می‌توانیم نماینده را به 
$\binom{n}{1}$
انتخاب کنیم. اگر $x = 2$ باشد آنگاه به
$\binom{n}{2}$
حالت 2 نفری که به کلاس خواهند رفت را انتخاب می‌کنیم. سپس به 2 حالت یکی از آن‌ها را به عنوان نماینده بر می‌گزینیم و نفر دیگری را به 3 حالت به کلاس می‌فرستیم.
اگر $x = 3$ آنگاه به
$\binom{n}{3}$
حالت 3 نفری که به کلاس خواهند رفت را انتخاب می‌کنیم. سپس به 3 حالت یکی از آنها را به عنوان نماینده بر می‌گزینیم و دو نفر باقیمانده را به 
$3^2$
حالت به کلاس می‌فرستیم.
اگر همینطور ادامه دهیم جواب نهایی برابر می‌شود با:
$$\binom{n}{1} + 2\times3\binom{n}{2} + 3\times3^2\binom{n}{3} + 4\times3^3\binom{n}{4} +\ldots +
n\times3^{n-1}\binom{n}{n}$$

\p
پس توانستیم با روش دوگانه شماری، تساوی دو طرف حکم مسئله را نشان دهیم.
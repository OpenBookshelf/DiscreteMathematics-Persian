    \begin{enumerate}
        \item 
        \p
        هر دو طرف تساوی تعداد حالات انتخاب
        $2n$
        نفر از بین
        $2n$
        زوج را می‌شمارند.
        
        \p
        یک حالت شمارش، انتخاب $2n$ نفر از بین $2n$ زوج یا همان $4n$ نفر است : $\binom{4n}{2n}$
        
        \p
        اما حالت دیگر، حالت بندی روی تعداد مردانی است که همراه همسرشان انتخاب می‌شوند :
        $$\underbrace{\sum\limits_{i=0}^{n}}_1 \underbrace{2^{2n-2i}}_3  \underbrace{\binom{2n}{2n-2i}}_2 \underbrace{\binom{2i}{i}}_4$$
        
        در این حالت ابتدا مردانی را در نظر می‌گیریم که یا فقط خودشان یا فقط همسرشان انتخاب می‌شوند و تعداد باقی‌مانده اختصاص به مردانی دارند که همراه همسرشان انتخاب می‌شود.
        
        \p
        طبق توضیحات فوق، برای انتخاب 
        $2n$
        نفر از بین
        $2n$
        زوج، متغیر
        $i$
        نشان‌دهنده‌ی
        تعداد مردانی است که همراه همسرشان انتخاب می‌شوند و از 0 تا $n$ متغیر است. $(1)$
        
        -
        ابتدا $2n-2i$ زوج از بین $2n$ زوج بر می‌داریم. $(2)$
        
        -
        سپس از هر کدام از $2n-2i$ زوج برداشته‌شده، زن یا مرد را انتخاب می‌کنیم (هر زوج دو حالت دارد). $(3)$
        
        -
        در آخر از $2i$ زوج باقی‌مانده $i$ زوج را انتخاب می‌کنیم ($i$ مرد به همراه همسرانشان). $(4)$
        
        \p
        بدین ترتیب در نهایت داریم:
        $$\sum\limits_{i=0}^{n} 2^{2n-2i} \binom{2n}{2n-2i} \binom{2i}{i} = \binom{4n}{2n}$$
        \item
    
        فرض کنید
        $P$
        بیانگر تعداد راه‌های انتخاب یک انجمن علمی
        $n$
        نفره از بین
        $n$
        دانشجوی کامپیوتر و 
        $n$
        دانشجوی برق و انتخاب یک دانشجوی کامپیوتر به عنوان مدیر انجمن است. این کار به دو روش امکان‌پذیر است:
        
        ابتدا یکی از دانشجو‌های کامپیوتر را به عنوان مدیر انجمن انتخاب می‌کنیم که به 
        $n$
        حالت انجام می‌شود.$(1)$
        
        سپس 
        $n-1$
        عضو دیگر انجمن را از
        $2n-1$
        دانشجوی باقی‌مانده انتخاب می‌کنیم.$(2)$
        $$P = \underbrace{n}_1 \underbrace{\binom{2n-1}{n-1}}_2$$
        حالت دیگر آن است که ابتدا 
        $k$
        عضو انجمن را از دانشجو‌های کامپیوتر 
        انتخاب کنیم.$(1)$
        
        سپس
        $n-k$
        عضو دیگر را از دانشجو‌های برق انتخاب می‌کنیم.$(2)$
        
        در نهایت یکی از
        $k$
        دانشجوی کامپیوتر عضو انجمن را به عنوان مدیر انجمن انتخاب خواهیم کرد.$(3)$
        $$\underbrace{\binom{n}{k}}_1 \underbrace{\binom{n}{n-k}}_2\times \underbrace{k}_3 = k{\binom{n}{k}}^2$$
        حال بسته به این که
        $k$
        برابر 
        $1,2,\ldots$
        یا
        $n$
        باشد، داریم: 
        \begin{align*}
        P &= 1{\binom{n}{1}}^2 + 2{\binom{n}{2}}^2 + \ldots + n{\binom{n}{n}}^2\\
        &= \sum\limits_{i=0}^{n} i {\binom{n}{i}}^2
        \end{align*}
        
        لازم به ذکر است که از آنجاییکه که قرار است یکی از دانشجو‌های کامپیوتر به عنوان مدیر انتخاب شود، همواره شرط 
        $1\leq k$
        برقرار است.
        از تساوی دو حالت فوق حکم مسأله ثابت می‌شود:
        $$ \sum\limits_{i=0}^{n} i {\binom{n}{i}}^2 = n\binom{2n-1}{n-1}$$
    \end{enumerate}
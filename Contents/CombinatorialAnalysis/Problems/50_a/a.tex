\p
طبق رابطه‌ی تعداد جایگشت‌های خطی با اعضای تکراری، به 
$10!$
تمام ده حرف را در یک ردیف قرار می‌دهیم. سپس جواب را تقسیم بر جایگشت حروف تکراری که در اینجا دو تا $n$ و سه تا $t$ هستند، می‌کنیم.
 در نتیجه به 
\underline{$\frac{10!}{2!3!}$ حالت}
 می‌توان حروف غیر از
 $i$ را در یک ردیف قرار داد.

 قبل و بعد هر یک از این 10 حرف، 11 جایگاه برای گذاشتن
 $i$ها 
  وجود دارد که 3 تا از آن‌ها قبل از حروف 
  $t$
   و غیرقابل استفاده
  هستند.
پس لازم است سه حرف غیر متمایز
$i$
را
در 8 جایگاه متمایز قرار دهیم. 
طبق تعداد پاسخ‌های معادله سیاله خطی با ضرایب واحد،
پاسخ برابر
\underline{$\frac{10!}{7!3!}$}
می‌باشد.
در نتیجه پاسخ نهایی برابر است با:
$$\frac{10!}{2!3!}  \times \frac{10!}{7!3!}$$


\p
  دنباله <$a_n$> را در نظر بگیرید که پاسخ مسئله است. تابع مولد این دنباله را از ضرب کردن توابع مولد مربوط به رنگ هر توپ به دست می‌آوریم. (اگر از تابع مولد رنگی مشخص، جمله $x^i$ هنگام ضرب کردن برداشته شود، یعنی از آن رنگ مشخص $i$ توپ انتخاب شده است) 

  \p
بنابراین ضریب جمله $x^n$ در تابع مولد نهایی یا همان $a_n$ تعداد حالاتی را نشان می‌دهد که می‌توان $n$ توپ با شرایط صورت سوال انتخاب کرد.

        
\begin{itemize}
    \item 
        برای توپ‌های آبی داریم:
        $$B(x) = 1 + x^2 + x^4 + ... = \frac{1}{1 - x^2}$$
    \item 
    برای توپ‌های قرمز داریم:
            $$R(x) = 1 + x^7 + x^{14} + ... = \frac{1}{1 - x^7}$$

    \item 
    برای توپ‌های سبز داریم:
            $$G(x) = 1 + x$$

    \item 
        برای توپ‌های زرد داریم:
            $$Y(x) = 1 + x + x^2 + x^3 + x^4 + x^5 + x^6 = \frac{1 - x^7}{1 - x}$$
\end{itemize}

  \p
  از حاصل ضرب این توابع مولد داریم:
$$A(x) = B(x)R(x)G(x)Y(x)$$
$$= \frac{1}{1 - x^2}\frac{1}{1 - x^7}\frac{1 + x}{1}\frac{1 - x^7}{1 - x}$$
$$= \frac{1}{(1 - x)(1 + x)} \frac{1 + x}{1} \frac{1}{1 - x}$$
$$= \frac{1}{(1 - x)^2}$$
که می‌دانیم:
        $$A(x) = 1 + 2x + 3x^2 + ... $$
  \p
  بنابراین پاسخ برابر 
        $n + 1$
         که ضریب جمله $x^n$ است می‌باشد.
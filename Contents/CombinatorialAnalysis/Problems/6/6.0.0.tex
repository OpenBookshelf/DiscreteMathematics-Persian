	\begin{enumerate}
		\item 
        \p
        طبق رابطه‌ی ضرایب چندجمله‌ای برای هر عدد
         طبیعی 
         $p$
         و هر عدد مثبت 
         $m$ 
          داریم:
        $$(a_1+a_2+\ldots+a_m)^p = $$
        $$\sum_{p_1+p_2+\ldots+p_m=p} \frac{p!}{p_1!p_2!\ldots p_k!} a_1^{p_1}a_2^{p_2}\ldots a_k^{p_k} \quad (1)$$
        حال بر حسب چندجمله‌ای داده‌شده، عبارت فوق را تعریف می‌کنیم. برای ساده‌سازی فرض کنید در این تعریف توان جمله‌ی اول، یعنی
        $p_1$
        را برابر $n_1$ ،
        توان جمله‌ی دوم یعنی 
        $p_2$
        را برابر 
        $n_2$
        و توان جمله‌ی سوم یعنی
        $p_3$
        را برابر $n_3$ در نظر بگیریم.
        
        \begin{align*}
		(1+x+x^2)^n = \sum_{n_1+n_2+n_3=n} \frac{n!}{n_1!n_2!n_3!} a_1^{n_1} a_2^{n_2} a_3^{n_3}\\
		= \sum_{n_1+n_2+n_3=n} \frac{n!}{n_1!n_2!n_3!} 1^{n_1} x^{n_2} (x^2)^{n_3}\\
		= \sum_{n_1+n_2+n_3=n} \frac{n!}{n_1!n_2!n_3!} x^{n_2 + 2n_3}
        \end{align*}
		در نهایت چندجمله‌ای داده شده به فرم بالا ساده می‌شود.
		حال سه‌تایی
         $(u , v , w)$
         را در نظر می‌گیریم که در شرایط 
         ($n_1 , n_2 , n_3$)
       در تعریف فوق صدق کند:
       $$u + v + w = n$$
        اگر این سه‌تایی  جملهٔ $x^{n + k}$ تولید کند, در این صورت سه‌تایی $(w , v , u)$ جملهٔ $x^{n - k}$ را تولید می‌کند زیرا:
        $$v + 2u = 2(u + v + w) - (v + 2w) = 2n - (n + k) = n -k$$
        
        \p
        حال در جهت عکس ثابت می‌کنیم که یک سه‌تایی که جمله‌ی
        $x^{n-k}$
        را تولید می‌کند، می‌تواند جمله‌ی
        $x^{n+k}$
        را تولید کند. بدین منظور فرض کنید سه‌تایی
        $(w', v', u')$
        در شرط موجود در عبارت صدق کند
        $(w' + v' + u'=n)$
        و جمله‌ی
        $x^{n-k}$
        را تولید کند.
        در این صورت سه‌تایی
        $(u',v',w')$
        طبق رابطه‌ی زیر، جمله‌ی 
        $x^{n+k}$
        را تولید می‌کند:
    	$$v' + 2w' = 2(u' + v' + w') - (v' + 2u') = 2n - (n - k) = n +k$$
	
        \p
    	در ضمن ضریب جمله‌ی متناظر با سه‌تایی 
        $(u , v , w)$ 
        با ضریب جمله‌ی متناظر با سه‌تایی 
        $(w , v , u)$
         طبق رابطه‌ی (1) برابر است و لذا 
         $a_{n + k} = a_{n - k}$.
	
	\item
        \p
          با توجه به قسمت  
          «آ»
          داریم:
    	$$a_0 = a_{2n} \; ,\; a_1 = a_{2n - 1}\; ,\; a_2 = a_{2n - 2}\; ,\; \dots \; ,\; a_{n - 1} = a_{n + 1}$$
        در نتیجه می‌توان گفت:
        \begin{align*}
            &a_0a_1 - a_1a_2 + a_2a_3 - ... - a_{2n-1}a_{2n} \\
            = &a_{2n}a_{2n-1} - a_{2n-1}a_{2n-2} +\dots - a_1a_0 \\ 
            = &A
        \end{align*}
    	\begin{align*}
    		\Rightarrow 2A = &a_0a_1 - a_1a_2 + a_2a_3 - \dots - a_{2n-1}a_{2n}\\
    		&+ a_{2n}a_{2n-1} - a_{2n-1}a_{2n-2} +\dots - a_1a_0 \\
            = &0\\
    		\Rightarrow &A = 0
    	\end{align*}
	
	\item
        \p
    	با توجه به راه حل قسمت (الف), $a_k$ برابر مجموع اعداد به صورت $\frac{n!}{n_1!n_2!n_3!}$ است که در آن:
    	$$n_1 + n_2 + n_3 = n$$ 
    	$$n_2 + 2n_3 = k$$
    	 حال اگر $b_k$ را ضریب $x^k$ در بسط $(1 - x + x^2)^n$ بگیریم, در این صورت $b_k$ برابر مجموع اعداد $\frac{(-1)^{n_2}n!}{n_1!n_2!n_3!}$ است (طبق قضیه‌ی چندجمله‌ای  به دست می‌آید) که در آن:
    	 $$n_1 + n_2 + n_3 =  n$$ 
    	 $$n_2 + 2n_3 = k$$
    	 پس اگر $k$ زوج باشد, $n_2$ نیز زوج است (طبق قید فوق) و $b_k = a_k$ و اگر $k$ فرد باشد, $n_2$ نیز فرد است و $b_k = -a_k$.
    	 
        \p
        همچنین چون $a_k$ ضریب $x^k$ در بسط $(1 + x + x^2)^n$ است, پس ضریب $x^{2k}$ در بسط $(1 + x^2 + x^4)^n$ برابر $a_k$ است. زیرا:
         $$(1+x^2+x^4)^n=\sum_{n_1+n_2+n_3=n} \frac{n!}{n_1!n_2!n_3!} x^{2n_2+4n_3}$$
         بنابراین ضریب $a_{2k}$ برابر مجموع اعداد $\frac{n!}{n_1!n_2!n_3!}$
          است که در آن\\
           $n_1+n_2+n_3=n$ و $2n_2+4n_3=2k$
           که برابر $a_k$ است. حال تساوی
        $$(1 + x + x^2)^n(1 - x + x^2)^n = (1 + x^2 + x^4)^n$$
        را در نظر بگیرید. ضریب $x^{2n}$ در سمت راست برابر $a_n$ و در سمت چپ برابر
        
        $$a_0b_{2n} + a_1b_{2n - 1} + a_2b_{2n - 2} + \dots + a_{2n}b_0$$
        
        است. با توجه به اینکه
        $$b_{2n} = a_{2n} = a_0,\quad b_{2n - 1} = -a_{2n - 1} = -a_1,$$
        $$\quad b_{2n - 2} = a_{2n - 2} = a_2,\quad \dots$$
    	
        نتیجه می‌گیریم عبارت فوق برابر است با
        $$a_0^2 - a_1^2 + a_2^2 - a_3^2 + \dots + a_{2n}^2 = $$
        $$2(a_0^2 - a_1^2 + a_2^2 - \dots + (-1)^{n - 1}a_{n - 1}^2) + (-1)^n a_n^2$$
    	
        چنانچه عبارت اخیر را برابر $a_n$ قرار دهیم, حکم ثابت می‌شود.
        
        
	\end{enumerate}

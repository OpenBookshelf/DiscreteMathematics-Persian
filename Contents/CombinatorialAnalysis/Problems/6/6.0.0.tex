	\begin{enumerate}
		\item 
        \p
        طبق قضیه‌ی چندجمله‌ای برای هر عدد
         طبیعی 
         $n$
         و هر عدد مثبت 
         $m$ 
         فرمول ضرایب چند جمله‌ای به صورت زیر خواهد بود:
        $$(x_1+x_2+\ldots+x_m)^n = $$
        $$\sum_{n_1+n_2+\ldots+n_m=n} \frac{n!}{n_1!n_2!\ldots n_k!} x_1^{n_1}x_2^{n_2}\ldots x_k^{n_k}$$
        حال بر حسب چندجمله‌ای داده‌شده، عبارت فوق را تعریف می‌کنیم. برای ساده‌سازی فرض کنید در این تعریف توان جمله‌ی اول، یعنی
        $n_1$
        را برابر $a$ ،
        توان جمله‌ی دوم یعنی 
        $n_2$
        را برابر 
        $b$
        و توان جمله‌ی سوم یعنی
        $n_3$
        را برابر $c$ در نظر بگیریم.
        
        \begin{align*}
		(1+x+x^2)^n = \sum_{a+b+c=n} \frac{n!}{a!b!c!} x_1^{a} x_2^{b} x_3^{c}\\
		= \sum_{a+b+c=n} \frac{n!}{a!b!c!} 1^{a} x^{b} (x^2)^{c}\\
		= \sum_{a+b+c=n} \frac{n!}{a!b!c!} x^{b + 2c}
        \end{align*}
		در نهایت چندجمله‌ای داده شده به فرم بالا ساده می‌شود.
		حال سه‌تایی $(a',b',c')$ را در نظر می‌گیریم که در شرایط تعریف فوق صدق کند.$(a'+b'+c' = n)$
        اگر این سه‌تایی  جملهٔ $x^{n + k}$ تولید کند, در این صورت سه‌تایی $(c', b', a')$ جملهٔ $x^{n - k}$ را تولید می‌کند زیرا:
        $$b' + 2a' = 2(a' + b' + c') - (b' + 2c') = 2n - (n + k) = n -k$$
        
        \p
        حال در جهت عکس ثابت می‌کنیم که یک سه‌تایی که جمله‌ی
        $x^{n-k}$
        را تولید می‌کند، می‌تواند جمله‌ی
        $x^{n+k}$
        را تولید کند. بدین منظور فرض کنید سه‌تایی
        $(c'', b'', a'')$
        در شرط موجود در عبارت صدق کند
        $(c''+b''+a''=n)$
        و جمله‌ی
        $x^{n-k}$
        را تولید کند.
        در این صورت سه‌تایی
        $(a'',b'',c'')$
        طبق رابطه‌ی زیر، جمله‌ی 
        $x^{n-k}$
        را تولید می‌کند:
    	$$b'' + 2c'' = 2(a'' + b'' + c'') - (b'' + 2a'') = 2n - (n - k) = n +k$$
	
        \p
    	در ضمن ضریب جملهٔ متناظر با سه‌تایی $(a', b', c')$ با ضریب جملهٔ سه‌تایی متناظر با $(c', b', a')$ طبق رابطه‌ی بیان شده در ابتدای سوال برابر است و لذا $a_{n + k} = a_{n - k}$.
	
	\item
        \p
    	با توجه به قسمت (الف) 
    
    	$$a_0 = a_{2n},\quad a_1 = a_{2n - 1},\quad a_2 = a_{2n - 2}, \dots ,\quad a_{n - 1} = a_{n + 1}$$
    
    	و لذا در عبارت داده شده, جملات دو به دو قرینهٔ یکدیگرند.
    	\begin{align*}
    		2A = a_0a_1 - a_1a_2 + a_2a_3 - \dots - a_{2n-1}a_{2n}\\
    		+ a_{2n}a_{2n-1} - a_{2n-1}a_{2n-2} +\dots - a_1a_0 \\
    		\Rightarrow 2A = 0 \Rightarrow A = 0
    	\end{align*}
	
	\item
        \p
    	با توجه به راه حل قسمت (الف), $a_k$ برابر مجموع اعداد به صورت $\frac{n!}{a!b!c!}$ است که در آن:
    	$$a + b + c = n$$ 
    	$$b + 2c = k$$
    	 حال اگر $b_k$ را ضریب $x^k$ در بسط $(1 - x + x^2)^n$ بگیریم, در این صورت $b_k$ برابر مجموع اعداد $\frac{(-1)^bn!}{a!b!c!}$ است (طبق قضیه‌ی چندجمله‌ای  به دست می‌آید) که در آن:
    	 $$a + b + c =  n$$ 
    	 $$b + 2c = k$$
    	 پس اگر $k$ زوج باشد, $b$ نیز زوج است (طبق قید فوق) و $b_k = a_k$ و اگر $k$ فرد باشد, $b$ نیز فرد است و $b_k = -a_k$.
    	 
        \p
        همچنین چون $a_k$ ضریب $x^k$ در بسط $(1 + x + x^2)^n$ است, پس ضریب $x^{2k}$ در بسط $(1 + x^2 + x^4)^n$ برابر $a_k$ است. زیرا:
         $$(1+x^2+x^4)^n=\sum_{a+b+c=n} \frac{n!}{a!b!c!} x^{2b+4c}$$
         بنابراین ضریب $a_{2k}$ برابر مجموع اعداد $\frac{n!}{a!b!c!}$ است که در آن $a+b+c=n$ و $2b+4c=2k$ که برابر $a_k$ است. حال تساوی
        $$(1 + x + x^2)^n(1 - x + x^2)^n = (1 + x^2 + x^4)^n$$
        را در نظر بگیرید. ضریب $x^{2n}$ در سمت راست برابر $a_n$ و در سمت چپ برابر
        
        $$a_0b_{2n} + a_1b_{2n - 1} + a_2b_{2n - 2} + \dots + a_{2n}b_0$$
        
        است. با توجه به اینکه
        $$b_{2n} = a_{2n} = a_0,\quad b_{2n - 1} = -a_{2n - 1} = -a_1,$$
        $$\quad b_{2n - 2} = a_{2n - 2} = a_2,\quad \dots$$
    	
        نتیجه می‌گیریم عبارت فوق برابر است با
        $$a_0^2 - a_1^2 + a_2^2 - a_3^2 + \dots + a_{2n}^2 = $$
        $$2(a_0^2 - a_1^2 + a_2^2 - \dots + (-1)^{n - 1}a_{n - 1}^2) + (-1)^n a_n^2$$
    	
        چنانچه عبارت اخیر را برابر $a_n$ قرار دهیم, حکم ثابت می‌شود.
        
        
	\end{enumerate}

\p
برای اثبات از دوگانه‌شماری استفاده می‌کنیم.
فرض کنید می‌خواهیم
یک گروه 
$n$
نفره 
از بین
$n$ 
نفر با کلاه قرمز و
$m$
نفر با کلاه آبی تشکیل دهیم.
این کار را به دو روش انجام می‌دهیم:
\begin{itemize}
    \item 
    روش اول: 
    $$\binom{m + n}{n}$$
    $n$
    نفر را از بین 
    $m + n$
    نفر انتخاب می‌کنیم.  
    \item 
    روش دوم: 
    $$ \underbrace{\sum\limits_{k=0}^{n}}_{\text{ج}} \underbrace{\binom{m}{k}}_{\text{آ}} \underbrace{\binom{n}{k}}_{\text{ب}}$$
    \begin{enumerate}
    \item 
    ابتدا 
    $k$ 
    نفری که قرار است کلاه آبی داشته باشند را انتخاب می‌کنیم.
    \item 
    سپس 
    $n-k$ 
    نفر باقی‌مانده را از کلاه قرمزها انتخاب می‌کنیم. توجه کنید که
    $ \binom{n}{n-k} = \binom{n}{k}$.
    \item 
    در نهایت طبق اصل جمع، به ازای همه‌ی 
    $k$
    های ممکن داریم:
     $$\sum\limits_{k=0}^{n} \binom{m}{k} \binom{n}{k}$$
    \end{enumerate}
\end{itemize}
با توجه به معادل بودن دو روش فوق، طبق دوگانه‌شماری داریم:
$$\binom{m + n}{n} = \sum\limits_{k=0}^{n} \binom{m}{k} \binom{n}{k}$$
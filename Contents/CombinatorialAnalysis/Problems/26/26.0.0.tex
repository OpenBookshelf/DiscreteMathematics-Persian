\p 
$2$ 
بسته‌ی
$at$
را در نظر می‌گیریم. از آن‌جایی که درون هر کدام از بسته‌ها
$2$ 
  حرف قابلیت جابه‌جایی دارند،
$4$ 
  حالت داریم. با درنظر گرفتن این دو بسته،
$11$ 
  عنصر خواهیم داشت که برای محاسبه تعداد حالات آن‌ها، از رابطه جایگشت تکراری استفاده می‌کنیم:
$$\binom{11!}{2!2!3!}$$ 
\p
  از این تعداد باید حالاتی که دارای
$atat$
   یا 
$tata$
   هستند را کم کنیم. تعداد این حالات برابر است با:
$$\binom{10!}{2!3!}\times2$$
\p
  همچنین حالتی که یک
$tat$
     جدا از یک 
$a$
      دیگر داشته باشیم نیز مورد قبول می‌باشد.تعداد این حالت برابر است با: 
      $$\binom{9!}{2!3!} \times {p(10,2)}$$
	\p
       ابتدا 
$9$ 
       عنصر را با استفاده از فرمول جایگشت تکراری می‌چینیم، سپس از بین 
$10$ 
      جایی که بین و اطراف این 
$9$ 
      عنصر به وجود آمده است،
$2$ 
      جا را برای
$tat$
      و
$a$ 
      انتخاب می‌کنیم.
       نتیجه نهایی به صورت زیر محاسبه می‌شود:
    \[\frac{11!}{2!2!3!}\times4 - \frac{10!}{2!3!}\times2 + {\frac{9!}{2!3!}}\times{p(10,2)}\]
\p 
$2$ 
بسته‌ی
$at$
را در نظر می‌گیریم. از آن‌جایی که درون هر کدام از بسته‌ها
$2$  
حرف قابلیت جابه‌جایی دارند،
$4$ حالت
داریم. با درنظر گرفتن این دو بسته،
$11$ 
  عنصر خواهیم داشت که برای محاسبه تعداد حالات آن‌ها، از رابطه جایگشت تکراری استفاده می‌کنیم:
$$\frac{11!}{2!2!3!} \times 4$$ 
\p
  از این تعداد باید حالاتی که دارای
$atat$
   یا 
$tata$
   هستند را کم کنیم. تعداد این حالات برابر است با:
$$\frac{10!}{2!3!}\times2$$
\p
  همچنین حالتی که یک
$tat$
     جدا از یک 
$a$
      دیگر داشته باشیم نیز مورد قبول می‌باشد.
      برای این کار، 
      ابتدا 
      $9$ 
             حرف دیگر را با استفاده از فرمول جایگشت تکراری می‌چینیم. سپس در 
      $10$ 
            جایگاهی که قبل و بعد از این 
      $9$ 
            حرف به وجود آمده است،
            طبق ترتیب
      $tat$
            و
      $a$ 
      را در
      $2$ 
      جایگاه‌ قرار می‌دهیم:
      $$\frac{9!}{2!3!} \times {p(10,2)}$$
	\p
     در نتیجه پاسخ نهایی برابر است با:
    \[(\frac{11!}{2!2!3!}\times4) - (\frac{10!}{2!3!}\times2 )+ ({\frac{9!}{2!3!}}\times{p(10,2)})\]
\begin{enumerate}
    \item 
     به
$\frac{10!}{2!3!}$
حالت می‌توان حروف غیر از i را در یک ردیف قرار داد
(یادآوری: به 
$10!$
تمام ده حرف را در ردیف جایگشت می‌دهیم و سپس جواب را تقسیم بر جایگشت اعداد تکراری که در اینجا دو تا n و سه تا t هستند می‌کنیم).
در بین این 10 حرف 11 جایگاه برای گذاشتن i ها وجود دارد که 3 تا از آن‌ها موجب ساخت عبارت it می‌شوند.
پس تعداد راه‌های قرار دادن سه حرف i به طوری که عبارت it تولید نشود برابر با تعداد جواب‌های معاله
$x_1 + x_2 \ldots + x_8 = 3$
در مجموعه اعداد صحیح و نامنفی یعنی برابر
$\binom{10}{7}$
است. پس پاسخ نهایی برابر است با:
$$\frac{10!}{2!3!}  \times \binom{10}{7}$$

\item
به
$\frac{10!}{3!2!}$
طریق می‌توانیم حروف غیر از t را در یک ردیف قرار دهیم که در 
$\frac{9!}{3!}$
حالت حروف n مجاورند.
(یادآوری: در این حالت انگار دو حرف n را با یکدیگر یک حرف فرض می‌کنیم. پس یکی از تعداد حروف کم شده و جایگشت ۹ حرف که ۳ تا از آن‌ها تکراری هستند را محاسبه می‌کنیم).
اگر دو حرف n مجاور باشند حروف t را در 8 فضا از 11 فضای خالی ایجاد شده می‌توانیم قرار دهیم. پس داریم:

$$x_1 + x_2 \ldots + x_8 = 3$$

که معادله بالا 
$\binom{10}{7}$
جواب در مجموعه اعداد صحیح و نامنفی دارد.
اگر دو حرف n غیرمجاور باشند حروف t را در 7 فضا از فضاهای خالی می‌توانیم قرار دهیم پس داریم:

$$x_1 + x_2 + \ldots + x_7 = 3$$

که معادله بالا 
$\binom{9}{6}$
جواب در مجموعه اعداد صحیح و نامنفی دارد. 
پس جواب نهایی برابر است با:

$$\frac{9!}{3!}\times\binom{10}{7} + (\frac{10!}{3!2!} - \frac{9!}{3!})\times\frac{9}{6}$$

\item
به
$\frac{7!}{2!}$
حالت
می‌توانیم حروف غیر از i و t را در یک ردیف قرار دهیم. حال از 8 فضا خالی بین این حروف دو جایگاه مجاور s هستند. پس 6 فضای مجاز برای حروف i و t وجود دارد.
همانند بخش پیشین، تعداد حالات قرار دادن ۶ حرف i و t در ۶ جایگاه از حل معادله سیاله زیر برابر با 
$\binom{11}{5}$
می‌شود.
$$x_1 + x_2 \ldots + x_6 = 6$$
در نهایت جایگشت حروف i و t را نیز محاسبه می‌کنیم.
پس جواب آخر برابر است با:
$$\frac{7!}{2!}\times\binom{11}{5}\times\frac{6!}{3!3!}$$

\item
حروف بی صدا را به 
$\frac{7!}{3!2!}$
طریق می‌توانیم در یک ردیف قرار دهیم که در 
$\frac{6!}{3!}$
حالت دو حرف n مجاورند.

-
اگر دو حرف n مجاور باشند، 
از بین ۷ جایگاه بین حروف بی صدا (توجه کنید بین دو حرف n جایگاهی وجود ندارد)، حروف صدادار در ۵ جایگاه می‌توانند قرار بگیرند.
پس مانند بخش‌های قبل ابتدا معادله سیاله زیر را حل می‌کنیم.

$$x_1 + x_2 \ldots + x_5 = 6$$
حال با محاسبه جایگشت‌های حروف صدادار جواب نهایی این بخش برابر است با:
$\binom{10}{4}\times\frac{6!}{3!}$

-
اگر دو حرف n مجاور نباشند حروف صدادار می‌توانند در ۴ جایگاه از ۸ جایگاه بین حروف بی صدا قرار بگیرند. که تعداد حالات آن با حل معادله سیاله زیر بدست می‌آید.
$$x_1 + x_2 \ldots + x_4 = 6$$
در نهایت با محاسبه جایگشت حروف صدادار جواب نهایی این بخش برابر است با:
$\binom{9}{3}\times\frac{6!}{3!}$

پاسخ نهایی از جمع دو حالت بالا بدست می‌آید.
$$\frac{6!}{3!}\times\binom{10}{4}\times\frac{6!}{3!}+ (\frac{7!}{3!2!} - \frac{6!}{3!})\times\binom{9}{3}\times\frac{6!}{3!}$$

\end{enumerate}
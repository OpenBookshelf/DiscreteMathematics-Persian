\p
فرض کنید
$n$
حالاتِ تعدادِ مختلف ممکن کتاب‌ها در قفسه‌ها باشد.
می‌خواهیم
$n$
را به کمک معادله‌ی زیر محاسبه کنیم:
$$x_1+x_2+x_3+x_4=24;\: x_i\geq 1;\: 1\leq i\leq 4$$
طبق 
\CROSSREF{محدودیت حداقلی متغیرها در معادله سیاله خطی با ضرایب واحد}،
معادله به شکل زیر بازنویسی شده و 
$n$
از آن به دست می‌آید:

$$ y_1+y_2+y_3+y_4=20; y_i\geq 0;\: 1\leq i\leq 4$$
$$\Rightarrow n = {20+4-1 \choose 4-1} = {23 \choose 3}$$
حال پس از مشخص شدن تعداد جایگاه‌ها در هر قفسه و با توجه به متمایز بودن کتاب‌ها، برای هر حالتِ تعداد کتاب‌ها
 به 
$24!$
حالت می‌توانیم کتاب‌ها را در قفسه‌ها بچینیم.
پس تعداد کل حالات برابر است با:
$${23 \choose 3} \times 24!$$

\p
با کمی تامل در صورت سوال می‌توان به ساختار سازمانی شرکت مورد نظر پی برد(شکل زیر).

\p
\centerimage{0.8}{./1.jpg}
\p

حال روی حضور یا عدم حضور مدیرعامل حالت‌بندی می‌کنیم
(کسانی که حضور دارند را با رنگ سبز و کسانی که حضور ندارند را با رنگ قرمز نشان می‌دهیم).
    
\begin{enumerate}
    \item 
    \textbf{مدیرعامل حضور داشته باشد:}
    در این صورت هیچ کدام از معاونین نمی‌توانند در مهمانی حضور داشته باشند و ساختار شکل زیر را خواهیم داشت.

    \p
    \centerimage{0.8}{./2.jpg}
    \p
        
    حال باید برای هر کدام از جفت‌های کارمند و دستیار، حالات ممکن را به دست بیاوریم. این حالات در شکل زیر نشان داده شده‌اند:

    \p
    \centerimage{0.15}{./3.jpg}
    \p
    
    پس برای هر کدام از جفت‌ها که از هم مستقل هستند، سه حالت داریم که یعنی طبق اصل ضرب
    $3^{15}$
    حالت می‌توانیم داشته باشیم.
    
    \item 
    \textbf{مدیرعامل حضور نداشته باشد:}
    در این صورت ساختاری شبیه شکل زیر خواهیم داشت. سه بخش از شرکت 7 نفر و سه بخش دیگر 5 نفر عضو دارد و هرکدام از این ۶ بخش، از یکدیگر مستقل هستند. پس طبق اصل ضرب تعداد حالات آن‌ها در یکدیگر ضرب می‌شود.
    
    \p
    \centerimage{0.8}{./4.jpg}
    \p
    
    \begin{itemize}
        \item 
        \textbf{بخش‌های 7 نفره:}
        روی حضور یا عدم حضور معاون بخش حالت‌بندی می‌کنیم. اگر در مهمانی حاضر نباشد، سه جفت کارمند و دستیار خواهیم داشت که در مجموع همانند قبل
        $3^3$
        حالت دارند.

        \p
        \centerimage{0.15}{./5.jpg}
        \p
        
        اگر معاون بخش در مهمانی حضور داشته باشد، کارمندان او نباید حضور داشته باشند. حال 3 دستیار مستقل داریم که هر کدام دو حالت دارند، پس طبق اصل ضرب
        $2^3$
        حالت داریم.

        \p
        \centerimage{0.15}{./6.jpg}
        \p
        
       در نتیجه طبق اصل جمع برای هر بخش 7 نفره‌ی شرکت می‌توانیم $3^3 + 2^3 = 35 $ حالت داشته باشیم.
        
        \item 
        \textbf{مجموعه 5 تایی:}
        روی حضور یا عدم حضور معاون بخش حالت بندی می‌کنیم.
        اگر در مهمانی حاضر نباشد، دو جفت کارمند و دستیار خواهیم داشت  که در مجموع همانند قبل
        $3^2$
        حالت دارند.

        \p
        \centerimage{0.15}{./8.jpg}
        \p
        
        اگر معاون بخش در مهمانی حضور داشته باشد، کارمندان او نباید حضور داشته باشند. حال 2 دستیار مستقل داریم که هر کدام دو حالت دارند، پس طبق اصل ضرب $2^2$ حالت داریم.

        \p
        \centerimage{0.15}{./7.jpg}
        \p
        
        در نتیجه طبق اصل جمع برای هر بخش 5 نفره‌ی شرکت می‌توانیم
        $ 3^2 + 2^2 = 13 $
        حالت داشته باشیم.
    \end{itemize}
    
    \p
    طبق اصل ضرب اگر مدیرعامل شرکت در مهمانی حضور نداشته باشد
    $$ 13^3 \times  35 ^ 3 $$
    حالت برای برگزاری مهمانی خواهیم داشت.
    
\end{enumerate}
    
\p
حال طبق اصل جمع باید پاسخ حالات به دست آمده برای حضور یا عدم حضور مدیرعامل را با یکدیگر جمع کنیم:   
$$13^3 \times  35 ^ 3  + 3^{15}$$

    \p
 می‌توانیم جدول را به دو صورت شطرنجی رنگ‌آمیزی ‌کنیم. در هر دو صورت هیچ دو خانه‌ی سیاهی مجاور ضلعی نیستند. 
 حال اگر هر کدام از خانه‌های سیاه را سفید کنیم، همچنان خانه‌های سیاه مجاور ضلعی نخواهند بود.
  چون هر يك از 50 خانه‌ي سياه جدول دو حالت براي رنگ‌آمیزی دارد،  
  پس جدول را به 
  $ (2 \times 2^{50})- 1 $ 
   حالت می‌توان رنگ‌آمیزی کرد که بیشتر از 
   $2^{50}$
  است.
  ‌  توجه كنيد حالتي كه همه‌ی خانه‌ها سفید هستند دو بار شمرده شده است پس یک حالت را کم کردیم.


    \p
     حال جدول را به 50 مستطیل 1×2 افراز میکنیم. هر یک از این مستطیل‌ها را می‌توان به حداکثر 3 حالت رنگ‌آمیزی کرد (هر دو خانه نمی‌توانند سیاه باشند جون با هم مجاورند). پس جدول را به حداکثر $3^{50}$ حالت می‌توان رنگ‌آمیزی کرد. دقت کنید که تعداد حالات رنگ‌آمیزی کمتر از این است چون رنگ‌آمیزی مستطیل‌ها مستقل از هم نیست و تعداد حالات رنگ‌آمیزی یک مستطیل با توجه با مستطیل‌های اطرافش می‌تواند کمتر از 3 باشد.
     براي مثال يكي از حالت‌هاي غير قابل قبول در شكل زير نشان داده شده است.
     \\\centerimage{0.4}{./0.png}  
      با توجه به شكل تعداد حالت‌هاي قابل قبول نمی‌تواند برابر 
      $3^{50}$
      باشد و كوچكتر از آن است.
    \p
    در نتیجه:
    $$2^{50} < \text{تعداد حالات رنگ‌آمیزی مناسب} < 3^{50}$$

    \p
    ابتدا نشان می‌دهیم که تعداد حالات رنگ‌آمیزی براي جدول بیشتر از 
    $2^{50}$
    است.
  می‌توانیم جدول را به صورت شطرنجی، با دو روش رنگ‌آمیزی ‌کنیم. در هر دو صورت هیچ دو خانه‌ی سیاهی مجاور ضلعی نیستند. همچنین هر یک از خانه‌های سیاه می‌تواند سفید بشوند یا نشوند (دو حالت)؛ زیرا اگر هر یک از خانه‌های سیاه، سفید شوند همچنان خانه‌های سیاه مجاور ضلعی نخواهند بود. 
  ‌توجه كنيد حالت خاصي كه همه‌ی خانه‌های سیاه، سفید شده‌اند دو بار شمرده شده است پس یک حالت را کم می‌کنیم.
  با توجه به اینکه هر حالت شطرنجی ۵۰ خانه‌ی سیاه دارد، در نتیجه می‌توان جدول را حداقل به
   $ (2 \times 2^{50})- 1 $ 
   حالت رنگ‌آمیزی کرد که این مقدار‎
    بیشتر از 
   $2^{50}$
  است.

    \p
     حال نشان می‌دهیم که تعداد حالات رنگ‌آمیزی کمتر از 
     $3^{50}$
     است.
      جدول را به 50 مستطیل 1×2 افراز می‌کنیم. هر یک از این مستطیل‌ها را می‌توان به حداکثر 3 حالت رنگ‌آمیزی کرد (هر دو خانه نمی‌توانند سیاه باشند چون با هم مجاورند). پس جدول را به حداکثر $3^{50}$ حالت می‌توان رنگ‌آمیزی کرد. دقت کنید که تعداد حالات رنگ‌آمیزی کمتر از این است چون رنگ‌آمیزی مستطیل‌ها مستقل از هم نیست و تعداد حالات رنگ‌آمیزی یک مستطیل با توجه به مستطیل‌های اطرافش می‌تواند کمتر از 3 باشد.
     براي مثال يكي از حالت‌هاي نامناسب در شكل زير نشان داده شده است:
     \\\centerimage{0.4}{./0.png}  
      با توجه به شكل، تعداد حالت‌هاي مناسب نمی‌تواند  
      $3^{50}$
       باشد و حداکثر برابر
      $3^{50}-1$
     است.
    \p
    در نتیجه:
    $$2^{50} < \text{تعداد حالات رنگ‌آمیزی مناسب} < 3^{50}$$

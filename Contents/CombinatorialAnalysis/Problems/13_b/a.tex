 \p
ابتدا کتاب‌های ریاضی را به
\underline{$7!$ حالت}،
به صورت زیر
 می‌چینیم. 
 با توجه به صورت مسئله، کتاب‌های اول و آخر باید کتاب‌های ریاضی باشند.
 \vspace*{+0.5cm}
 \centerimage{0.6}{./0.png}
 \vspace*{+0.4cm}
 سپس با
 \underline{ترتیب 3 از 6}
کتاب‌های فیزیک را در 
 3 خانه بین کتاب‌های ریاضی قرار می‌دهیم.
طبق اصل ضرب تعداد کل حالت‌ها برابر است با:
$$7! \times \frac{6!}{3!}$$
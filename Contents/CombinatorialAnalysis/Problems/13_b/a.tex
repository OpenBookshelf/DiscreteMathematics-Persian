\p
فرض کنید
 $M$
 ‌ها نشان دهنده‌ی کتاب‌های ریاضی و مربع‌های خالی نشان دهنده خانه‌های مجاز برای کتاب فیزیک باشند، حالت قرار گرفتن کتاب‌ها به صورت زیر می‌شود:
 \centerimage{0.8}{./0.jpg}
 کتاب‌های ریاضی را به
\underline{$7!$ حالت}
 می‌چینیم.
با ترتیب 3 از 6 
کتاب‌های فیزیک را در 
3 خانه قرار می‌دهیم.
طبق اصل ضرب تعداد کل حالت‌ها برابر است با:
$$7! \times \frac{6!}{3!}$$
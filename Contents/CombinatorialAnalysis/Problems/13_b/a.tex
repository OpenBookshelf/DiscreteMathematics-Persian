\p
فرض کنید
 $M$
 ‌ها نشان دهنده‌ی کتاب‌های ریاضی و مربع‌های خالی نشان دهنده خانه‌های مجاز برای کتاب فیزیک باشند، حالت قرار گرفتن کتاب‌ها به صورت زیر می‌شود:
 \centerimage{0.8}{./0.jpg}
 کتاب‌های ریاضی را به
\underline{$7!$ حالت}
 می‌چینیم.
 برای قرار دادن کتاب‌های فیزیک باید به
 \underline{$\binom{6}{3}$ حالت}
‌از 6 خانه‌ی ممکن، 3 تا را انتخاب کنیم
و به
\underline{$3!$ حالت}
کتاب‌ها را در 3 خانه قرار دهیم(ترتیب).
طبق اصل ضرب تعداد کل حالت‌ها برابر است با:
$$7! \times \binom{6}{3} \times 3!$$
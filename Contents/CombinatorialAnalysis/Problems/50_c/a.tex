\p
به
$\frac{7!}{2!}$
حالت
می‌توانیم حروف غیر از i و t را در یک ردیف قرار دهیم. حال از 8 فضا خالی بین این حروف دو جایگاه مجاور s هستند. پس 6 فضای مجاز برای حروف i و t وجود دارد.
همانند بخش پیشین، تعداد حالات قرار دادن ۶ حرف i و t در ۶ جایگاه از حل معادله سیاله زیر برابر با 
$\binom{11}{5}$
می‌شود.
$$x_1 + x_2 \ldots + x_6 = 6$$
در نهایت جایگشت حروف i و t را نیز محاسبه می‌کنیم.
پس جواب آخر برابر است با:
$$\frac{7!}{2!}\times\binom{11}{5}\times\frac{6!}{3!3!}$$

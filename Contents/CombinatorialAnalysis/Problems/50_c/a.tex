\p
به
\underline{$\frac{7!}{2!}$ حالت}  
می‌توانیم حروف غیر از $i$ و $t$ را در یک ردیف قرار دهیم. حال از 8 فضای خالی بین این حروف دو جایگاه مجاور $s$ هستند. پس 6 فضای مجاز برای حروف $i$ و $t$ وجود دارد.
همانند بخش پیشین، تعداد حالات قرار دادن ۶ حرف 
$i$ و $t$
 در ۶ جایگاه از حل معادله سیاله زیر برابر با 
 \underline{$\binom{11}{5}$}
می‌شود.
$$x_1 + x_2 \ldots + x_6 = 6$$
در نهایت جایگشت حروف 
$i$ و $t$ 
را 
به صورت 
\underline{$\frac{6!}{3!3!}$}
محاسبه می‌کنیم.
پس جواب آخر برابر است با:
$$\frac{7!}{2!}\times\binom{11}{5}\times\frac{6!}{3!3!}$$

\p
طبق رابطه‌ی تعداد جایگشت‌های خطی با اعضای تکراری،
به
\underline{$\frac{7!}{2!}$ حالت}  
 حروف غیر از $i$ و $t$ 
 را در یک ردیف قرار می‌دهیم. حال از 8 جایگاه قبل و بعد این حروف، دو جایگاه مجاور
  $s$ 
  هستند. پس 6 جایگاه مجاز برای حروف 
  $i$ و $t$ وجود دارد.
  با غير متمایز فرض كردن
  $i$
  و
  $t$،
 تعداد حالات قرار دادن اين ۶ حرف 
 ‌‌در ۶ جایگاه مجاز، طبق معادله سیاله‌ی
 $x_1 + x_2 \ldots + x_6 = 6$
برابر  
 \underline{$\binom{11}{5}$}
به دست می‌آید.
در نهایت جایگشت حروف 
$i$ و $t$ 
در 6 جایگاه انتخاب شده 
به صورت 
\underline{$\frac{6!}{3!3!}$}
محاسبه مي‌شود.
پس جواب آخر برابر است با:
$$\frac{7!}{2!}\times\binom{11}{5}\times\frac{6!}{3!3!}$$

\p
حاصل‌ضرب تمام خانه‌های جدول را درنظر می‌گیریم. از یک طرف این حاصل‌ضرب برابر است با
$(-1)^n$
و از طرف دیگر برابر است با
$(-1)^m$
.
. بنابراین برای این که مسئله جواب داشته باشد، باید داشته باشیم:
$$n \equiv m (mod\;2)$$
حالا سطر اول را به‌گونه‌ای پر می‌کنیم که حاصل‌ضرب اعداد آن
$-1$
شود. کافی است
$m - 1$
خانه‌ی اول را به طریق دل‌خواه پر کنیم و فقط خانه‌ی آخر را به‌گونه‌ای قرار دهیم که این حاصل‌ضرب برابر
$-1$
شود. پس تعداد راه‌های پر کردن سطر اول برابر است با
$2^{m-1}$
. به طریق مشابه
$n - 1$
سطر را پر می‌کنیم و به ازای هر ستون خانه‌ی آخر را به‌گونه‌ای انتخاب می‌کنیم که حاصل‌ضرب این ستون برابر
$-1$
شود. پس سطر آخر به صورت منحصر به فرد مشخص می‌شود. حالا اثبات می‌کنیم که حاصل‌ضرب اعداد سطر آخر برابر
$-1$
است. دوباره مشابه بالا حاصل‌ضرب تمام اعداد جدول را در نظر می‌گیریم.
\p
حاصل‌ضرب هر ستون برابر
$-1$
و حاصل‌ضرب اعضای
$m - 1$
سطر اول هم برابر
$-1$
است. پس اگر حاصل‌ضرب اعداد سطر آخر را برابر
$x$
بگیریم، داریم:
$$(-1)^n = (-1)^{m-1} \times x \Rightarrow because\;of\;m \equiv n(mod\;2) \Rightarrow x = -1$$
پس حاصل‌ضرب اعداد سطر آخر هم برابر
$-1$
است. بنابراین تمامی جدول‌هایی که درست کرده‌ایم، دارای خاصیت مورد نیاز مسئله هستند. پس داریم:
$$number = 2^{m-1} for\;each\;row \Rightarrow total = (2^{m-1})^{n-1} = 2^{(m-1)(n-1)}$$
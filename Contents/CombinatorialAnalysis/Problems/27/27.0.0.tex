\p
لم: تعداد حالاتی که متحرک می‌تواند از نقطه
$(x_1 , y_1)$
به نقطه
$(x_2 , y_2)$
‌برود را به صورت زیر محاسبه می‌کنیم:
$$X = |x_1 - x_2|$$
$$Y = |y_1 - y_2|$$
$$d(X , Y) = \frac{X!}{(\frac{X - |Y|}{2})!(\frac{X + |Y|}{2})!} $$
\p
برای حل این سوال باید تعداد حالاتی که متحرک به مقصد می‌رسد و از نقطه 
$(7, 3)$ 
عبور می‌کند را محاسبه کنیم. سپس تعداد حالاتی که متحرک هم از نقطه
$(7, 3)$ 
و هم از نقطه 
$(13, 3)$ 
رد می‌شود را از آن کم ‌کنیم. برای سهولت در بیان، ابتدا حرکت‌ها را نام گذاری می‌کنیم:
$$(x ,y) \to (x + 1, y + 1) : A $$
$$ (x, y) \to (x + 1, y - 1) : B$$
طبق لم فوق، حالات رسیدن از مبدا به نقطه
$(7, 3)$
 به صورت
$d(7, 3)$
و حالات رسیدن به نقطه
$(17, 5)$
از 
$(7, 3)$
به صورت 
$d(10, 2)$
محاسبه می‌شود. در نتیجه طبق اصل ضرب، تعداد حالاتی که متحرک از مبدا به مقصد رسیده و از نقطه 
$(7, 3)$
عبور می‌کند برابر است با:
$$d(7, 3) \times d(10, 2) = \frac{7!}{5!2!}\times\frac{10!}{4!6!}$$

اکنون تعداد حالاتی که با شرط عبور از
2 
نقطه
$(7, 3)$ 
و
$(13, 3)$،
از مبدا به مقصد می‌توان رسید  
را به دست می‌آوریم.
تعداد حالات رسیدن به 
$(7, 3)$
از مبدا 
به صورت
$d(7, 3)$
محاسبه می‌شود.
از
$(7, 3)$ 
می‌توان به 
$d(6, 0)$
حالت به
$(13, 3)$
رسیده و از آن‌جا به 
$d(4, 2)$
حالت به مقصد رفت.
 طبق اصل ضرب، تعداد حالاتی که متحرک از
  $(7, 3)$ 
 و
 $(13, 3)$
 عبور کرده و به
 $(17, 5)$
 برسد برابر است با:
 $$d(7, 3) \times d(6, 0) \times d(4, 2) = \frac{7!}{5!2!}\times\frac{6!}{3!3!}\times\frac{4!}{3!1!}$$

با توجه به توضیحات فوق، پاسح نهایی به صورت زیر محاسبه می‌شود:
$$d(7, 3) \times d(10, 2) - d(7, 3) \times d(6, 0) \times d(4, 2) = \frac{7!}{5!2!}\times\frac{10!}{4!6!} - \frac{7!}{5!2!}\times\frac{6!}{3!3!}\times\frac{4!}{3!1!}$$
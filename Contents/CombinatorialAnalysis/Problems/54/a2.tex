\p		
از دوگانه‌شماری استفاده می‌کنیم. می‌خواهیم از بین
$n + 1$
توپ متمایز، که 
$m$
عدد از آن‌ها آبی و بقیه قرمز هستند، 
$k + 1$
توپ انتخاب کنیم به طوری که حداقل یکی از آن‌ها قرمز باشد. به دو روش این کار را انجام می‌دهیم:
\begin{itemize}
\item
از میان کل توپ‌ها، بدون در نظر گرفتن رنگشان، 
 $k + 1$
توپ را انتخاب می‌کنیم. با توجه به شرط قرمز بودن حداقل یکی از توپ‌ها، طبق اصل متمم، تعداد حالات نامطلوب یعنی حالاتی که تمام توپ‌های انتخاب شده آبی باشند را از حاصل به‌دست آمده، کم می‌کنیم:
$$\binom{n + 1}{k + 1} - \binom{m}{k + 1}$$
\item
فرض می‌کنیم تعداد توپ‌های قرمز
$q$
بوده و آن‌ها را با
$a_1, a_2, \cdots, a_q$
نشان می‌دهیم. می‌توانیم تنها
$a_1$
را از میان توپ‌های قرمز انتخاب و
$k$
توپ دیگر را از توپ‌های آبی انتخاب ‌کنیم:
$$\binom{m}{k}$$
یا 
$a_2$
را انتخاب کرده و بقیه توپ‌ها را از میان تمام توپ‌های آبی و 
$a_1$
 انتخاب می‌کنیم:
$$\binom{m + 1}{k}$$
یا 
$a_3$
را انتخاب کرده و بقیه توپ‌ها را از بین تمام توپ‌های آبی و 
$a_1$
 و 
 $a_2$
  انتخاب می‌کنیم:
$$\binom{m + 2}{k}$$
برای دیگر حالات انتخاب، به همین صورت پیش می‌رویم تا در نهایت 
$a_q$
را انتخاب کرده و بقیه توپ‌ها را از میان تمام توپ‌های آبی و 
$q - 1$
تا توپ قرمز قبل از
$a_q$
 انتخاب ‌کنیم:
$$\binom{n}{k}$$
حال طبق اصل جمع، تعداد کل حالات شمارش در این روش برابر است با:
$$\binom{m}{k} + \binom{m + 1}{k} + \binom{m + 2}{k} +\ldots +
\binom{n}{k}$$
\end{itemize}
با توجه به معادل بودن هر دو روش فوق، طبق دوگانه‌شماری:
$$\binom{m}{k} + \binom{m + 1}{k} +\ldots +
\binom{n}{k} = \binom{n + 1}{k + 1} - \binom{m}{k + 1}$$
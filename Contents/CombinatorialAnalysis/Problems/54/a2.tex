\p		
از دوگانه‌شماری استفاده می‌کنیم. می‌خواهیم از بین
$n + 1$
توپ متمایز، که 
$m$
عدد از آن‌ها آبی و بقیه قرمز هستند، 
$k + 1$
توپ انتخاب کنیم به طوری که حداقل یکی از آن‌ها قرمز باشد. به دو روش این کار را انجام می‌دهیم:
\begin{itemize}
\item
روش اول: 
از میان کل توپ‌ها، بدون در نظر گرفتن رنگشان،
 $k + 1$
توپ را انتخاب می‌کنیم. با توجه به شرط قرمز بودن حداقل یکی از توپ‌ها، طبق اصل متمم، تعداد حالات نامطلوب یعنی حالاتی که تمام توپ‌های انتخاب شده آبی باشند را از حاصل به‌دست آمده، کم می‌کنیم:
$$\binom{n + 1}{k + 1} - \binom{m}{k + 1}$$
\item 
روش دوم:\[\sum\limits_{p=m}^{n}\binom{p}{k}\]
فرض می‌کنیم تعداد توپ‌های قرمز
$q$
بوده و آن‌ها را با
$a_1, a_2, \cdots, a_q$
نشان می‌دهیم. 
می‌خواهیم توپ‌های انتخاب شده را در جعبه قرار دهیم.
\centerimage{0.9}{./0.png}
توجه کنیم که توپ‌ها متمایز هستند. بنابراین برای شمردن همه‌ی حالت‌ها، هر بار به ترتیب یک توپ قرمز را انتخاب می‌کنیم تا حداقل یک توپ قرمز در جعبه قرار گیرد.
سپس 
 K
 توپ دیگر را از بین توپ‌های آبی و توپ‌های قرمز انتخاب شده در مراحل قبل انتخاب می‌کنیم.
 بار اول تنها
$a_1$
را از میان توپ‌های قرمز انتخاب و
$k$
توپ دیگر را از توپ‌های آبی انتخاب می‌کنیم:
$$\binom{m}{k}$$
\centerimage{0.5}{./1.png}
 بار دوم
$a_2$
را انتخاب کرده و بقیه توپ‌ها را از میان تمام توپ‌های آبی و 
$a_1$
 انتخاب ‌می‌کنیم:
 $$\binom{m + 1}{k}$$
 \centerimage{0.5}{./2.png}
بار سوم
$a_3$
را انتخاب کرده و بقیه توپ‌ها را از بین تمام توپ‌های آبی و 
$a_1$
 و 
 $a_2$
  انتخاب ‌می‌کنیم:
 $$\binom{m + 2}{k}$$
  \centerimage{0.5}{./3.png}
 در نهایت 
$a_q$
را انتخاب کرده و بقیه توپ‌ها را از میان تمام توپ‌های آبی و 
$q - 1$
تا توپ قرمز قبل از
$a_q$
 انتخاب می‌کنیم:
 $$\binom{n}{k}$$
 \centerimage{0.5}{./4.png}
حال طبق اصل جمع، تعداد کل حالات در این روش به صورت :
 $\sum\limits_{p=m}^{n}\binom{p}{k}$
محاسبه می‌شود.
\end{itemize}
با ‌توجه به معادل بودن هر دو روش فوق، طبق دوگانه‌شماری داریم:
$$\sum\limits_{p=m}^{n}\binom{p}{k} = \binom{m}{k} + \binom{m + 1}{k} +\ldots +
\binom{n}{k} = \binom{n + 1}{k + 1} - \binom{m}{k + 1}$$
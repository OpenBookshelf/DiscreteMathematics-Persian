\p
حروف بی صدا را به 
\underline{$\frac{7!}{3!2!}$ حالت}
 می‌توانیم در یک ردیف قرار دهیم که در 
 \underline{$\frac{6!}{3!}$ حالت}
دو حرف $n$ مجاورند.
  برای مجاور بودن و مجاور نبودن دو حرف 
$n$،
دو حالت وجود دارد. حال می‌خواهیم جایگشت‌های هر حالت را محاسبه ‌کنیم:
\begin{itemize}
\item 
اگر دو حرف $n$ مجاور باشند، 
از بین ۷ جایگاه بین حروف بی صدا (توجه کنید بین دو حرف $n$ جایگاهی وجود ندارد)، حروف صدادار در ۵ جایگاه می‌توانند قرار بگیرند.
 برای محاسبه تعداد حالات آن، معادله سیاله زیر را حل می‌کنیم.

$$x_1 + x_2 \ldots + x_5 = 6$$
حال با محاسبه جایگشت‌های حروف صدادار جواب نهایی این بخش برابر
\underline{$\binom{10}{4}\times\frac{6!}{3!}$}
است.
\item 
اگر دو حرف $n$ مجاور نباشند حروف صدادار می‌توانند در ۴ جایگاه از ۸ جایگاه بین حروف بی صدا قرار بگیرند. که تعداد حالات آن با حل معادله سیاله زیر بدست می‌آید.
$$x_1 + x_2 \ldots + x_4 = 6$$
در نهایت با محاسبه جایگشت حروف صدادار جواب نهایی این بخش برابر
\underline{$\binom{9}{3}\times\frac{6!}{3!}$}
است.
\end{itemize}
پاسخ نهایی از جمع دو حالت بالا بدست می‌آید:
$$\frac{6!}{3!}\times\binom{10}{4}\times\frac{6!}{3!}+ (\frac{7!}{3!2!} - \frac{6!}{3!})\times\binom{9}{3}\times\frac{6!}{3!}$$

\p
فرض کنید
$n$
تعداد ِحالاتِ ممکن برای
\underline{ِتعداد}
کتاب‌های هر قفسه‌ باشد.
تعداد جواب‌هاي معادله زير برابر
$n$
خواهد بود:
$$x_1+x_2+x_3+x_4=24;\: x_i\geq 1;\: 1\leq i\leq 4$$
طبق 
\CROSSREF{محدودیت حداقلی متغیرها در معادله سیاله خطی با ضرایب واحد}،
معادله به شکل زیر بازنویسی شده و 
$n$
از آن به دست می‌آید:
$$ y_i \xleftarrow{\text{تغيير متغير}} x_i - 1;\: 1\leq i\leq 4$$
$$ y_1+y_2+y_3+y_4=20;\: y_i\geq 0;\: 1\leq i\leq 4$$
$$\Rightarrow n = {20+4-1 \choose 4-1} = {23 \choose 3}$$
حال پس از مشخص شدن تعداد جایگاه‌ها در هر قفسه و با توجه به متمایز بودن کتاب‌ها، برای هر حالتِ تعداد کتاب‌ها
 به 
$24!$
حالت می‌توانیم کتاب‌ها را در قفسه‌ها بچینیم.
پس تعداد کل حالات برابر است با:
$${23 \choose 3} \times 24!$$

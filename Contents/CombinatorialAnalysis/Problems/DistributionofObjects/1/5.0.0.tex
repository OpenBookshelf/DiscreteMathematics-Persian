\p
برای قرار دادن
$n+1$
شی متمایز در
$k+1$
دسته یکسان به صورتی که هیچ دسته‌ای خالی نماند($\genfrac{\{}{\}}{0pt}{}{n + 1}{k + 1}$)، به صورت زیر عمل می‌کنیم:
در ابتدا انتخاب می‌کنیم که کدام اشیا با شی اول در یک دسته قرار نگیرند.
تعداد این اشیا می تواند بین
$k$ تا $n$ 
 باشد
 که آن را $i$ می‌نامیم.
 $k \leq i \leq n$
چون
$k$
دسته‌ی باقی‌مانده هر کدام نیازمند حداقل يك شی هستند.
   این اشیا به
${n\choose i}$ 
حالت انتخاب 
و به 
$\genfrac{\{}{\}}{0pt}{}{i}{k}$
حالت بین دسته‌ها توزیع می شوند.
$n-i$ 
شی دیگر نیز به یک حالت با یکدیگر هم‌دسته مي‌باشند. در نتیجه داریم:
$${n\choose i} \genfrac{\{}{\}}{0pt}{}{i}{k}$$
\p
در نهایت با جمع زدن همه حالات ممکن برای $i$ به جواب مسئله می‌رسیم:

$$\genfrac{\{}{\}}{0pt}{}{n + 1}{k + 1} = \sum\limits_{i=k}^{n} {n\choose i} \genfrac{\{}{\}}{0pt}{}{i}{k}$$





\p
اگر تعداد ساعتی که کارمند اول برای کار 
\lr{i} 
صرف مي‌كند را با $x_i$
 و تعداد ساعتی که کارمند دوم کار
 \lr{i}
انجام می‌دهد را با
 $y_i$
نشان ‌دهیم:
\begin{align*}
 x_1 \geq 2 \quad &, \quad y_1 = 0 \\
 x_2 \geq 0 \quad &, \quad y_2 = 2x_2 \\
 x_3=x_4=x_5=0 \quad &, \quad y_3  + y_4 + y_5 \geq 0 
\end{align*}
\p
با توجه به اینکه 
مجموع ساعات کارمند اول 6 ساعت و کارمند دوم 9 ساعت است، دو معادله سیاله 
(1) 
و
(2)
 را در نظر می‌گیریم:
\begin{align*}
&x_1 + x_2  = 6 \quad ;\quad x_1 \geq 2 , x_2 \geq  0 \quad (1)\\
 \Rightarrow \quad  &x_1 = 6 - x_2 \xLongrightarrow{\text{$x_1 \geq 2$}}  0 \leq x_2 \leq 4  \\
&y_2 + y_3 + y_4 + y_5 = 9\quad ;\quad y_2 = 2x_2 , y_3 + y_4 + y_5  \geq 0\quad (2)\\
 \Rightarrow\quad &y_3 + y_4 + y_5 = 9 - y_2 \xLongrightarrow{\text{$y_3 + y_4 + y_5  \geq 0$}} 0 \leq y_2 \leq 9 
\end{align*}
\p
با فرض اینکه
$x_2 = j$
باشد،
به ازای هر مقدار ممکن  
$j$
تنها
\underline{یک جواب}
برای
معادله سیاله‌ی 
 $$x_1 = 6 - j$$
 داریم.
 حال با توجه به اینکه 
 $y_2 = 2x_2 = 2j$،
 معادله‌ سیاله‌ی دوم به شكل 
 $$y_3 + y_4 + y_5 = 9 - 2j$$
 بازنویسی شده و به ازای هر 
 $j$
 به صورت
 \underline{ $\binom{9 - 2j + 2}{2}$}
 محاسبه می‌گردد.
\p
در نهایت طبق اصل جمع تعداد پاسخ‌ها برای همه‌ی مقادیر ممکن 
$j$
برابر است با: 
$$\sum_{j=0}^{4}\binom{11 - 2j}{2} = 125$$

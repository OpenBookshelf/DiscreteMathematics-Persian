\p
ابتدا تعداد جایگشت‌های  حروف غیر از 
$a$ که اولین $m$
 قبل از اولین $n$ 
 و چسبیده به آن آمده ‌باشد را می‌شماریم.
به این منظور عبارت $mn$ را یک عبارت مستقل فرض می‌کنیم.
حال 7 جایگاه برای حروف
$m$، 
$n$،
$g$، 
$e$، 
$e$، 
$t$
و عبارت
 $mn$ 
در نظر می‌گیریم. 
به 
\underline{${7\choose 4}$ حالت}
 4 جایگاه آن را برای قرار دادن حروف 
$g$،
$e$، 
$e$، 
$t$
انتخاب می‌کنیم.
 سپس طبق جایگشت حروف باتکرار، این حروف را به
 \underline{$\frac{4!}{2!}$ حالت}
در جایگاه‌های انتخاب شده می‌چینیم.
حال با توجه به خواسته‌ی سوال، در میان 3 جایگاه موجود،  
جایگاه اول متعلق به عبارت
 $mn$ می‌باشد.
  در نهایت به
 \underline{2 حالت}
 دو حرف
 $m$ و $n$
را در 2 جایگاه باقی‌مانده 
 قرار می‌دهیم.
اکنون باید حرف $a$ را در عبارت قرار دهیم.
به این منظور، دو حالت داریم:
\begin{enumerate}
  \item 
  
  هر دو حرف 
  $a$ بعد از دو حرف
  $n$ قرار داشته باشد. در این صورت کافیست به
  \underline{1 حالت}
  دو حرف 
  $a$ را بعد از دو حرف
  $n$ 
  موجود در جایگشت قرار دهیم.
  \item
  
  فقط یک عبارت 
  $na$
  در کلمه داشته‌ باشیم. در این صورت ابتدا به
  \underline{${2\choose 1}$ حالت}
  از میان 2 جایگاه موجود بعد از دو حرف
  $n$،  
  یکی را انتخاب کرده و یک حرف
  $a$
   را در آن قرار می‌دهیم. 
    سپس از میان ۷ جایگاه ممکن بین عبارات،
   به
   \underline{${7\choose 1}$ حالت}
   یک جایگاه را برای حرف 
   $n$ باقی‌مانده انتخاب می‌کنیم.
   توجه شود که جایگاهی بین دو حرف عبارات 
   $mn$ و $na$ 
   در نظر نمی‌گیریم.
   هم‌چنین مجاز به قرار دادن حرف 
   $a$ بعد از حرف $n$
    باقی‌مانده نیستیم چرا که این حالت در قسمت «آ» شمرده شده است.
\end{enumerate}
\p
در نتیجه پاسخ نهایی برابر است با:
$${7\choose 4} \times \frac{4!}{2!} \times 2 \times (\underbrace{1}_{\text{آ}} + \underbrace{{2\choose 1}{7\choose 1}}_{\text{ب}})$$

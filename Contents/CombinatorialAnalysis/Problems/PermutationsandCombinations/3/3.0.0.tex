  
    1. 
    \p
    تعداد جایگشت‌های حروف غیر از $m$ و $n$ طبق جایگشت حروف با تکرار، برابر $\frac{6!}{2!2!}$ است
    (هر یک از حروف $a$ و $e$ دوبار تکرار شده‌اند).
    حال جایگاه حروف $m$ و $n$ را تعیین می‌کنیم.
    برای این کار، چهار حالت داریم:
    \begin{enumerate}
      \item 
      حالتی که در آن هر کدام از عبارات $ma$ و $ne$
      دوبار تکرار شوند. بدین منظور، به
      \underline{1 حالت} 
       دو حرف 
      $m$ 
      را قبل از دو حرف 
      $a$
      قرار می‌دهیم.
      به طور مشابه دو حرف
       $n$ 
     به
     \underline{1 حالت}
       قبل از دو حرف 
       $e$
       قرار می‌گیرد.
      بدین ترتیب تعداد حالات ممکن موجود در این حالت 
      $1\times1$
      خواهد بود.
      
      \item
      حالتی که در آن عبارت $ma$ دو بار و عبارت $ne$ 
      یک بار آمده باشد. در این صورت مانند قسمت آ به
      \underline{1 حالت}  
       دو حرف 
      $m$
      را قبل از دو حرف 
      $a$
      قرار می‌دهیم.
      همچنین طبق حالت بیان شده، یکی از حروف 
      $n$ باید قبل از یکی از دو حرف $e$
      بیاید. در نتیجه
      \underline{2 حالت}
      برای قرار دادن یک حرف $n$ داریم. 
       حرف دوم 
      $n$
      را می‌توان به 
      \underline{6 حالت}
       در 6 جایگاه قبل و بعد از عبارات و حروف
      $ma$, 
      $ma$, 
      $ne$, 
      $t$,   
      $g$
      قرار داد.
      توجه کنید ما مجاز نیستیم حرف دوم
      $n$
      را قبل از حرف $e$
      باقی‌مانده در جایگشت حروف قرار دهیم زیرا به حالت آ می‌رسیم.
     در نهایت تعداد جایگشت‌های ممکن در این حالت به صورت 
      $1\times 2\times 6$
      خواهد بود.

      \item
      حالتی که در آن عبارت $ma$ یک بار و عبارت $ne$
    دو بار آمده باشد.
      طبق تقارن موجود براي 
      $\{m,a\}$ و
      $\{n,e\}$، 
    تعداد جایگشت‌های ممکن در این حالت نیز مانند قسمت ب برابر 
    $1\times 2\times 6$ 
      خواهد بود.
      
      \item
      حالت نهایی مختص به زمانی است که هریک از عبارات $ma$ و $ne$ تنها یک بار آمده باشد.
      بدین منظور ابتدا به
      \underline{2 حالت}
      حرف 
      $n$
      را قبل یکی از دو حرف 
      $e$
      و به
      \underline{2 حالت}  
      حرف  
      $m$
      را قبل يكي از دو حرف 
      $a$ 
      قرار می‌دهیم.
      برای قرار دادن حرف $m$ باقی‌مانده،
      \underline{6 حالت} 
       داریم.
      قبل و بعد همه‌ی عبارات و حروف 
      $ma$، $ne$، $t$، $g$، $e$
      می‌توانیم حرف $m$ را قرار دهیم.
      دقت شود که قبل از حرف $a$ مجاز به قرار دادن حرف $m$ نیستیم چرا که این حالت قبلا شمرده شده‌است.

      همچنین به 
      \underline{7 حالت}
      می‌توانیم قبل و بعد عبارات و حروف
      $ma$،
      $ne$،
      $m$،
      $t$، $g$، $a$
      حرف $n$ را قرار دهیم.
      قابل ذکر است که در این حالت نمی‌توان قبل حرف $e$
      حرف $n$ را قرار داد چون این حالت قبلا شمرده شده است.
      بدین ترتیب حالت نهایی به صورت 
      $2\times 2 \times 6 \times 7$
      خواهد بود.
    \end{enumerate}
    \p
    پس تعداد کل حالات برای قرار دادن حروف 
    $m$ و $n$ 
     برابر است با:
    $$\underbrace{1}_{\text{آ}} + \underbrace{2\times 6}_{\text{ب}} + \underbrace{2\times 6}_{\text{ج}}+ \underbrace{2\times 2 \times 6 \times 7}_{\text{د}}‌=‌193$$
    می‌باشد.
    \p
    در نتیجه پاسخ نهایی به صورت زیر می‌باشد:
    $$\frac{6!}{2!2!} \times 193$$
    \\2.
    \p
    ابتدا همه‌ی حروف غیر از دو حرف $m$ را مطابق شرایط سوال می‌چینیم.
    از میان ۸ جایگاه موجود برای ۸ حرف، به 
    \underline{${8\choose 4}$ حالت}، 
    4
   جایگاه را انتخاب کرده و به
   \underline{1 حالت}
   حروف صدادار را می‌چینیم.
    سپس چهار حرف صامت باقی‌مانده
    را طبق جایگشت حروف باتکرار به
     \underline{$\frac{4!}{2!}$ حالت}
    در ۴ جایگاه باقی‌مانده قرار می‌دهیم.
    \p
    حال به دو صورت می‌توانیم دو حرف $m$ را قرار دهیم به نحوی که حداقل یک عبارت 
    $ma$ تولید شود:
    \begin{enumerate}
      \item 

      دو عبارت  $ma$  در جایگشت داشته باشیم.
       بدین ترتیب تنها کافیست به 
      \underline{1 حالت}
       در دو جایگاه قبل از دو حرف 
       $a$،  
       دو حرف 
       $m$ را قرار دهیم.

      \item
      فقط یک عبارت $ma$ داشته‌ باشیم. 
      در این صورت ابتدا باید به
      \underline{2 حالت}
      حرف 
      $m$ 
      را قبل یکی از دو حرف 
      $a$ 
      قرار دهیم. 
      سپس
       دومین حرف
      $m$
      را
      به
      \underline{8 حالت}
      در یکی از 8 جایگاه قبل و بعد از عبارت 
      $ma$
       و حروف 
       $n$, 
       $n$, 
       $e$, 
       $e$, 
       $t$, 
       $g$ 
       قرار دهیم.
       دقت شود که مجاز به قرار دادن 
      $m$
      قبل از حرف 
      $a$ 
      باقی‌مانده 
      نیستیم چرا که این حالت در قسمت آ شمرده شده است.
    \end{enumerate}
    \p
    در نتیجه پاسخ نهایی به صورت زیر خواهد بود:
    $${8\choose 4} \times 1 \times \frac{4!}{2!} \times (\underbrace{1}_{\text{آ}} + \underbrace{2 \times 8}_{\text{ب}})$$
    3.
    \p
    ابتدا تعداد جایگشت‌های  حروف غیر از $a$ که اولین $m$ قبل از اولین $n$ و چسبیده به آن آمده ‌باشد را می‌شماریم.
    به این منظور عبارت $mn$ را یک عبارت مستقل در نظر می‌گیریم.
    حال به غیر از دو حرف $m$ و $n$ و عبارت
     $mn$، 
    چهار حرف داریم که در آن $e$ دوبار تکرار شده است.
    از هفت جایگاه موجود برای کلمات، چهار جایگاه آن را برای قرار دادن این حروف انتخاب می‌کنیم(1)
    
    \p
    سپس طبق جایگشت باتکرار، آن‌ها را در جایگاه‌های انتخاب شده می‌چینیم.(2)
    
    \p
    حال در میان سه جایگاه باقی‌مانده، طبق خواسته‌ی سوال، جایگاه اول متعلق به $mn$ می‌باشد
    (اولین حرف $m$ قبل از اولین حرف $n$ وچسبیده به آن قرار داشته باشد)
    و در دو جایگاه باقی‌مانده دو حرف $m$ و $n$ را به دو صورت می‌توانیم قرار دهیم.(3)
    \p
    بدین ترتیب حالات قرار دادن حروف به غیر از $a$ به صورت زیر خواهد بود:
    $$\underbrace{{7\choose 4}}_{1} \times \underbrace{\frac{4!}{2!}}_{2} \times \underbrace{2}_{3}$$
    
    حال باید حرف $a$ را در عبارت قرار دهیم.
    به این منظور مشابه قسمت قبل، دو حالت داریم:
    \begin{enumerate}
      \item 
      
      هر دو حرف $a$ بعد از دو حرف $n$ قرار داشته باشد. در این صورت کافیست به 1 حالت دو حرف $a$ را بعد از دو حرف $n$ موجود در جایگشت قرار دهیم.(1)
      \item
      
      فقط یک عبارت $na$ در کلمه داشته‌باشیم. در این صورت ابتدا از میان دو جایگاه موجود پس از دو حرف $n$ در کلمه، یکی را انتخاب می‌کنیم و یک حرف $a$ را قرار می‌دهیم. سپس از میان ۷ جایگاه ممکن، یکی را برای حرف $n$ باقی‌مانده انتخاب می‌کنیم.
      توجه شود که عبارت $mn$ و $na$ ایجاد شده را یک کلمه‌ در نظر می‌گیریم. هم‌چنین مجاز به قرار دادن حرف $a$ پس از حرف $n$ باقی‌مانده نیستیم چرا که این حالت در قسمت آ شمرده شده است.(2)
    \end{enumerate}
    \p
    بنابراین تعداد حالات قرار دادن حرف $a$ به صورت زیر می‌باشد:
    $$\underbrace{1}_{1} + \underbrace{{2\choose 1}{7\choose 1}}_{2} = 15$$ 
    پس پاسخ نهایی برابر است با:
    
    $${7\choose 4} \times \frac{4!}{2!} \times 2 \times 15$$

    4.
    \p
    تعداد جایگشت حروف غیر از $m$
    طبق حروف باتکرار به صورت زیر خواهد بود:
    (حروف $n$ و $e$ و $a$ هر کدام دوبار تکرار شده‌اند)
    $$\frac{8!}{2!2!2!}$$
    
    از طرفی تعداد جایگشت‌هایی از میان جایگشت‌های فوق که عبارت $gn$ داشته باشد را محاسبه می‌کنیم. برای این کار کافیست عبارت $gn$ را یک حرف در نظر بگیریم. حال طبق جایگشت حروف باتکرار، دو حرف $a$ و $e$ هرکدام دوبار تکرار شده‌اند. پس تعداد جایگشت‌هایی که این شرایط را داشته‌باشند به صورت زیر است:
    $$\frac{7!}{2!2!}$$
    
    حال به جایگذاری دو حرف $m$ باقی‌مانده می‌پردازیم. برای این کار، دو حالت داریم:
    \begin{enumerate}
      \item 
      
      عبارت $gn$ در کلمه‌ی فعلی نباشد:
            در این صورت، یا یکی از ۸ جایگاه ممکن را انتخاب می‌کنیم و هر دو حرف $m$ را دقیقا کنار هم می‌گذاریم
      (1).
            و یا اینکه از ۸ جایگاه موجود، دو جایگاه را انتخاب کرده و دو حرف $m$ را قرار می‌دهیم. دقت شود که دو حرف یکسان می‌باشد و ترتیب جایگذاری آن اهمیتی ندارد (2).
      قابل ذکر است که طبق خواسته‌ی سوال نباید عبارت  $mg$ در جایگشت موجود داشته‌باشیم، به همین دلیل، ۸ انتخاب داریم(قبل و بعد تمام حروف موجود به جز قبل از حرف $g$ در جایگشت).
      $$\underbrace{8\choose 1}_{1} + \underbrace{8\choose 2}_{2}$$
      \item
      
      عبارت $gn$ در کلمه‌ی فعلی وجود داشته‌باشد:
      در این صورت حتما باید یک حرف $m$ میان دو حرف 
      $g$ و $n$ قرار دهیم تا عبارت $gn$ دیگر وجود نداشته باشد.
      سپس از ۸ جایگاه باقی‌مانده یک جایگاه را برای قرار دادن حرف دوم $m$ انتخاب می‌کنیم. دقت شود تعداد انتخاب‌های موجود برای قرار دادن حرف $m$ هشت‌تاست چرا که علاوه بر آنکه قبل حرف $g$ نمی‌توانیم حرف $m$ را قرار دهیم، قرار دادن آن قبل و بعد حرف $m$ موجود در کلمه، یک حالت به شمار می‌آید. پس در نهایت ۸ انتخاب خواهیم داشت.
      $${8\choose 1}$$
    \end{enumerate}
    \p
    بدین ترتیب پاسخ نهایی به صورت زیر خواهد بود:
    $$(\frac{8!}{2!2!2!} - \frac{7!}{2!2!})({8\choose 1} + {8\choose 2}) + \frac{7!}{2!2!} \times {8\choose 1}$$
  
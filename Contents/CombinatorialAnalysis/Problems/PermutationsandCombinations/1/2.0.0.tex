\p
فرض می‌کنیم
$n$
حداکثر تعداد زیرمجموعه‌هایی باشد که شرایط مسئله را دارند. مجموعه‌ی این زیرمجموعه‌ها را
$A$
می‌نامیم. می‌توان گفت هر زیرمجموعه که از اعضای
$A$
نباشد، باید حداقل با یکی از اعضای
$A$
دقیقا سه عضو مشترک داشته باشد؛
در غیر اینصورت می‌توانیم بدون از بین بردن شرایط مسئله، آن زیرمجموعه را به مجموعه‌ی
$A$
اضافه کنیم که با حداکثر بودن
$n$
در تناقض است.
به ازای هر عضو از مجموعه‌ی 
$A$
مانند
$X$،
تعداد
$k$
زیرمجموعه‌ی ۴ عضوی از مجموعه اعداد ۱ تا ۱۰۰ وجود دارد که با
$X$
دقیقا سه عضو مشترک دارند. برای بدست آوردن مقدار
$k$،
از شمارش استفاده می‌کنیم. فرض کنید می‌خواهیم زیرمجموعه‌ی ۴ عضوی
$Y$
را با شرایط گفته شده بسازیم. ابتدا به ۴ حالت، عضوی از
$X$
را انتخاب می‌کنیم که در مجموعه‌ی
$Y$
دستخوش تغییر شده است. برای مقدار جدید این عضو، ۹۶ انتخاب
(اعداد ۱ تا ۱۰۰ بجز ۴ عضو $X$)
خواهیم داشت. بنابراین:
$$k = 4 \times 96$$
مجموعه‌ی این
$4 \times 96$
زیرمجموعه به ازای
$X$
از مجموعه‌ی
$A$
را
$B_X$
‌می‌نامیم.
اجتماع مجموعه‌ی
$A$
با مجموعه‌های
$B_X$
به ازای تمام
$X$های
ممکن، تمام زیرمجموعه‌های ۴ عضوی از اعداد
$$\{1,2,...,100\}$$
را تشکیل می‌دهد.
به بیانی دیگر، اگر
$F$
مجموعه‌ی تمام مجموعه‌های ۴ عضوی باشد:
$$(\bigcup\limits_{X \in A} B_X) \cup A = F$$
پس:
$$|F| \leq n \times (1 + 4 \times 96)$$
علامت کوچکتر یا مساوی به این خاطر است که ممکن است تعدادی از این زیر مجموعه‌ها را چند بار شمرده باشیم.
همچنین می‌دانیم:
$$|F| = {100 \choose 4} $$
$$\Rightarrow {100 \choose 4} \leq n \times (1 + 4 \times 96)$$
$$\Rightarrow n \geq 10185$$
بنابراین می‌توان 10000 زیرمجموعه با خاصیت گفته شده داشت.
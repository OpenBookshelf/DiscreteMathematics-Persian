  \p
    تعداد جایگشت‌های حروف غیر از $m$ و $n$ طبق جایگشت حروف با تکرار، برابر $\frac{6!}{2!2!}$ است
    (هر یک از حروف $a$ و $e$ دوبار تکرار شده‌اند).
    حال جایگاه حروف $m$ و $n$ را تعیین می‌کنیم.
    برای این کار، چهار حالت داریم:
    \begin{enumerate}
      \item 
      حالتی که در آن هر کدام از عبارات $ma$ و $ne$
      دوبار تکرار شوند. بدین منظور، به
      \underline{1 حالت} 
       دو حرف 
      $m$ 
      را قبل از دو حرف 
      $a$
      قرار می‌دهیم.
      به طور مشابه دو حرف
       $n$ 
     به
     \underline{1 حالت}
       قبل از دو حرف 
       $e$
       قرار می‌گیرد.
      بدین ترتیب تعداد حالات ممکن موجود در این حالت 
      $1\times1$
      خواهد بود.
      
      \item
      حالتی که در آن عبارت $ma$ دو بار و عبارت $ne$ 
      یک بار آمده باشد. در این صورت مانند قسمت «آ» به
      \underline{1 حالت}  
       دو حرف 
      $m$
      را قبل از دو حرف 
      $a$
      قرار می‌دهیم.
      همچنین طبق حالت بیان شده، یکی از حروف 
      $n$ باید قبل از یکی از دو حرف $e$
      بیاید. در نتیجه
      \underline{2 حالت}
      برای قرار دادن یک حرف $n$ داریم. 
       حرف دوم 
      $n$
      را می‌توان به 
      \underline{6 حالت}
       در 6 جایگاه قبل و بعد از عبارات و حروف
      $ma$، 
      $ma$، 
      $ne$، 
      $t$،  
      $g$
      قرار داد.
      توجه کنید ما مجاز نیستیم حرف دوم
      $n$
      را قبل از حرف $e$
      باقی‌مانده در جایگشت حروف قرار دهیم زیرا به حالت آ می‌رسیم.
     در نهایت تعداد جایگشت‌های ممکن در این حالت به صورت 
      $1\times 2\times 6$
      خواهد بود.

      \item
      حالتی که در آن عبارت $ma$ یک بار و عبارت $ne$
    دو بار آمده باشد.
      طبق تقارن موجود براي 
      $\{m,a\}$ و
      $\{n,e\}$، 
    تعداد جایگشت‌های ممکن در این حالت نیز مانند قسمت «ب» برابر 
    $1\times 2\times 6$ 
      خواهد بود.
      
      \item
      حالت نهایی مختص به زمانی است که هریک از عبارات $ma$ و $ne$ تنها یک بار آمده باشد.
      بدین منظور ابتدا به
      \underline{2 حالت}
      حرف 
      $n$
      را قبل یکی از دو حرف 
      $e$
      و به
      \underline{2 حالت}  
      حرف  
      $m$
      را قبل يكي از دو حرف 
      $a$ 
      قرار می‌دهیم.
      می‌توانیم حرف
       را به $m$ باقی‌مانده،
      \underline{6 حالت} 
       در 6 جایگاه
      قبل و بعد همه‌ی عبارات و حروف 
      $ma$، $ne$، $t$، $g$، $e$
       قرار دهیم.
      دقت شود که قبل از حرف $a$ مجاز به قرار دادن حرف $m$ نیستیم چرا که این حالت قبلا شمرده شده‌است.
      همچنین به 
      \underline{7 حالت}
      می‌توانیم قبل و بعد عبارات و حروف
      $ma$،
      $ne$،
      $m$،
      $t$، $g$، $a$
      حرف $n$ را قرار دهیم.
      قابل ذکر است که در این حالت نمی‌توان قبل حرف $e$
      حرف $n$ را قرار داد چون این حالت قبلا شمرده شده است.
      بدین ترتیب حالت نهایی به صورت 
      $2\times 2 \times 6 \times 7$
      خواهد بود.
    \end{enumerate}
    \p
    پس تعداد کل حالات برای قرار دادن حروف 
    $m$ و $n$ 
     برابر است با:
    $$\underbrace{1}_{\text{آ}} + \underbrace{2\times 6}_{\text{ب}} + \underbrace{2\times 6}_{\text{ج}}+ \underbrace{2\times 2 \times 6 \times 7}_{\text{د}}‌=‌193$$
    
    \p
    در نتیجه پاسخ نهایی به صورت زیر می‌باشد:
    $$\frac{6!}{2!2!} \times 193$$
  

\p
طبق رابطه‌ی تعداد جایگشت‌های خطی با اعضای تکراری،
حروف بی‌صدا را به 
\underline{$\frac{7!}{3!2!}$ حالت}
 می‌توانیم در یک ردیف قرار دهیم.
 بار ديگر دو حرف
$n$ 
را یک حرف فرض می‌کنیم. پس یکی از تعداد حروف کم شده و جایگشت 6 حرف بی‌‌صدا که 3 تا از آن‌ها تکراری هستند را 
محاسبه می‌کنیم.
در نتیجه در 
\underline{$\frac{6!}{3!}$ حالت} 
 حروف 
$n$ مجاورند.

\begin{itemize}
\item 
اگر دو حرف
 $n$
 مجاور باشند، 
از بین 8 جایگاه قبل و بعد حروف بی‌صدا،
 حروف صدادار می‌توانند در ۵ جایگاه قرار بگیرند.
پس داریم:
$$x_1 + x_2 + \ldots + x_5 = 6$$
طبق تعداد پاسخ‌های معادله سیاله خطی با ضرایب واحد،
اين عمل به 
\underline{$\binom{10}{4}$}
حالت ممكن
است.
با محاسبه جایگشت‌های حروف صدادار، جواب نهایی این بخش برابر
\underline{$\binom{10}{4}\times\frac{6!}{3!}$}
است.
\item 
اگر دو حرف
 $n$
 مجاور نباشند، حروف صدادار می‌توانند در ۴ جایگاه از ۸ جایگاه  قبل و بعد حروف بی‌صدا قرار بگیرند. پس داریم:
$$x_1 + x_2 + \ldots + x_4 = 6$$
طبق تعداد پاسخ‌های معادله سیاله خطی با ضرایب واحد،
اين عمل به 
\underline{$\binom{9}{3}$}
حالت ممكن
است.
در نهایت با محاسبه جایگشت حروف صدادار، جواب نهایی این بخش برابر
\underline{$\binom{9}{3}\times\frac{6!}{3!}$}
است.
\end{itemize}
\p
پاسخ نهایی از جمع دو حالت بالا بدست می‌آید:
$$\frac{6!}{3!}\times\binom{10}{4}\times\frac{6!}{3!}+ (\frac{7!}{3!2!} - \frac{6!}{3!})\times\binom{9}{3}\times\frac{6!}{3!}$$

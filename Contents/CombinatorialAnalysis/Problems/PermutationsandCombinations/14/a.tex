\p
تعداد جایگشت‌های حروف غیر از $m$
طبق جایگشت‌ حروف باتکرار به صورت 
$\frac{8!}{2!2!2!}$
می‌باشد
(حروف $n$ ، $e$ و $a$ هر کدام دوبار تکرار شده‌اند).
 همچنین تعداد جایگشت‌هایی از میان جایگشت‌های فوق که عبارت
 $gn$ 
 را داشته باشند، برابر
 \underline{$\frac{7!}{2!2!}$} 
  می‌باشد 
  (جایگشت عبارت
  $gn$
  و حروف
  $a$، 
  $a$، 
  $e$، 
  $e$، 
  $n$، 
  $t$
  )
  . 
 در نتیجه، طبق اصل متمم تعداد جایگشت‌هایی که فاقد عبارت 
  $gn$
  و حرف 
  $m$   
  هستند به صورت  
  \underline{$\frac{8!}{2!2!2!} - \frac{7!}{2!2!}$} 
  محاسبه می‌شود.
حال به جایگذاری دو حرف 
$m$
 باقی‌مانده می‌پردازیم. برای این کار، دو حالت داریم:
\begin{enumerate}
  \item 
  عبارت 
  $gn$ 
 در کلمه‌ی فعلی نباشد؛
 در این صورت از 8 جایگاه ممکن 
  قبل و بعد تمام حروف موجود به جز قبل از حرف $g$ در جایگشت، 
یکی را به 
\underline{${8\choose 1}$ حالت}
 انتخاب می‌کنیم و هر دو حرف $m$ را دقیقا کنار هم می‌گذاریم
 و یا اینکه از ۸ جایگاه موجود، به 
\underline{${8\choose 2}$ حالت}
دو جایگاه را انتخاب کرده و دو حرف
 $m$
 را قرار می‌دهیم.
دقت شود که دو حرف یکسان می‌باشند و ترتیب جایگذاری آن‌ها اهمیتی ندارد.

  \item
  عبارت $gn$ 
  در کلمه‌ی فعلی وجود داشته ‌باشد؛
  در این صورت حتما باید یک حرف $m$ میان دو حرف 
  $g$ و $n$ قرار دهیم تا عبارت $gn$ دیگر وجود نداشته باشد.
  سپس از ۸ جایگاه باقی‌مانده به 
  \underline{${8\choose 1}$ حالت}
  یک جایگاه را برای قرار دادن حرف دوم $m$ انتخاب می‌کنیم. دقت شود تعداد انتخاب‌های موجود برای قرار دادن حرف $m$ هشت‌تاست چرا که علاوه بر آنکه قبل حرف $g$ نمی‌توانیم حرف $m$ را قرار دهیم، قرار دادن آن قبل و بعد حرف $m$ موجود در کلمه، یک حالت به شمار می‌آید. پس در نهایت ۸ انتخاب خواهیم داشت.
  
\end{enumerate}
\centerimage{1.0}{./0.png}
\p
بدین ترتیب پاسخ نهایی به صورت زیر خواهد بود:
$$(\frac{8!}{2!2!2!} - \frac{7!}{2!2!})(\underbrace{{8\choose 1} + {8\choose 2}}_{\text{آ}}) + \frac{7!}{2!2!} \times \underbrace{{8\choose 1}}_{\text{ب}}$$

\p
 با توجه به صورت مسئله، در جایگاه اول و آخرِ قفسه باید کتاب‌های ریاضی قرار گیرند.
  حال یکی در میان جایگاه‌هایی را برای کتاب‌های فیزیک خالی گذاشته و در بقیه‌ی خانه‌ها سایر کتاب‌های ریاضی و کتاب‌های شیمی را قرار می‌دهیم.
 در این صورت حالت قرار گرفتن کتاب‌ها به شکل زیر می‌باشد:
\vspace*{+0.5cm}
\centerimage{0.7}{./0.png}  
\vspace*{+0.4cm}
\p
توجه کنید که در هر یک از جایگاه‌های مشخص شده برای کتاب‌های ریاضی و شیمی، به جز جایگاه اول و آخر، امکان قرار گرفتن یک کتاب ریاضی یا یک کتاب شیمی وجود دارد. 
در نتیجه با
\underline{ترتیب 2 از 7}
کتاب‌های شیمی را در دو جایگاه ممکن 
 قرار می‌دهیم.
 سپس کتاب‌‌های ریاضی را به
 \underline{$7!$ حالت}
 در خانه‌های مناسب
 می‌چینیم.
 در نهایت با
 \underline{ترتیب 3 از 8}
  کتاب‌های فیزیک را
در 3 جایگاه قرار می‌دهیم.
  طبق اصل ضرب جواب نهایی برابر است با:
$$\frac{7!}{5!}\times7! \times \frac{8!}{5!}$$
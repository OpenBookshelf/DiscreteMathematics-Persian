\p
	اگر نام این نقاط را به ترتیب در یک سطر بنویسیم و سپس
	یک جایگشت خطی از آن‌ها را زیر آن‌ها بنویسیم و این دو سطر را «ترتیب درهم‌ریختگی» بنامیم، می‌توان
	ادعا کرد که اگر از هر نقطه‌ی
	$x$
	خطی جهت‌دار به نقطه‌ای که در ترتیب درهم‌ریختگی، نام آن
	زیر نام
	$x$
	نوشته شده‌است وصل کنیم، شرایط مسئله برقرار است.
	اثبات این ادعا ساده است زیرا که در سطر اول ترتیب درهم‌ریختگی،
	نام هر نقطه تنها یکبار آمده پس از هر نقطه فقط یک خط خارج می‌شود.
	همچنین در سطر دوم ترتیب درهم‌ریختگی، نام هر نقطه دقیقا یکبار ظاهر، پس
	به هر نقطه فقط و فقط یک خط وارد می‌شود. بنابراین با قرار دادن هر جایگشت خطی
	از نام نقاط در سطر دوم ترتیب درهم‌ریختگی، به یک جواب جدید می‌رسیم.
	پس تعداد حالات ممکن برای این مسئله، تعداد جایگشت‌های خطی نام این
	$n$
	نقطه، یعنی 
	$n!$
	است.

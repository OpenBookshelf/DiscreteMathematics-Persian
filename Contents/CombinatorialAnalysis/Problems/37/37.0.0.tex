\p
برای حل این سوال از دوگانه‌شماری استفاده می‌کنیم. فرض کنید 
$2n$
پیراشکی با طعم‌های متفاوت داریم. 
$n$
تا از آن‌ها در جعبه اول و 
$n$
تای دیگر در جعبه دوم هستند. می‌خواهیم 
$2$
پیراشکی برای صبحانه انتخاب کنیم. به دو روش می‌توانیم این کار را انجام دهیم:
\begin{itemize}
    \item
    دو پیراشکی را از کل پیراشکی‌ها انتخاب کنیم:
    $$\binom{2n}{2}$$

    \item
    سه حالت برای انتخاب داریم:
        \begin{enumerate}
        \item 
        می‌توانیم به 
        $\binom{n}{2}$
        طریق هر دو پیراشکی را از جعبه اول انتخاب کنیم.

        \item
        می‌توانیم به
        $\binom{n}{2}$
        طریق هر دو پیراشکی را از جعبه دوم انتخاب کنیم.

        \item
        می‌توانیم به 
        $\binom{n}{1}$
        طریق یک پیراشکی را از جعبه اول و دیگری را
        به
        $\binom{n}{1}$
        طریق از جعبه دوم انتخاب کنیم.
        \end{enumerate}

    \p
    طبق اصل جمع، تعداد کل حالات در این روش شمارش برابر است با:
    $$\binom{n}{2} + \binom{n}{2} + \binom{n}{1} \binom{n}{1} = 2\binom{n}{2} + n^2$$
\end{itemize}

\p
با توجه به معادل بودن دو روش فوق، طبق دوگانه‌شماری داریم:
$$\binom{2n}{2} = 2\binom{n}{2} + n^2$$
\p
برای حل این سوال از دوگانه‌شماری استفاده می‌کنیم. فرض کنید 
$2n$
پیراشکی با طعم‌های متفاوت داریم. 
$n$
تا از آن‌ها در جعبه اول و 
$n$
تای دیگر در جعبه دوم هستند. می‌خواهیم 
$2$
پیراشکی برای صبحانه انتخاب کنیم. به دو روش می‌توانیم این کار را انجام دهیم:
\begin{itemize}
\item
روش اول:
 دو پیراشکی را از کل پیراشکی‌ها انتخاب کنیم:
 $$\binom{2n}{2}$$

\item
روش دوم:
$$\underbrace{\binom{n}{2}}_{\text{آ}} + \underbrace{\binom{n}{2}}_{\text{ب}}  + \underbrace{\binom{n}{1}\binom{n}{1}}_{\text{ج}} = \underbrace{2\binom{n}{2} + n^2}_{\text{د}}$$
\begin{enumerate}
\item 
می‌توانیم هر دو پیراشکی را از جعبه اول انتخاب کنیم.

\item
می‌توانیم هر دو پیراشکی را از جعبه دوم انتخاب کنیم.

\item
می‌توانیم یک پیراشکی را از جعبه اول و دیگری را از جعبه دوم انتخاب کنیم.

\item
 طبق به اصل جمع، تعداد کل حالات در این روش شمارش برابر است با:
$$2\binom{n}{2} + n^2$$
\end{enumerate}
\end{itemize}

\p
با توجه به معادل بودن دو روش فوق، طبق دوگانه‌شماری داریم:
$$\binom{2n}{2} = 2\binom{n}{2} + n^2$$
\p
برای پاسخ به این سوال از اصل شمول و عدم شمول استفاده می‌کنیم.
\begin{itemize}
    \item
    فرض کنید تعداد حالاتی که شغل افراد اتاق 
    $A$
    متفاوت باشد را 
    با 
    $n_1$،
    تعداد حالاتی که شغل افراد اتاق 
    $B$
    متفاوت باشد را 
    با 
    $n_2$
     و تعداد حالاتی که شغل افراد اتاق 
     $C$
     متفاوت باشد را 
     با 
     $n_3$
     نشان دهیم.
     در این صورت تعداد حالات قرار دادن  3 نفر با شغل‌های متفاوت در حداقل یک اتاق برابر است با:
    $$|n_1| + |n_2| + |n_3| = \underbrace{\binom{3}{1}}_{\text{آ}}\underbrace{\binom{5}{1}^3}_{\text{ب}}\underbrace{\binom{12}{3}\binom{9}{3}\binom{6}{3}\binom{3}{3}}_{\text{ج}}$$
    \begin{enumerate}
    \item 
    یکی از 3 اتاق 
    $A$،
     $B$ 
    و $C$
     را انتخاب می‌کنیم.
     \item 
     از هر شغل یک نفر را وارد آن اتاق می‌کنیم.
     \item 
     بقیه افراد را سه تا سه تا وارد سایر اتاق‌ها می‌کنیم.
    \end{enumerate}
    \item
    تعداد حالاتی که در حداقل 2 اتاق افراد با شغل‌های متفاوت قرار گیرند برابر است با:
    $$|n_1 \cap n_2| + |n_1 \cap n_3| + |n_2 \cap n_3| = \underbrace{\binom{3}{2}}_{\text{آ}}\underbrace{\binom{5}{1}^3\binom{4}{1}^3}_{\text{ب}}\underbrace{\binom{9}{3}\binom{6}{3}\binom{3}{3}}_{\text{ج}}$$
    \begin{enumerate}
    \item 
     دو اتاق مشخص را انتخاب می‌کنیم.
     \item 
     از هر شغل یک نفر را وارد اتاق اول و یک نفر را وارد اتاق دوم می‌کنیم.
     \item 
     سایر افراد را سه تا سه تا وارد اتاق‌های خالی می‌کنیم.
    \end{enumerate}
    \item
    در نهایت تعداد حالاتی که در 3 اتاق شغل تمامی افراد متفاوت باشد برابر است با:
    $$|n_1 \cap n_2 \cap n_3| = \underbrace{\binom{5}{1}^3\binom{4}{1}^3\binom{3}{1}^3}_{\text{آ}}\underbrace{\binom{6}{3}\binom{3}{3}}_{\text{ب}}$$
    \begin{enumerate}
    \item
    از هر شغل یک نفر را وارد اتاق اول، یک نفر را وارد اتاق دوم و یک نفر را وارد اتاق سوم می‌کنیم.
    \item
    در نهایت 6 نفر باقی‌مانده را وارد اتاق‌های
    $D$
     و
   $E$ 
    می‌کنیم.
    \end{enumerate}
    \end{itemize}
     پاسخ نهایی مسئله با استفاده از اصل شمول و عدم شمول برابر است با:
     \begin{align*}
    |n_1 \cup n_2 \cup n_3| &= 3\binom{5}{1}^3\binom{12}{3}\binom{9}{3}\binom{6}{3}\binom{3}{3}\\
    &-3\binom{5}{1}^3\binom{4}{1}^3\binom{9}{3}\binom{6}{3}\binom{3}{3}\\
    &+\binom{5}{1}^3\binom{4}{1}^3\binom{3}{1}^3\binom{6}{3}\binom{3}{3}
    \end{align*}
\SECTION{شمارش به کمک توابع مولد}
\p
مسائل شمارش را می‌توان با یک مدل‌سازی ساده، به کمک توابع مولد حل کرد.
این کار همیشه منجر به افزایش سهولت حل نمی‌شود اما در برخی از مسائل چاره‌گشاست.
نحوه این استفاده را با چند مثال نشان خواهیم داد.

\SUBSECTION{شمارش بدون ترتیب}

\subfile{./GeneratingFunctions/example1.tex}
\subfile{./GeneratingFunctions/example2.tex}
\subfile{./GeneratingFunctions/example3.tex}
\subfile{./GeneratingFunctions/example4.tex}

\SUBSECTION{شمارش با ترتیب}
\p
برای کمک گرفتن از توابع مولد در حل مسائل شمارشی که در آن‌ها ترتیب با اهمیت است،
ساده‌تر است اگر از \CROSSREF{تابع مولد نمایی} استفاده شود.
استفاده از این روش را با چند مثال نشان خواهیم داد.

\subfile{./ExponentialGeneratingFunctions/example1.tex}
\subfile{./ExponentialGeneratingFunctions/example2.tex}
\subfile{./ExponentialGeneratingFunctions/example3.tex}

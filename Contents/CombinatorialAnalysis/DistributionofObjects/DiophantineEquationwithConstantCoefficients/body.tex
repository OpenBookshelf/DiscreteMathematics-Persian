\SUBSECTION{معادله سیاله خطی با ضرایب واحد}

\subfile{./DiophantineEquationExtra.tex}

\begin{DEFINITION}
    \p
    به هر معادله‌ی چندجمله‌ای\footnote{حاصل جمع توان‌هایی از متغیر‌ها؛ در معادلات چندجمله‌ای،
    توابعی نمایی، لگاریتمی، سینوسی یا ... از متغیر‌ها مجاز نیستند.}
    خطی\footnote{توان تمام مجهولات واحد است.}
    که در آن ضریب تمام متغیر‌ها برابر یک باشد
    و متغیر‌ها فقط مجاز به اخذ مقادیر صحیح باشند،
    یک
    \FOCUSEDON{معادله}
    \FOCUSEDON{سیاله}\LTRfootnote{Diophantine Equation}
    \FOCUSEDON{خطی}\LTRfootnote{Linear Diophantine Equation}
    \FOCUSEDON{با}
    \FOCUSEDON{ضرایب}
    \FOCUSEDON{واحد}
    گفته می‌شود.
    شکل کلی این معادلات را می‌توان به صورت زیر نمایش داد:
    $$\sum\limits_{i}^{} {x_i} = s ; x_i \in \mathbb{Z}$$
\end{DEFINITION}

\p
این معادلات معمولا دارای چند پاسخ هستند.
به عنوان مثال معادله زیر را درنظر بگیرید:
$$x + y = 2$$
این یک معادله سیاله ساده است.
واضح است که هر زوج
$(a, 2-a)$
می‌تواند یک جواب معادله برای زوج
$(x,y)$
باشد. بنابراین تعداد پاسخ‌های این معادله بینهایت است.
بسیار پیش می‌آید که هدف ما یافتن پاسخ‌های معادله در مجموعه اعداد طبیعی یا حسابی باشد.
این شرایط زمانی پیش می‌آید که یک مسئله طبیعی ساده را با معادله سیاله مدل کنیم
(مثلا می‌خواهیم از انواع مهره‌ها، تعداد متفاوتی برداریم به طوری که در نهایت ۱۰ مهره داشته باشیم).
در این صورت تعداد جواب‌های معادله بسیار محدود‌تر خواهد شد.
همان مثال بالا را درنظر بگیرید؛ این معادله در مجموعه اعداد حسابی تنها ۳ پاسخ خواهد داشت.

\NOTE{معمولا در مواجهه با معادلات سیاله، هدف یافتن تعداد پاسخ‌های معادله است.}

\subfile{./Solving/body.tex}
\subfile{./Solving/example1.tex}
\subfile{./Solving/example2.tex}

\p
در بخش شمارش به کمک توابع مولد، با
\CROSSREF[روشی کلی برای حل معادلات سیاله خطی]{حل معادلات سیاله خطی به کمک توابع مولد}
(بدون فرض واحد بودن ضرایب)
آشنا خواهید شد.
\begin{PROBLEM}
  \p
    در یک مجلس رسمی، میز گردی وجود دارد که اسامی دعوت شدگان بر روی صندلی‌ها نوشته شده است.
    به چند طریق ممکن است مهمانان به دور این میز بنشینند به شرطی که تنها یک نفر
    بر روی صندلی خود بنشیند؟ تعداد مهمانان ۱۲ می‌باشد.

    \SOLUTION{
      \p
        با توجه به نام‌دار بودن صندلی‌ها، وجود حلقه در مسئله تفاوتی ایجاد نمی‌کند.
        کافی است در ابتدا یک نفر را انتخاب کرده و پس از آن، تعداد پریش‌های افراد باقیمانده را با توجه به رابطه‌ی بدست آمده در سوال قبل حساب کنیم.
        طبق اصل ضرب، پاسخ مسئله برابر است با حاصل ضرب تعداد انتخاب‌های ممکن برای فردی که بر جای خودش می‌نشیند
        و تعداد پریش‌های مجموعه باقیمانده :
          $$\binom{12}{1} \times 11! \sum\limits_{i=0}^{11} \frac{(-1)^i}{i!} = 12 \times 14684570 = 176214840$$
    }
\end{PROBLEM}
\EPROBLEM
\p
چند عدد طبیعی حداکثر ۹ رقمی وجود دارد که مجموع ارقام آن برابر با ۳۲ باشد؟

\EWSOLUTION{
  \p
    سوال را با اصل شمول و عدم شمول حل می‌کنیم: 
    
     \[|A_1\cup A_2\cup... \cup A_9|=\]
     \[\binom{9}{1}|A_1|+\binom{9}{2}|A_1\cap A_2|+...+\binom{9}{9}|A_1\cap A_2\cap...\cap A_9|\]
     
     حال مقدار عبارت‌ها را حساب می‌کنیم:
     
     \[|A_1|=\binom{30}{8}\]
     \[|A_1\cap A_2|=\binom{20}{8}\]
     \[|A_1\cap A_2\cap A_3|=\binom{10}{8}\]
     برای بقیه جمله‌ها جواب برابر ۰ است.
     حال از اصل متمم برای به دست آوردن جواب نهایی استفاده می‌کنیم:
     
        -کل حالات:
      \[\binom{40}{8}\]
        - حالات مطلوب:
      \[\binom{40}{8}- \binom{9}{1}\binom{30}{8}+\binom{9}{2}\binom{20}{8}-\binom{9}{3}\binom{10}{8}\]

    \begin{align*}
    \longrightarrow a_{12} = \binom{16}{12} 4 ^ {12}
    \end{align*}
}

\NOTE{تعریف متغیر‌های $A_i$ ضروری است، چون در غیر این صورت منظور از بقیه استدلال‌ها به هیج وجه مشخص نیست.}
\NOTE{اثبات و یا در صورت وضوح، اشاره به تقارن میان مجموعه‌ها برای استفاده از اصل شمول و عدم شمول به این شکل، ضروری است.}

\ESOLUTION[پاسخ اول]{
  \p
    رقم $i$ام این عدد را با $x_i$ نشان می‌دهیم.
    بنابراین به دنبال یافتن تعداد جواب‌های صحیح معادله زیر هستیم:
    \[\sum\limits_1^9 x_i=32\]
    \[\forall i \in [1,9] : x_i\leq 9\]
  تعداد جواب‌های صحیح این معادله را به کمک اصل متمم پیدا می‌کنیم:
  
     -کل حالات: تعداد جواب‌های صحیح نامنفی معادله  $\sum\limits_1^9 x_i=32 $ .این یک معادله سیاله است و تعداد جواب‌های صحیح آن برابر است با:
     
     \[\binom{40}{8}\]
     
    -حالات نامطلوب: تعداد جواب‌های صحیح نامنفی معادله $\sum\limits_1^9 x_i=32$  
     به طوری که:
     $$\exists i \in [1,9] : x_i\geq 10$$
     حال اگر مجموعه حالت‌هایی که در آن $x_i\geq 10 $ است را با $A_i$نشان دهیم، کافی است تعداد اعضای اجتماع این مجموعه‌ها را بیابیم.
     طبق اصل شمول و عدم شمول و با توجه به تقارن میان $A_i$‌ها داریم: 
     \[|A_1\cup A_2\cup... \cup A_9|=\]
     \[\binom{9}{1}|A_1|+\binom{9}{2}|A_1\cap A_2|+...+\binom{9}{9}|A_1\cap A_2\cap...\cap A_9|\]
   برای محاسبه مقدار عبارت‌ها، در معادله سیاله متناظر، در صورتی که $x_i\geq 10$ بود قرار می‌دهیم $x_i=y_i+10 $ و در غیر این صورت قرار میدهیم $x_i=y_i$، اگر تعداد $i$ هایی را که به ازای آن‌ها $x_i\geq 10 $ است را با $k$ نشان بدهیم، به دنبال تعداد جواب‌های صحیح نامنفی معادله سیاله $$\sum\limits_1^9 y_i=32-10k$$ هستیم، که برابر است با: 
   \[f(k) = \binom{40-10k}{8}\] 
    حال مقدار عبارت‌ها را حساب می‌کنیم:
    
    \[|A_1|=f(1)=\binom{30}{8}\]
    \[|A_1\cap A_2|=f(2)=\binom{20}{8}\]
    \[|A_1\cap A_2\cap A_3|=f(3)=\binom{10}{8}\]
    و چون:
    $$\forall m \in \mathbb{N}, m \geq 4: f(m) = {-(m-4) \choose 8} = 0$$
    پس بقیه‌ی جملات سمت راست اصل شمول برابر صفر هستند.
    پس کل حالات نامطلوب برابر است با:
    \[\binom{9}{1}\binom{30}{8}-\binom{9}{2}\binom{20}{8}+\binom{9}{3}\binom{10}{8}\]
     - حالات مطلوب طبق اصل متمم برابر است با:
     \[\binom{40}{8}- \binom{9}{1}\binom{30}{8}+\binom{9}{2}\binom{20}{8}-\binom{9}{3}\binom{10}{8}\]
}

\ESOLUTION[پاسخ دوم]{
  \p
با قرار دادن تعداد مناسبی صفر در سمت چپ اعدادی که کمتر از ۹ رقم دارند، می‌توان آن‌ها را به اعدادی با دقیقا ۹ رقم تبدیل کرد. اگر 
$\overline{x_1x_2...x_9}$
عدد مورد نظر باشد، داریم: 
$$x_1 + x_2 + ... + x_9 = 32$$
$$0 \leq x_i \leq 9$$
\p
اگر تعداد کل جواب‌های این معادله را بشماریم، در بعضی جواب‌ها به ازای مقادیر $i$
 خواهیم داشت
  $x_i \geq 10$
 که نامطلوب هستند و باید از کل حالات کم شوند. اما نکته‌ای که وجود دارد این است که تعداد چنین 
$x_i$
هایی  
می‌تواند حداکثر 3 باشد(در غیر این صورت مجموع ارقام بزرگتر یا مساوی ۴۰ می‌شود که خلاف فرض سوال است).
\p
فرض کنید حداقل
$k$
رقم داشته باشیم که مقدار بزرگتر یا مساوی ۱۰ گرفته‌اند.
همچنین فرض کنید
$r(k)$
برابر تعداد جواب‌های معادله فوق باشد؛ به شرطی که 
$k$
تا از 
$x_i$ها مقدار بزرگتر یا مساوی ۱۰ گرفته باشند. 
می‌دانیم تعداد جواب‌های چنین معادله‌ای برابر است با تعداد جواب‌های معادله زیر که 
$\binom{32 - 10k + 9 - 1}{9 - 1}$
می‌باشد:
$$y_1 + y_2 + ... + y_9 = 32 - 10k$$ 
$$y_i \geq 0$$
با توجه به اینکه
برای انتخاب این 
$k$
رقم،
$\binom{9}{k}$
حالت 
داریم:
$$r(k) = \binom{9}{k}\binom{32 - 10k + 9 - 1}{9 - 1}$$
\p
طبق اصل شمول و عدم شمول، پاسخ مسئله به صورت زیر محاسبه می‌شود:
$$r(0) - r(1) + r(2) - r(3)$$
$$= \binom{9}{0}\binom{40}{8} - \binom{9}{1}\binom{30}{8} + \binom{9}{2}\binom{20}{8} - \binom{9}{3}\binom{10}{8}$$
}
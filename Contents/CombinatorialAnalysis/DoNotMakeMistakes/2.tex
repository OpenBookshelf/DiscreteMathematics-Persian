\EPROBLEM
\p
اتحاد زیر را ثابت کنید.
  $$1^2 {n \choose 1} + 2^2 {n \choose 2} + 3^2 {n \choose 3} + ... + n^2 {n \choose n} = n(n+1)2^{n-2}$$
  
\EWSOLUTION{
    \p
    فرض کنید $ P = \displaystyle\sum_{k=0}^{n} {{k^2}\binom{n}{k}}$ بیانگر تعداد راه‌های انتخاب یک کمیته از بین $n$ کاندیدا است به طوری که یک فرد یا دو فرد متمایز، رئیس کمیته باشند. حال این شمارش را به روش دیگری انجام می‌دهیم.
    \p
    با فرض داشتن یک رئیس، رئیس را انتخاب کرده و تصمیم می‌گیریم که بقیه افراد حضور داشته باشند یا خیر. بار دیگر 2 رئیس انتخاب می‌کنیم و در مورد حضور یا عدم حضور بقیه افراد تصمیم می‌گیریم.
    حال مجموع این دو حالت را درنظر می‌گیریم:
        \[P = n\times{2^{n-1}} + n\times(n-1)\times{2^{n-2}} = n\times(n+1)\times{2^{n-2}}\]
    \p
    از تساوی این ۲ حالت، حکم مسئله اثبات می‌شود:
        \[\displaystyle\sum_{k=0}^{n} {{k^2}\binom{n}{k}} = n\times(n+1)\times{2^{n-2}}\]
}
 
\NOTE{توضیحات فارسی با شمارش انجام شده (عبارات ریاضی) تطابق ندارد. انتخاب دو رئیس از میان $n$ نفر $\binom{n}{2}$ حالت دارد نه $n\times(n-1)$!}
\NOTE{یک طرف دوگانه شماری که نیازمند اثبات است، بدیهی در نظر گرفته شده است.}
\NOTE{بهتر است روش اثبات (دوگانه شماری) ذکر شود.}

\ESOLUTION{
    \p
    سوال را با دوگانه شماری حل می‌کنیم:
    \p
    فرض کنید $ P$ بیانگر تعداد راه‌های انتخاب یک کمیته از بین $n$ کاندیدا است به طوری که یک فرد رئیس کمیته و یک نفر معاون باشند و رئیس و معاون می‌توانند یک نفر باشند. شمارش این راه‌ها به دو روش امکان پذیر است.
    \begin{enumerate}
        \item 
        با فرض یکسان بودن رئیس و معاون، رئیس را انتخاب کرده و تصمیم می‌گیریم که بقیه افراد حضور داشته باشند یا خیر. بار دیگر، رئیس و معاون متمایز را انتخاب کرده و در مورد حضور یا عدم حضور بقیه افراد تصمیم می‌گیریم.
        سپس تعداد حالات این دو روش را با هم جمع می‌کنیم:
        \[P = n\times{2^{n-1}} + n\times(n-1)\times{2^{n-2}} = n\times(n+1)\times{2^{n-2}}\]
        \item
        ابتدا انتخاب می‌کنیم چه کسانی عضو کمیته باشند. تعداد این افراد می‌تواند هر عددی باشد. سپس رئیس و معاون یکسان یا متمایز را از بین آن‌ها انتخاب می‌کنیم
        ($k$ تعداد اعضای کمیته است):
        \[P = \displaystyle\sum_{k=0}^{n} {\binom{n}{k}(k(k-1)+k)} = \displaystyle\sum_{k=0}^{n} {{k^2}\binom{n}{k}}\]
    \end{enumerate}
    \p
    از تساوی 2 حالت فوق، حکم مسئله اثبات می‌شود:
    \[\displaystyle\sum_{k=0}^{n} {{k^2}\binom{n}{k}} = n\times(n+1)\times{2^{n-2}}\]
}
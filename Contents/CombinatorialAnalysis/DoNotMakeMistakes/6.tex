\EPROBLEM
\p
نشان دهید تعداد روش‌های انتخاب ۴ عضو دو به دو  نامتوالی از مجموعه اعداد 
$1,2,3,...,n$
برابر است با 
$n-3 \choose 4$.

\EWSOLUTION{
  \p
    یک زیرمجموعه از این نوع مثلا $\{1,3,7,10\}$ را انتخاب و نابرابری‌های اکید 
    $$0 < 1 < 3 < 7 < 10 < n+1$$
    را در نظر می‌گیریم. بررسی می‌کنیم چند عدد صحیح بین هر دو عدد متوالی از این اعداد وجود دارند. در اینجا
    $0,1,3,2,10n-1$
    را به دست می‌آوریم:
    \begin{itemize}
      \item 
      $0$؛
      زیرا عددی صحیح بین
      $0, 1$
      وجود ندارد.
      \item 
      $1$؛
      زیرا تنها عدد $۲$ بین $1,3$ وجود دارد.
      \item 
      $3$؛
      زیرا اعداد صحیح $4,5,6$ بین $3,7$ وجود دارند.
      \item 
      $\dots$
  \end{itemize}
    مجموع این ۵ عدد صحیح برابر
    \lr{ $ 0 + 1 + 3 + 2 + n-10 = n-4 $} 
    است.
    پس تابع مولد زیر را داریم:
  
    $$G(x) = ( 1 + x^2 + x^3 +...)^2 (x + x^2 + x^3 +...)^3 $$
    $$= ( \sum_{k = 0}^{\infty} x^k)^2 ( \sum_{k = 0}^{\infty} x^{k+1})^3 $$
    $$= \frac{1}{(1-x)^2} \times (\frac{x}{1-x})^3 = \frac{x^3 }{(1-x)^5} $$
    $$= x^2 (1-x)^{-5} = x^3 \sum_{k=0}^{\infty} \binom{k+4}{k} x ^ k $$
    $$= \sum_{k=0}^{\infty} \binom{k+4}{k} x^{k+3} = \sum_{k=0}^{\infty}\binom{k+1}{k-3} x^k$$
    
    به دنبال ضریب
    $ x ^ {n-4 } $
    می‌گشتیم پس 
    $ k = n-4 $
    و جواب نهایی برابر است با:
    $$\binom{n-3}{n-7} = \binom{n-3}{4}$$
}

\NOTE{مثال زدن باید به صورتی باشد که حذف آن اختلالی در فهم جواب ایجاد نکند. در اینجا اگر مثال پاراگراف اول را حذف کنیم، مشخص نیست تابع مولد بر چه اساسی نوشته شده است. پس باید توضیحی درمورد تابع مولد و جمله‌ای که به دنبال ضریب آن هستیم بدهیم.}
\NOTE{نیاز است که کاملا گفته شود چه تغییر متغیری انجام می‌شود. در اینجا تغییر متغیر $ k \rightarrow {k+3} $ را داریم.}
\NOTE{همیشه به هنگام تغییر متغیر توجه کنیم ممکن است کران‌ها تغییر کنند. در اینجا کران پایین از صفر به سه می‌رود. 
  صورت اصلاح شده به این شکل است:
  $$ \sum_{k=3}^{\infty}\binom{k+1}{k-3} x^k $$
}

\ESOLUTION{
  \p
فرض کنید
4
عضو انتخاب شده را با 
$s_1 , s_2 , s_3 , s_4$
نشان دهیم.
تعداد عضو‌های انتخاب نشده کوچکتر از
$s_1$
را 
 $x_1$،
 تعداد عضوهای انتخاب نشده بین  
$s_1 , s_2$
را
 $x_2$،
 تعداد عضو‌های انتخاب نشده بین 
$s_2 , s_3$
را
$x_3$،
تعداد عضو‌های انتخاب نشده بین 
$s_3 , s_4$
را
$x_4$ و
تعداد عضوهای انتخاب نشده بزرگتر از 
$s_4$
را 
$x_5$
در نظر می‌گیریم. حال تعداد جواب‌های صحیح نامنفی معادله زیر را 
می‌شماریم:
$$x_1 + x_2 + x_3 + x_4 + x_5 = n - 4$$
$$x_1 , x_5 \geq 0  \: , \:  x_2 , x_3 , x_4 \geq 1$$

پاسخ آن برابر با ضریب
$x^{n - 4}$
 در عبارت زیر است:
$$(1 + x + x^2 + ...)\times(x + x^2 + x^3 + ...)\times(x + x^2 + x^3 + ...)$$
$$\times(x + x^2 + x^3 + ...)\times(1 + x + x^2 + ...) = \frac{x^3}{(1 - x)^5}$$
 بنابراین کافی است ضریب 
 $x^{n - 7}$
 را در بسط 
 $(1 - x)^{-5}$
 بشماریم.
 \p
 طبق جدول
 توابع مولد پرکاربرد
داریم:
$$(1 - x) ^ {-n} = \sum_{k = 0}^{\infty} \binom{n + k - 1}{k} x^{k} $$

بنابراین در این سوال داریم:
$$(1 - x)^{-5} = \sum_{k = 0}^{\infty} \binom{5 + k - 1}{k}x^k = \sum_{k = 0}^{\infty} \binom{k + 4}{k}x^k$$
$x^{n - 7}$ به ازای $k = n -7$ ساخته می‌شود. بنابراین جواب برابر است با:
$$\binom{n - 3}{4}$$
}

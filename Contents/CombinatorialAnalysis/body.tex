\CHAPTER[./combinations.jpg]{آنالیز ترکیبی}{
    آنچه در این فصل مورد بحث قرار خواهد گرفت،
    مبحث شمارش است که به محاسبه‌ی تعداد حالات رخداد یک پدیده،
    بدون بررسی تک تک حالات می‌پردازد.
    از کاربردهای این فصل می‌توان به
    محاسبه‌ی احتمالات پیش‌آمد‌ها،
    تخمین زمان اجرا و منابع مصرفی برنامه‌ها،
    برخی از تحلیل‌ها در گراف
    و ...
    اشاره کرد.
}{
    \href{https://www.freeimages.com/photographer/CDWaldi-48140}{Christian Kitazume}
}

% \subfile{./TheBasicsofCounting/body.tex}

% \subfile{./PermutationsandCombinations/body.tex}

% \subfile{./DistributionofObjects/body.tex}

% \subfile{./Identities/body.tex}

% \subfile{./Inclusion-Exclusion/body.tex}

% \subfile{./CountingwithGeneratingFunctions/body.tex}

% \subfile{./DoNotMakeMistakes/body.tex}

\subfile{./Problems/body.tex}
 
% \subfile{./Exercises/body.tex}

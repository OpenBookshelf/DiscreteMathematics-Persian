\EXERCISE
\begin{enumerate}
\item
نقطه‌ی
$(x, y, z)$
را در فضا نقطه‌ی شبکه‌ای می‌گویند هر گاه
$x$
و
$y$
و
$z$
اعدادی صحیح باشند.
$9$
نقطه‌ی شبکه‌ای در فضا مفروضند. آن‌ها را دو به دو به هم وصل می‌کنیم. ثابت کنید دست‌کم وسط یکی از پاره‌خط‌های رسم شده نقطه‌ای شبکه‌ای است.
\item
هشت عدد طبیعی متمایز که هیچ‌کدامشان از
$15$
بزرگ‌تر نیستند، مفروضند. ثابت کنید دست‌کم سه جفت از این اعداد یکی است. (لازم است این جفت‌ها به صورت مجموعه‌ی جدا از هم باشند.)
\item
فرض کنید
$n$
عددی طبیعی باشد. ثابت کنید عددی طبیعی وجود دارد که بسط اعشاری آن در مبنای
$10$
فقط از
$0$
و
$1$
تشکیل شده و در ضمن بر
$n$
بخش‌پذیر است.
\item
$33$
عدد طبیعی داده شده است که مقسوم‌علیه‌های اول آن‌ها فقط از اعداد
$7, 5, 3, 2, 11$
هستند. ثابت کنید حاصل‌ضرب دو تا از این اعداد مربع کامل است.
\end{enumerate}
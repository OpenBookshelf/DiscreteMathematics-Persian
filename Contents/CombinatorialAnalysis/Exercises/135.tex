\EXERCISE
مسئله‌ی برج امیرآباد: در این مسئله همانند مسئله‌ی برج هانوی سه ستون داریم که در ستون اول
$n$
تا دیسک از بالا به پایین، از کوچک به بزرگ قرار دارند، اما این‌جا ستون‌ها دور یک دایره چیده شده‌اند و حرکت دادن دیسک‌ها فقط به صورت ساعتگرد ممکن است (علاوه بر این شرط که هیچ دیسکی نباید روی دیسک کوچک‌تر قرار بگیرد) هدف انتقال همه دیسک‌ها از ستون اول به ستون دوم است.
\begin{enumerate}
\item
یک روش بازگشتی برای این کار این‌گونه است که ابتدا
$n - 1$
دیسک اول را به ستون سوم منتقل می‌کنیم، سپس بزرگ‌ترین دیسک را به ستون دوم ببریم و نهایتا
$n - 1$
دیسک اول را روی آن قرار دهیم. یک رابطه بازگشتی برای تعداد حرکات این الگوریتم بیابید و سپس با استفاده از توابع مولد، فرم بسته‌ی آن را به‌دست آورید.
\item
یک راه بهتر برای این کار استفاده از دو روش بازگشتی است:
$P_1(n)$
برای جلو بردن
$n$
دیسک به اندازه‌ی یک ستون و
$P_2(n)$
برای جلو بردن
$n$
دیسک به اندازه‌ی دو ستون:
\\
$P_1(n)$
: با یک‌بار استفاده از
$P_2(n - 1)$
،
$n - 1$
دیسک نخست را دو ستون جلو می‌بریم سپس دیسک آخر را یک ستون جابه‌جا می‌کنیم بعد دوباره با استفاده از
$P_2(n - 1)$
دیسک‌های باقی‌مانده را روی بزرگ‌ترین دیسک قرار می‌دهیم.
\\
$P_2(n)$
: با استفاده از
$P_2(n - 1)$
،
$n - 1$
دیسک اول را دو ستون جلو می‌بریم سپس دیسک آخر را یک ستون جابه‌جا می‌کنیم. حال با استفاده از
$P_1(n - 1)$
دیسک‌های دیگر را یک ستون جلو می‌بریم. سپس دیسک بزرگ را یک ستون دیگر جلو می‌بریم و نهایتا با استفاده‌ی دوباره از
$P_2(n - 1)$
بقیه دیسک‌ها را دو ستون جلو می‌بریم تا روی دیسک بزرگ قرار گیرند.
\\
$t_n$
را تعداد حرکات انجام شده توسط
$P_1(n)$
در نظر بگیرید. ثابت کنید:
$$t_n = 2t_{n-1} + 2t_{n-2} + 3$$
\item
فرم بسته‌ی رابطه‌ی بالا را با استفاده از توابع مولد به‌دست آورید.
\end{enumerate}
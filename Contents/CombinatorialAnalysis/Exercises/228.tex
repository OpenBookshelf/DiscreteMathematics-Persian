\EXERCISE
یک مستطیل را با تعدادی مستطیل کوچک‌تر پوشانده‌ایم به‌گونه‌ای که مستطیل‌ها فقظ می‌توانند در راس‌ها و اضلاع با یکدیگر مشترک هستند. همچنین اضلاع مستطیل‌های پوشاننده، موازی اضلاع مستطیل اصلی هستند و هیچ قسمتی از این مستطیل‌ها بیرون از مستطیل اصلی قرار نمی‌گیرند. ثابت کنید مجموع تعداد خطوط افقی، تعداد خطوط عمودی و تعداد چهارراه‌ها برابر با تعداد مستطیل‌های پوشاننده به علاوه‌ی سه است. مثلا در شکل زیر تعداد خطوط عمودی برابر
$5$
، تعداد خطوط افقی برابر
$5$
، تعداد چهارراهها برابر
$2$
و تعداد مستطیل‌های پوشاننده برابر
$9$
است. همچنین
$AB$
یک خط افقی،
$C$
یک چهارراه و
$DEFG$
یک مستطیل پوشاننده است. (راهنمایی: به رابطه‌ی تعداد سه‌راهی‌ها و خطوط عمودی و افقی توجه کنید.)
    \begin{center}
     	\includegraphics[scale=0.2]{./11.png}
    \end{center}
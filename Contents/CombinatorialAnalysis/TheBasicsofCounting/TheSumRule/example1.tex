\begin{PROBLEM}
    \p
    در هر یک از اشکال زیر، چند طریق سفر از نقطه پایین سمت چپ تصویر به نقطه بالا سمت راست وجود دارد، اگر در هر نقطه
    فقط مجاز به حرکت به سمت راست یا بالا باشیم؟

    \begin{enumerate}
        \item 
        \p
        \centerimage{0.35}{./1.jpg}

        \SOLUTION{
            \p
            سفر را از نقطه پایین سمت چپ تصویر آغاز می‌کنیم، پس تنها یک حالت برای رسیدن به این نقطه وجود دارد.
            برای دیگر نقاط، با توجه به اینکه تنها حرکت‌های رو به بالا و راست مجاز هستند، طبق اصل جمع، تعداد راه‌های رسیدن
            به هر نقطه برابر است با حاصل جمع تعداد طریقه‌های رسیدن به نقطه سمت چپ و پایین آن، به شرطی که مسیری به آن نقطه داشته باشد.
            با اعمال روش بالا بر روی تک تک نقاط (به ترتیب از پایین چپ تا بالا راست تصویر)، مطابق شکل زیر می‌توانیم تعداد مسیر‌های رسیدن
            به هر نقطه را محاسبه کنیم. بدین‌ترتیب، پاسخ سوال برابر با $84$ است.
            \p
            \centerimage{0.6}{./1-0.jpg}
        }

        \item 
        \p
        \centerimage{0.35}{./2.jpg}

        \SOLUTION{
            \p
            مشابه قسمت قبل عمل می‌کنیم با این تفاوت که در برخی قسمت‌ها که بیش از یک مسیر بین دو نقطه وجود دارد،
            از اصل ضرب استفاده می‌کنیم. برای این کار، تعداد مسیر‌های منتهی به نقطه‌ی مبدا را در تعداد راه‌های رسیدن آن به نقطه‌ی مقصد ضرب
            کرده و سپس پاسخ را در اصل جمع لحاظ می‌کنیم. بدین‌ترتیب، پاسخ مسئله برابر 297 خواهد بود.
            \p
            \centerimage{0.6}{./2-0.jpg}
        }
    \end{enumerate}
\end{PROBLEM}
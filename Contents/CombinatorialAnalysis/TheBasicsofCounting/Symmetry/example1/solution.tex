\p
اگر از تساوی برخی از حالات با چرخش ۱۸۰ درجه صرف‌نظر کنیم و مهره‌های قلعه را متمایز در نظر بگیریم،
طبق اصل ضرب، تعداد چینش‌های ممکن که آن را با $T_1$ نشان می‌دهیم، برابر است با :
  $$T_1 = 64 \times 49 = 3136$$
که ۶۴ برای انتخاب یک خانه برای رخ اول و ۴۹ برای انتخاب یک خانه خارج از سطر و ستون
مربوط به رخ اول، برای رخ دوم است. برای از بین بردن تمایز بین رخ‌ها، می‌توان طبق اصل تقسیم نوشت :
  $$T_2 = \frac{T_1}{2} = \frac{3136}{2} = 1568$$
  که در آن، $T_2$ تعداد چینش‌های ممکن با درنظر گرفتن رخ‌های نامتمایز است.
  \p
می‌دانیم هر چیدمانی که در آن جایگاه دو رخ نسبت به مرکز صفحه متقارن نباشد، توسط چرخش ۱۸۰ درجه به
یک چیدمان جدید در پاسخ ما تبدیل شده‌اند.
برای از بین بردن این تمایز، طبق اصل تقسیم، تعداد این چیدمان‌ها را بر دو تقسیم می‌کنیم.
توجه کنید که چیدمان‌هایی که در آن‌ها جایگاه دو رخ نسبت به مرکز متقارن است،
طی چرخش ۱۸۰ درجه به خودشان تبدیل می‌شوند. پس از ابتدا یک بار شمرده شده‌اند. تعداد این چیدمان‌ها
که آن را با $T_3$ نشان می‌دهیم، برابر است با : 
  $$T_3 = \frac{64 \times 1}{2} = 32$$
که در آن ۶۴ برای جایگاه رخ اول و ۱ برای جایگاه رخ دوم است که ملزم است در تقارن رخ اول باشد.
تقسیم بر دو نیز برای از بین بردن تمایز بین رخ‌ها طبق اصل تقسیم است.
تعداد چیدمان‌هایی که در آن‌ها جایگاه دو رخ متقارن نیست را با
$T_4$
نشان می‌دهیم و طبق اصل متمم، این مقدار برابر است با :
  $$T_4 = T_2 - T_3 = 1568 - 32 = 1536$$
  \p
حال تعداد چیدمان‌های نامتقارن متمایز در برابر چرخش ۱۸۰ درجه را با
$T_5$
نشان می‌دهیم و طبق اصل تقسیم، داریم:
  $$T_5 = \frac{T_4}{2} = \frac{1536}{2} = 768$$
  \p
طبق اصل جمع، تعداد چیدمان‌های متمایز در برابر چرخش ۱۸۰ درجه، اعم از متقارن و نامتقارن، که آن را با $T_6$ نشان خواهیم داد،
 برابر است با :
  $$T_6 = T_5 + T_3 = 768 + 32 = 800$$
  که همان خواسته‌ی مسئله است.
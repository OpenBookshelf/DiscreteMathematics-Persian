\SUBSECTION{جایگشت خطی با اعضای تکراری}
\TARGET{جایگشت خطی با اعضای تکراری}

\begin{PROBLEM}[جایگشت خطی با اعضای تکراری]
    \p
    فرض کنید
    $A$
    یک جعبه حاوی
    $c_i$
    مهره از نوع
    $t_i$
    باشد و 
    $$i \leq n , i \in \mathbb{N}$$
    تعداد جایگشت‌های خطی این اعضا را محاسبه کنید.

    \SOLUTION[پاسخ اول: استفاده از اصل تقسیم]{
        \p
        ابتدا تمام اعضا را متمایز متصور می‌شویم. تعداد کل مهره‌های درون
        A
        را $m$ می‌نامیم که:
        $$m = \sum\limits_{i=1}^n c_i \times t_i$$
        \p
        می‌دانیم تعداد جایگشت‌های خطی این اعضا (درصورتی که تمام مهره‌ها را از هم متمایز متصور شویم)
        برابر است با $m!$.
        حال می‌دانیم که هر
        $c_i!$
        جایگشت در این جایگشت‌ها، حاصل جایگشت‌های مختلف مهره‌های 
        $t_i$
        در مکان‌های یکسان است که اگر بخواهیم این مهره‌ها را نامتمایز بدانیم،
        یکسان خواهند بود. بنابراین اگر مهره‌های از نوع
        $t_i$
        را نامتمایز فرض کنیم، طبق اصل تقسیم، تعداد جایگشت‌ها برابر است با:
        $$\frac{m!}{c_i!}$$
        \p
        طبق استدلال مشابه، اگر تمام مهره‌های هم نوع را نامتمایز متصور شویم،
        طبق اصل تقسیم، تعداد جایگشت‌ها برابر است با:
        $$\frac{m!}{\prod\limits_{i=1}^n c_i!}$$
    }

    \SOLUTION[پاسخ دوم: استفاده از ترکیب چندگانه]{
        \p
        نام‌گذاری زیر را درنظر بگیرید:
        $$m = \sum\limits_{i=1}^n c_i \times t_i$$
        \p
        می‌خواهیم تعداد حالات به صف کردن
        $m$
        مهره بعضاً نامتمایز را پیدا کنیم.
        نگاه به مسئله را کمی تغییر می‌دهیم.
        صف نهایی را به شکل تعداد 
        $m$
        جایگاه شماره‌گذاری شده (به صف) متصور شوید.
        برای چینش این $m$ مهره در این جایگاه‌ها،
        نیاز داریم در ابتدا مشخص کنیم کدام جایگاه‌ها،
        مربوط به کدام نوع مهره خواهند بود.
        برای اینکار، نیاز داریم مجموعه جایگاه‌ها را
        به زیرمجموعه‌های
        $B_i; i\in\mathbb{N}, i \leq n$
        افراز کنیم به نحوی که مجموعه
        $B_i$
        دارای
        $c_i$
        عضو باشد و این اعضا، برای مهره‌های از نوع
        $t_i$
        گزینش شده باشند.
        پس از این افراز، با توجه به نامتمایز بودن مهره‌های از یک نوع،
        با یک حالت، مهره‌های هر نوع را در جایگاه‌های همان نوع قرار می‌دهیم.
        طبق ترکیب چندگانه می‌دانیم تعداد راه‌های افراز یک مجموعه
        $m$
        عضوی به زیرمجموعه‌های
        $c_i;  i\in\mathbb{N}, i \leq n$
        عضوی، برابر است با:
        $${m \choose c_1,c_2,...,c_n} = \frac{m!}{\prod\limits_{i=1}^n c_i!}$$
    }
\end{PROBLEM}

\TARGET{ارتباط میان جایگشت خطی با اعضای تکراری و ترکیب چندگانه}
\NOTE{در پاسخ دوم مسئله بالا مشاهده کردیم که مسئله جایگشت خطی با اعضای تکراری، درواقع بیانی متفاوت از ترکیب چندگانه است.}

\begin{THEOREM}
    \TARGET{تعداد جایگشت‌های خطی متمایز با اعضای تکراری}
    \p
    تعداد جایگشت‌های خطی متمایز، برای گردایه‌ی
    $A$
    که به ازای
    $$i \leq n , i \in \mathbb{N}$$،
    دارای
    $c_i$
    عضو یکسان از نوع
    $t_i$
    باشد، برابر است با: 
    $$m = \sum\limits_{i=1}^n c_i \times t_i$$
    $${m \choose c_1,c_2,...,c_n} = \frac{m!}{\prod\limits_{i=1}^n c_i!}$$
\end{THEOREM}

\begin{PROBLEM}
    \p
    با حروف
    $t,t,a,c,a,t,s$
    چند رشته حرفی متمایز به طول ۷ می‌توان ساخت؟

    \SOLUTION{
        \p
        طبق تعداد جایگشت‌های خطی با اعضای تکراری:
        $$\frac{7!}{3!2!1!1!} = 420$$
    }
\end{PROBLEM}
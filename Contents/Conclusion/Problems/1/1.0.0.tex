\p
    از قضیه اساسی حساب می‌توان نتیجه گرفت که هر عدد طبیعی را می‌توان به صورت
    $2^{\alpha}{\beta}$
    نوشت.
    که
    $\beta$ 
     عددی فرد و
     $\alpha$ 
     عددی صحیح و نامنفی است.
    با توجه به اینکه میان 
    $1$
    تا
    $2n$
    تعداد 
    $n$
    عدد فرد وجود دارد طبق اصل لانه کبوتری در میان اعداد
    $a_1$
    ,
    $a_2$
    ,...
    و
    $a_{n+1}$
    دو عدد مانند
    $a_i$
    و
    $a_j$
    ($i\neq j$)
    وجود دارند که قسمت فرد آن‌ها برابر است
    ,
    یعنی:
    $$a_i=2^{\alpha_1}\beta,a_j=2^{\alpha_2}\beta$$
    به این ترتیب اگر
    $\alpha_1 \leq \alpha_2$
    باشد
    ,
    آن‌گاه
    $a_i|a_j$
    و در غیر این صورت
    $a_j|a_i$
    .
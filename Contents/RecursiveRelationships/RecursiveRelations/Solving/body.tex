\SUBSECTION{حل روابط بازگشتی}

در این قسمت با برخی روش های اولیه حل روابط بازگشتی آشنا می‌شویم.

یکی از روش‌های حل روابط بازگشتی روش حدس و استقرا است, در این روش ابتدا با کار هایی مانند جایگذاری در رابطه, جواب رابطه را حدس می‌زنیم, سپس به کمک استقرا درستی حدس خود را اثبات می‌کنیم.

\begin{PROBLEM}
    جواب رابطه بازگشتی 
    $$f(n)=f(n-1)+2n-1$$
    $$f(1)=1$$
    را بیابید
    \SOLUTION{
        ابتدا با جایگذاری در رابطه چند جمله اول آن را به دست می‌آوریم:
        $$f(1)=1$$
        $$f(2)=f(1)+3=4$$
        $$f(3)=f(2)+5=9$$
        $$f(4)=f(3)+7=16$$
        با توجه به اعداد به دست آمده, حدس می‌زنیم که جواب برابر 
        $f(n)=n^2$
        باشد.
        می‌خواهیم به کمک استقرا حدس خود را ثابت کنیم.
        \begin{itemize}
            \item پایه استقرا:
            $$f(1)=1^2=1$$
            \item فرض استقرا:
            $$f(n)=n^2$$
            \item حکم استقرا:
            $$f(n+1)=(n+1)^2$$
        \end{itemize}
        اثبات حکم استقرا:
        $$f(n+1)=f(n)+2(n+1)-1=n^2+2n+1=(n+1)^2$$
        بنابراین ثابت می‌شود که حکم استقرا صحیح است و جواب رابطه بازگشتی برابر
        $f(n)=n^2$
        است.
    }

 \end{PROBLEM}
 
 ما از قبل با رابطه مجموع جملات برخی از دنباله‌ها مانند دنباله اعداد ۱ تا 
$n$
 یا دنباله اعداد حسابی و هندسی
 آشنایی داریم, یکی از روش های حل معادله بازگشتی جایگذاری رابطه 
 $f(n-1), f(n-2), ...f(1)$
 با توجه به رابطه بازگشتی اولیه در رابطه و سپس محاسبه عبارت حاصل است.

 برای فهم بهتر, همان مثال بالا را به این روش حل می‌کنیم.
 \begin{PROBLEM}
    جواب رابطه بازگشتی 
    $$f(n)=f(n-1)+2n-1$$
    $$f(1)=1$$
    را بیابید
    \SOLUTION{
        $$f(n)=f(n-1)+2n-1=f(n-2)+2(n-1)-1+2n-1=\sum_{i=1}^{n} 2i-1=2\sum_{i=1}^{n}+\sum_{i=1}^{n}1=2\times\dfrac{n(n+1)}{2}+n=n^2$$
    }

 \end{PROBLEM}
 
\begin{PROBLEM}
    \p
    تعداد حالت‌های پرکردن یک جدول 
    $2\times n$
    با کاشی‌های
    $2\times 1$
    را به صورت یک رابطه بازگشتی 
    $f(n)$
    بیابید.
    چرخاندن کاشی‌ها مجاز است.
    \begin{center}
        \includegraphics[totalheight=4cm]{im1.png}
        \end{center}
    \SOLUTION{
        \p
        دو حالت برای قرار دادن کاشی در خانه اول از سمت چپ وجود دارد:
        \begin{itemize}
            \item 
        اگر کاشی اول را به صورت عمودی قرار بدهیم، بقیه مسئله تبدیل به کاشی کاری یک جدول
            $2\times n-1$
        می‌شود.
        \begin{center}
            \includegraphics[totalheight=4cm]{im2.png}
            \end{center}
        \item 
        اگر کاشی اول را به صورت افقی قرار بدهیم، باید پایین آن یک کاشی افقی دیگر قرار بدهیم و بقیه مسئله تبدیل به کاشی‌کاری یک جدول
        $2\times n-2$
        می‌شود.
        \begin{center}
            \includegraphics[totalheight=4cm]{im3.png}
            \end{center}
        \end{itemize}
        بنابراین طبق اصل جمع کل حالت‌ها برابر
        $f(n)=f(n-1)+f(n-2)$
        است.
        \p
        همانطور که در بخش دنباله‌های بازگشتی اشاره شد، لازم است که دو شرط اولیه برای این رابطه بازگشتی پیدا کنیم. بنابرین لازم است حالت
        $n=1$
        و
        $n=2$
        را به صورت دستی بررسی کنیم:
        $$f(1)=1$$
        \begin{center}
            \includegraphics[totalheight=3cm]{im5.png}
            \end{center}
        $$f(2)=2$$
        \begin{center}
            \includegraphics[totalheight=3cm]{im4.png}
            \end{center}
   
    }

 \end{PROBLEM}


\begin{DEFINITION}
    \p
    \FOCUSEDON{رابطه‌ بازگشتی خطی از درجه $k$}
    رابطه‌ای به فرم
   \[a_n=c_{1}a_{n-1}+c_{2}a_{n-2}+...+c_{k}a_{n-k}+F(n)\]
   است که
   $\forall{i}:c_i \in \mathbb{R} $
   و 
   $ c_k\neq 0$
.
\end{DEFINITION}

 \begin{DEFINITION}
    \p
    \FOCUSEDON{رابطه‌ بازگشتی خطی همگن از درجه $k$}
    رابطه‌ای به فرم
   \[a_n=c_{1}a_{n-1}+c_{2}a_{n-2}+...+c_{k}a_{n-k}\]
   است که
   $\forall{i}:c_i \in \mathbb{R} $
   و 
   $ c_k\neq 0$.


\end{DEFINITION}

\p
برای نمونه رابطه‌ 
$f_n=f_{n-1}+f_{n-2}$
یک رابطه‌ خطی همگن از درجه ۲ و رابطه‌ 
$a_n=a_{n-1}+3a_{n-2}+3n+2^n$
یک رابطه‌ خطی ناهمگن از درجه ۲ است.

\TARGET[]{اثبات یکتایی پاسخ رابطه بازگشتی}
\begin{PROBLEM}[اثبات یکتایی پاسخ رابطه بازگشتی]
    \p
    با در نظر گرفتن رابطه بازگشتی به فرم
    \[f_{n}=h(f_{n-1},f_{n-2},...,f_{n-k})\]
    به سوالات زیر پاسخ دهید.
    \begin{enumerate}
        \item  نشان دهید اگر این رابطه دارای 
        $k$
        شرط اولیه
        $$f_0=C_0, f_1=C_1, ..., f_{k-1}=C_{k-1}$$
        باشد
    ،پاسخ صریح این رابطه به صورت یکتا مشخص می‌شود.

        \SOLUTION{
            \p
            برای اثبات این موضوع کافی است نشان دهیم که تمام جملات این دنباله به صورت یکتا مشخص می‌شوند:
            \p
            این حکم را با استفاده از استقرا ثابت می‌کنیم.
            \begin{itemize}
                \item فرض استقرا:
                جمله k ام به صورت یکتا مشخص می‌شود.
                با توجه به این که جمله $k$ تابعی از $k$ جمله قبلی است که همه آن‌ها مشخص هستند، این جمله به صورت یکتا تعیین می‌شود.
                \item فرض استقرا:
                جملات $k$ تا $n-1$ به صورت یکتا مشخص می‌شوند.
                \item حکم استقرا:
                جمله $n$ ام به صورت یکتا مشخص می‌شود.
                \p
                اثبات:
                چون جمله $n$ ام تابعی از $k$ جمله قبلی خود است که همه آن‌ها طبق فرض استقرا مشخص هستند، بنابراین این جمله هم به صورت یکتا مشخص می‌شود.
            \end{itemize}
            \p
        }

        \item 
        نشان دهید اگر این رابطه دارای 
        $k$
        شرط اولیه
        مستقل از هم باشد
        باشد،
        پاسخ صریح این رابطه به صورت یکتا مشخص می‌شود.

        \SOLUTION{
            \p
            کافی است نشان دهیم که با داشتن k جمله مستقل می‌توان k جمله اول را به دست آورد.
            رابطه بازگشتی را برای آن $k$ جمله مشخص می‌نویسیم، سپس به طور متوالی همه جملات ایجاد شده را با رابطه بازگشتی جایگزین می‌کنیم تا زمانی که به $k$ جمله اول برسیم.
            این کار به ما $k$ معادله بر حسب $k$ مجهول می‌دهد که می‌توان از حل آن $k$ جمله اولیه این رابطه را پیدا کرد.
            با داشتن $k$ جمله‌ی اول، طبق قسمت اول همین سوال، پاسخ معادله به صورت یکتا مشخص می‌شود.
        }
    \end{enumerate}
\end{PROBLEM}


\begin{THEOREM}
    \p
    اگر رابطه بازگشتی به فرم
    \[f_{n}=h(f_{n-1},f_{n-2},...,f_{n-k})\]
    داشته باشیم که دارای 
    $k$
    شرط اولیه مستقل از هم باشد،
    در این صورت پاسخ صریح این رابطه به صورت یکتا مشخص می‌شود.
\end{THEOREM}




\p
	تابع $f$ را به این صورت تعریف می‌کنیم:
	
	تابع $f$ برای جایگشت $G$ نشان دهنده تعداد اعدادی است که در جایگشت $G$ در جایگاه خود قرار دارند.
	
فرض کنید $W$ مجموعه تمام جایگشت‌های مجموعه 1 تا $n$ باشد و $A=\sum_{G \in W} f(G)$. به دو صورت می‌توان مقدار $A$ را بدست آورد:
	\begin{enumerate}
		\item 
		برای هر عدد حساب کنیم که در چند جایگشت در جایگاه خود قرار دارند.
		هر عدد در $(n-1)!$ جایگشت در جایگاه خود قرار دارد. زیرا عدد را در جایگاه خود قرار می‌دهیم و بقیه اعداد را به $(n-1)!$ طریق می‌چینیم. $n$ عدد وجود دارد. بنابراین مقدار $A$ برابر $n!$ است.
		
		\item
		از طرف دیگر طبق تعریف مقدار تابع $f$ برای هر جایگشت برابر تعداد اعدادی که در جایگاه خود قرار دارند خواهد بود. برای بدست آوردن مقدار $A$ می‌تون به صورت زیر حالت بندی کرد:
		
		\begin{itemize}
			\item 
			جایگشت‌هایی که $0$ عدد در جایگاه خود قرار دارند. تعداد این جایگشت‌ها برابر $D_n(0)$ است که مقدار $f$ این جایگشت‌ها برابر $0\times D_n(0)$ است.
			\item
			جایگشت‌هایی که $1$ عدد در جایگاه خود قرار دارند. تعداد این جایگشت‌ها برابر $D_n(1)$ است که مقدار $f$ این جایگشت‌ها برابر $1\times D_n(1)$ است.
			$$\dots$$
			\item
			جایگشت‌هایی که $n$ عدد در جایگاه خود قرار دارند. تعداد این جایگشت‌ها برابر $D_n(n)$ است که مقدار $f$ این جایگشت‌ها برابر $n\times D_n(n)$ است.
		\end{itemize}
	بنابراین مجموع $f$ ها برابر حاصل جمع اعداد بالا خواهد بود که برابر است با $$\sum_{k=0}^{n} k \times D_n(k)$$.
	\end{enumerate} 
	
	
	بنابراین خواهیم داشت:
	$$\sum_{k=0}^{n} kD_n(k) = n!$$

\p
در مسئله برج هانوی فرض کنید که سه میله داریم و هدف این است که n دیسک را با کمترین تعداد حرکت از میله ۱ به میله ۳ منتقل کنیم، با این تفاوت که نمی‌توانیم دیسکی را مستقیم از میله ۱ به ۳ و یا بالعکس انتقال دهیم و هر انتقالی بین میله‌های ۱ و ۳ می‌بایست حتما شامل گذر از میله ۲ نیز باشد. همچنین مشابه مسئله هانوی اصلی، نمی‌توانیم دیسکی را بر روی دیسک کوچکتر خودش قرار دهیم.

رابطه بازگشتی 
$H_n$ برای حداقل تعداد حرکات لازم با n دیسک را به‌دست آورید.

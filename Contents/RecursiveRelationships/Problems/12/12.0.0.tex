\p
$H_n$ نشان‌دهنده کمترین تعداد حرکات مورد نیاز برای حل مسئله برج هانوی است به نحوی که هر حرکت دیسک بین میله‌های ۱ و ۳ شامل میله ۲ نیز باشد.

حالت پایه: اگر ۰ دیسک داشته باشیم، می‌دانیم که نیاز به هیچ حرکتی نداریم و بنابراین:
\bigbreak
$$ H_{0} = 0 $$
\bigbreak
در گام بعدی می‌خواهیم مسئله را برای n دیسک حل کنیم:
\bigbreak
۱. در ابتدا بزرگ‌ترین دیسک را ثابت نگه داشته و $n-1$ دیسک کوچکتر را با حداقل حرکت ممکن، به میله شماره ۳ انتقال می‌دهیم. (هزینه این گام = $H_{n-1}$

۲. در گام بعدی بزرگ‌ترین دیسک را به میله شماره ۲ منتقل می‌کنیم. (هزینه این گام = ۱)

۳. حال ($n-1$ دیسک موجود در میله ۳ را با حداقل حرکت ممکن، به میله ۱ انتقال می‌دهیم. در واقع انگار میله مبدا را ۳ و میله مقصد را ۱ در نظر گرفته‌ایم و بنابراین هزینه این مرحله نیز مانند مرحله ۱ خواهد بود. (هزینه این گام = $H_{n-1}$

۴. حال بزرگ‌ترین دیسک را از میله ۲ به میله ۳ انتقال می‌دهیم. (هزینه این گام = ۱)

۵. و در نهایت نیز ($n-1$ دیسک موجود در میله ۱ را با حداقل حرکت ممکن، به میله ۳ و بر روی بزرگ‌ترین دیسک انتقال می‌دهیم. (هزینه  این گام = $H_{n-1}$
\bigbreak
به این ترتیب، با حداقل جابجایی (کمترین تعداد جابجایی $n-1$ دیسک ممکن برای حل مسئله) توانستیم n دیسک را به میله ۳ انتقال دهیم. بنابراین هزینه کل برابر خواهد بود با جمع هزینه‌های مراحل ۱ تا ۵ و داریم:
\bigbreak
$$H_{n} = 3H_{n-1} + 2$$
\bigbreak

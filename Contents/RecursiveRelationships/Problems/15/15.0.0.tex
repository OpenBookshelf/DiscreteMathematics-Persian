\p
    با توجه به قید کمتر مساوی بودن مقدار عنصر $i$ ام از مقدار عناصر
    $2i$
    و
    $2i+1$
    نتیجه می‌گیریم که عنصر اول از تمامی عناصر دیگر مقدار کمتری دارد. اثبات آن نیز با استقرای قوی صورت می‌گیرد. 
     

    پایه استقرا: عنصر اول از عنصر دوم و سوم کمتر است. 
    
    فرض استقرا: مقدار عنصر اول از عنصر2 تا $k$ ام کمتر باشد. 
    
    حکم استقرا: عنصر اول از عنصر $k + 1$  ام کمتر است. بنابراین برای گام استقرا تلاش می‌کنیم تا از روی برقرار بودن فرض برای $k$ عنصر اول، گزاره را برای عنصر $k+1$ ام هم اثبات کنیم. 
    
    اثبات: می‌دانیم که اگر $k+1$ فرد باشد از عدد $\frac{k}{2}$ بزرگتر است و طبق تعدی از عنصر 1 بزرگتر است. همینطور اگر $k+1$ زوج باشد نیز از $\frac{k+1}{2}$ بیشتر است. بنابراین اثبات می‌شود که عنصر $k+1$ ام از عنصر $1$ بیشتر است و حکم استقرا ثابت است. 
    
    بنابراین اگر مانند شکل $1$ یک درخت دودویی در نظر بگیریم که در آن هر راس از فرزندانش کوچکتر مساوی باشد، می‌توانیم این راس‌ها را به ترتیب از بالا به پایین و چپ به راست از $1$ تا $n$ برچسب بزنیم. در این صورت فرزندان راس $n$ ام می‌شوند راس‌های ${2n}$ و ${2n + 1}$ و هر درخت به این فرم یک جواب مساله است. همینطور این درخت با تعداد راس‌های
     $2^{n}-1$ دقیقا یک درخت پر با ارتفاع ${n}$ را تشکیل می‌دهد. حال که در درخت اصلی، کوچکترین عضو را پیدا کردیم آن را در ریشه می‌گذاریم و بعد راس‌ها را به دلخواه به دو زیر درخت کامل با ارتفاع ${n-1}$ تقسیم می‌کنیم که تعداد حالات آن   
     $ { 2^n-2 \choose 2^{n-1}-1 }$ است. سپس برای هرکدام از زیردرخت‌ها دوباره یک مساله ترتیبی مثل درخت اصلی داریم که باید تعداد حالات ساختن زیردرخت‌ها را در آن با ارتفاع $n-1$ پیدا کنیم. بنابراین رابطه‌ی بازگشتی برای دو زیردرخت به این فرم در می‌آید:
    $$ T_n = {2^n-2 \choose 2^{n-1}-1 } T_{n-1}^2 $$
که هر درخت را به زیر درخت‌هایش مرتبط می‌کند. مقادیر نشان داده شده در نودهای درخت زیر، میزان سهم مصرفی را نشان می‌دهند از ماده‌ای که در آن جایگاه از درخت قرار می‌گیرد        .
\p  
\centerimage{0.3}{./exmp3.jpg}{شکل ۱: درخت نامساوی}

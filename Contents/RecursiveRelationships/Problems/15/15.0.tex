\p 
    می‌خواهیم یک معجون بسازیم،
    این معجون از 
    $n$
    ماده تشکیل می‌شود که
    $n$
    به صورت
    $2^{k}-1$
    است و ترتیب ریختن مواد ثابت است یعنی همیشه ماده 
    $i$م
    در مرحله 
    $i$م
    ریخته می‌شود
    و از هر ماده می‌توان حداکثر 
    $n$
    واحد استفاده کرد.
    در ساخت این معجون در هر مرحله می‌توان فقط یک ماده را اضافه کرد و اگر ماده‌ای در مرحله
    $i‌$م
    اضافه شده باشد، مقدار ماده‌ای که در مرحله 
    $2i$
    و
    $2i+1$
    اضافه می‌شود حتما باید از مقدار این ماده کمتر نباشد(مقادیر به صورت صحیح و ناصفرند).
    رابطه‌ای بازگشتی برای تعداد معجون‌های مختلفی که می توان تولید کرد بیابید.


    برای مثال می‌توان یک بار از هر ماده به اندازه 1 واحد بریزیم و حالت دیگر از ماده
    $i$ به اندازه $i$ بریزیم. لازم به ذکر است که این که نسبت مواد ثابت باشد دلیلی بر یکسان بودن ترکیب ندارد و رابطه بازگشتی بر حسب $k$ باشد.(مقدار استفاده شده از هیچ دو ماده‌ای برابر نیست و از هر ماده حداقل $1$ واحد استفاده می‌شود.)
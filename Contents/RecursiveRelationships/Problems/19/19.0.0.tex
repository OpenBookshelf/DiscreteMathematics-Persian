\p
    این مساله را با حالت‌بندی روی غار اول حل می‌کنیم. در غار اول می‌توان حداکثر \( \lfloor \frac{n}{k} \rfloor \) جنس قرار داد زیرا اگر از این مقدار بیشتر شود مجبوریم در یک غار دیگر کمتر از این مقدار قرار دهیم که ممکن نیست. بنابراین رابطه‌ی نهایی برای \( A_{n,k} \) که تعداد حالات تقسیم n کالا در k غار است یک سیگما روی این حالات غار اول است که می‌شود:
\[ A_{n,k} = \sum\limits_{i=1}^{\lfloor \frac{n}{k} \rfloor}  A_{n-ik,k-1} \]در جواب بالا، تفریق مقدار \lr{ik} در اندیس به این خاطر است که هر تعداد وسیله که در اژدها در غار اول قرار دهد، باید در غارهای دیگر نیز حداقل به آن تعداد وسیله قرار دهد و بنابراین برای i غار، تعداد ik وسیله ابتدا باید در هر غار به صورت ثابت قرار بگیرد و امکان حالت‌بندی ندارد. \\ بدلیل صریحا مطرح نشدن قید وجود حداقل یک شی در هر غار، برای اندیس شروع هر دو مقدار ۱ و ۰ در تصحیح نمره می‌گیرند.

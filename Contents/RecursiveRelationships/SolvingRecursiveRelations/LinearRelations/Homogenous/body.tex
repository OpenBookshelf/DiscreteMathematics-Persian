\SUBSECTION{حل روابط بازگشتی خطی همگن}

\begin{DEFINITION}
    \p
    اگر رابطه‌ بازگشتی خطی همگنی به فرم
    \[a_n=c_{1}a_{n-1}+c_{2}a_{n-2}+...+c_{k}a_{n-k}\]
    داشته باشیم، به معادله
    \[r^k-c_{1}r^{k-1}-c_{2}r^{k-2}-...-c_{k}=0\]
    \FOCUSEDON{معادله مشخصه}
    %\footnote{characteristic-equation}
     این رابطه‌ بازگشتی می‌گوییم.
     جواب‌های این معادله \FOCUSEDON{ریشه‌های مشخصه}
    %\footnote{(characteristic-roots)}
    نامیده می‌شوند.

\end{DEFINITION}
\begin{PROBLEM}[جواب رابطه بازگشتی خطی همگن درجه ۲ با ریشه متمایز]
    \p
    نشان دهید
    $a_n$
    جواب رابطه‌ بازگشتی خطی همگن درجه ۲ 
    $$a_n=c_1 a_{n-1}+c_2 a_{n-2}$$
    است، اگر و تنها اگر
    $a_n$
    به فرم
    $a_n=\alpha_1 r_1^n+\alpha_2 r_2^n$ 
    باشد
    که
    $r_1$
    و
    $r_2$
    ریشه‌های متمایز معادله مشخصه و 
    $\alpha_1$
    و
    $\alpha_2$
    اعداد حقیقی
    هستند.
    \SOLUTION{
        \p

        ابتدا نشان می‌دهیم که اگر 
        $a_n=\alpha_1 r_1^n+\alpha_2 r_2^n$ 
        آنگاه
        $a_n$
    جواب رابطه‌ بازگشتی خطی همگن است.
    از جایگذاری این رابطه در معادله داریم:
    $$c_1 a_{n-1}+c_2 a_{n-2}=c_1 (\alpha_1 r_1^{n-1}+\alpha_2 r_2^{n-1})+c_2(\alpha_1 r_1^{n-2}+\alpha_2 r_2^{n-2})$$ 
    $$=\alpha_1 r_1^{n-2}(c_1r_1+c_2)+\alpha_2 r_2^{n-2}(c_1r_2+c_2)$$
     چون $r_1$
     و
     $r_2$
     ریشه معادله مشخصه هستند، می‌توان نوشت:
     $$c_1 a_{n-1}+c_2 a_{n-2}=\alpha_1 r_1^{n-2}(r_1^{2})+\alpha_2 r_2^{n-2}(r_1^{2})=a_n$$
    \p
     حال نشان می‌دهیم که اگر 
     $a_n$
     جواب معادله بازگشتی باشد،
     به فرم
     $a_n=\alpha_1 r_1^n+\alpha_2 r_2^n$
     است.
    فرض کنید 
    $a_n$
    جواب رابطه بازگشتی است،
    اگر شرط اولیه این رابطه را به صورت
    $a_0=C_0$
    و
    $a_1=C_1$
    نشان دهیم،
    ابتدا ثابت می‌کنیم که
    $\alpha_1$
    و
    $\alpha_2$
    وجود دارند به طوری که 
    $b_n=\alpha_1 r_1^n+\alpha_2 r_2^n$
    هم در شرایط اولیه صدق کند.
    پس باید داشته باشیم:
    $$a_0=C_0=\alpha_1+\alpha_2$$
    $$a_1=C_1=\alpha_1r_1+\alpha_2r_2$$
    این معادله را برای 
    $\alpha_1$
    و
    $\alpha_2$
    حل می‌کنیم:
    $$C_1=\alpha_1r_1+(C_0-\alpha_1)r_2$$
    $$\rightarrow \alpha_1=\dfrac{C_1-C_0r_2}{r_1-r_2}$$
    $$\rightarrow \alpha_2=\dfrac{C_0r_1-C_1}{r_1-r_2}$$
    با توجه به این که می‌دانیم
    $r_1$
    و
    $r_2$
    متمایز هستند، معادلات بالا قطعا جواب دارند.
    بنابراین
    $b_n=\alpha_1 r_1^n+\alpha_2 r_2^n$
    هم جواب این معادله است.
    از طرفی می‌دانیم که روابط بازگشتی با شرط اولیه جواب یکتا دارند، بنابراین
    $a_n=b_n=\alpha_1 r_1^n+\alpha_2 r_2^n$
    است.
    پس ثابت کردیم که جواب معادله بازگشتی به فرم 
    $\alpha_1 r_1^n+\alpha_2 r_2^n$
    است.
    }
\end{PROBLEM}
\begin{THEOREM}
    \p$a_n$
    جواب رابطه‌ بازگشتی خطی همگن درجه ۲ 
    $$a_n=c_1 a_{n-1}+c_2 a_{n-2}$$
    می‌باشد، اگر و تنها اگر
    $a_n$
    به فرم
    $a_n=\alpha_1 r_1^n+\alpha_2 r_2^n$ 
    باشد
    که
    $r_1$
    و
    $r_2$
    ریشه‌های متمایز معادله مشخصه و 
    $\alpha_1$
    و
    $\alpha_2$
    اعداد حقیقی
    هستند.
\end{THEOREM}
\NOTE{در عبارت بالا، 
$\alpha_1$
و
$\alpha_2$
با توجه به شرایط اولیه مسئله مشخص می‌شود.}
\begin{PROBLEM}
    \p
    جواب رابطه‌ بازگشتی 
    $a_n=a_{n-1}+2a_{n-2}$
    را در صورتی که
    $a_0=2$
    و
    $a_1=7$
    باشد به دست آورید.
    \SOLUTION{
        \p
        معادله مشخصه این رابطه‌
        $r^2-r-1=0$
        است که ریشه‌های آن
        $r_1=2$
        و
        $r_2=-1$
        است.
        طبق قضیه بالا، جواب
        $\{a_n\}$
        به فرم زیر است:
        \[a_n=\alpha_1 2^n+\alpha_2 (-1)^n\]
        که
        $\alpha_1$
        و
        $\alpha_2$
        اعداد ثابت هستند.
        حال از جایگذاری شرایط اولیه داریم:
        \[a_0=\alpha_1+\alpha_2=2\]
        \[a_1=2\alpha_1-\alpha_2=7\]
        \[\rightarrow \alpha_1=3\]
        \[\rightarrow \alpha_2=-1\]
        پس جواب نهایی برابر است با:
        \[a_n=3\times 2^n-(-1)^n\]
    }
\end{PROBLEM}
\begin{PROBLEM}[جواب رابطه بازگشتی خطی همگن درجه ۲ ریشه مضاعف]
    \p
    نشان دهید
    $a_n$
    جواب رابطه‌ بازگشتی خطی همگن درجه ۲ 
    $$a_n=c_1 a_{n-1}+c_2 a_{n-2}$$
    است، اگر و تنها اگر
    $a_n$
    به فرم
    $a_n=\alpha_1 r_1^n+n\alpha_2 r_1^n$ 
    باشد
    که
    $r_1$
    ریشه مضاعف معادله مشخصه و 
    $\alpha_1$
    و
    $\alpha_2$
    اعداد حقیقی
    هستند.
    \SOLUTION{
        \p
        ابتدا نشان می‌دهیم که اگر 
        $a_n=\alpha_1 r_1^n+n\alpha_2 r_1^n$ 
        آنگاه
        $a_n$
    جواب رابطه‌ بازگشتی خطی همگن است.
    از جایگذاری این رابطه در معادله داریم:
    $$c_1 a_{n-1}+c_2 a_{n-2}=c_1 (\alpha_1 r_1^{n-1}+(n-1)\alpha_2 r_1^{n-1})+c_2(\alpha_1 r_1^{n-2}+(n-2)\alpha_2 r_1^{n-2})$$ 
    $$=\alpha_1 r_1^{n-2}(c_1r_1+c_2)+n\alpha_2 r_1^{n-2}(c_1r_1+c_2)+\alpha_2 r_1^{n-2}(-c_1r_1-2c_2)$$
    چون 
    $r_1$
    ریشه مضاعف است می‌توان نوشت:
    $$(r-r_1)^2=0\rightarrow c_1=-2r_1 , c_2=r_1^2$$
    بنابراین:
    $$c_1 a_{n-1}+c_2 a_{n-2}=$$
    $$\alpha_1 r_1^{n-2}(c_1r_1+c_2)+n\alpha_2 r_1^{n-2}(c_1r_1+c_2)+\underbrace{\alpha_2 r_1^{n-2}(2r_1^2-2r_1^2)}_{0}$$
     چون $r_1$
     ریشه معادله مشخصه است، می‌توان نوشت:
     $$c_1 a_{n-1}+c_2 a_{n-2}=\alpha_1 r_1^{n-2}(r_1^{2})+n\alpha_2 r_1^{n-2}(r_1^{2})=a_n$$
    \p
     حال نشان می‌دهیم که اگر 
     $a_n$
     جواب معادله بازگشتی با شروط بالا باشد،
     به فرم
     $a_n=\alpha_1 r_1^n+n\alpha_2 r_1^n$ 
     است.
     دو حالت داریم:
     \begin{enumerate}
        \item
        $r_1=0$:
        در این حالت تنها جواب معادله
        $a_n=0$
        است چون به معادله 
        $a_n^2=0$
        می‌رسیم.

        \item
        $r_1\neq 0$:
        فرض کنید 
        $a_n$
        جواب رابطه بازگشتی است،
        اگر شرط اولیه این رابطه را به صورت
        $a_0=C_0$
        و
        $a_1=C_1$
        نشان دهیم،
        ابتدا ثابت می‌کنیم که
        $\alpha_1$
        و
        $\alpha_2$
        وجود دارند به طوری که 
        $b_n=\alpha_1 r_1^n+n\alpha_2 r_1^n$
        هم در شرایط اولیه صدق کند.
        پس باید داشته باشیم:
        $$a_0=C_0=\alpha_1$$
        $$a_1=C_1=\alpha_1r_1+\alpha_2r_1$$
        این معادله را برای 
        $\alpha_1$
        و
        $\alpha_2$
        حل می‌کنیم:
        $$\alpha_1=C_0$$
        $$\alpha_2=C_1-\dfrac{\alpha_1r_1}{r_1}=\dfrac{r_1(C_1-C_0)}{r_1}$$
        چون
        $r_1\neq 0$ 
        معادلات بالا قطعا جواب دارند.
        بنابراین
        $b_n=\alpha_1 r_1^n+n\alpha_2 r_1^n$
        هم جواب این معادله است.
        از طرفی می‌دانیم که روابط بازگشتی با شرط اولیه جواب یکتا دارند، بنابراین
        $a_n=b_n=\alpha_1 r_1^n+n\alpha_2 r_1^n$
        است.
        پس ثابت کردیم که جواب معادله بازگشتی به فرم 
        $\alpha_1 r_1^n+n\alpha_2 r_1^n$
        است.
    \end{enumerate}
}
\end{PROBLEM}
\begin{THEOREM}
    \p
    \p$a_n$
    جواب رابطه‌ بازگشتی خطی همگن درجه ۲ 
    $$a_n=c_1 a_{n-1}+c_2 a_{n-2}$$
    است، اگر و تنها اگر
    $a_n$
    به فرم
    $a_n=\alpha_1 r_1^n+n\alpha_2 r_1^n$ 
    باشد
    که
    $r_1$
    ریشه مضاعف معادله مشخصه و 
    $\alpha_1$
    و
    $\alpha_2$
    اعداد حقیقی
    هستند.

\end{THEOREM}
\begin{PROBLEM}
    \p
    جواب رابطه‌ بازگشتی 
    $a_n=6a_{n-1}-9a_{n-2}$
    را در صورتی که
    $a_0=1$
    و
    $a_1=6$
    باشد به دست آورید.
    \SOLUTION[\REF{R} پاسخ]{
        \p
        معادله مشخصه این رابطه‌
        $r^2-6r+9=0$
        بوده که ریشه آن
        $r_1=3$
        می‌باشد.
        طبق قضیه بالا، جواب
        $a_n$
        به فرم زیر است:
        \[a_n=\alpha_1 3^n+n\alpha_2 (3)^n\]
        که
        $\alpha_1$
        و
        $\alpha_2$
        اعداد ثابت هستند.
        حال از جایگذاری شرایط اولیه داریم:
        \[a_0=\alpha_1=1\]
        \[a_1=3\alpha_1+3\alpha_2=6\]
        \[\rightarrow \alpha_2=1\]
        پس جواب نهایی برابر است با:
        \[a_n=3^n+n3^n\]
    }
\end{PROBLEM}
\p
می‌توان تعمیم قضایای فوق به رابطه‌ بازگشتی درجه
$k$
به صورت مشابه اثبات کرد.
اثبات این قضیه
به عنوان 
\CROSSREF[تمرین]{اثبات قضیه پاسخ رابطه بازگشتی خطی همگن درجه}
 بر عهده شما است.
%\begin{THEOREM}
%    \p
%    رابطه‌ بازگشتی درجه 
%    $k$
%    ام
%    \[a_n=c_{1}a_{n-1}+c_{2}a_{n-2}+...+c_{k}a_{n-k}\]
%    را در نظر بگیرید. فرض کنید که معادله مشخصه آن دارای 
%    $k$
%    ریشه حقیقی و متمایز 
%    $r_1, r_2, ..., r_k $
%    است.
%    در این صورت دنباله 
%    $a_n$
%    جواب مسئله است اگر و تنها اگر
%    \[a_n=\alpha_1 r_1^n+\alpha_2 r_2^n+...+\alpha_k r_k^n\]
%    که
%    $\alpha_1,\alpha_2,..., \alpha_k$
%    اعداد ثابت هستند.
%
%\end{THEOREM}


\begin{THEOREM}
    \p
    رابطه‌ بازگشتی درجه 
    $k$
    ام
    \[a_n=c_{1}a_{n-1}+c_{2}a_{n-2}+...+c_{k}a_{n-k}\]
    را در نظر بگیرید. فرض کنید که معادله مشخصه آن دارای 
    $t$
    ریشه حقیقی و متمایز 
    $r_1, r_2, ..., r_t $
    است که ریشه
    $i$ام 
    $m_i$
    بار تکرار شده و  
    $m_i \geq 1$
    و
    $\sum_1^k m_i=k$،
    در این صورت دنباله 
    $a_n$
    جواب مسئله است اگر و تنها اگر
    \[a_n=(\alpha_{1,0}+ n\alpha_{1,1} + ... + n^{m_1-1}\alpha_{1,m_1-1})r_1^n\]
    \[+(\alpha_{2,0}+ n\alpha_{2,1} + ... + n^{m_2-1}\alpha_{2,m_2-1})r_2^n\]
    \[+...\]
    \[+ (\alpha_{t,0} + n\alpha_{t,1} + ... + n^{m_t-1}\alpha_{t,m_t-1})r_t^n\]
    \[=\sum_{i=1}^t\sum_{j=0}^{m_i-1}r^i(n^j\alpha_{i,j})\]
    که
    $\alpha_{i,j}$
    ها
    اعداد ثابت هستند.

\end{THEOREM}

\begin{PROBLEM}
    \p
    جواب رابطه‌ بازگشتی 
    $a_n=2a_{n-1}+a_{n-2}-2a_{n-3}$
    را در صورتی که
    $a_0=3$،
    $a_1=2$
    و
    $a_2=6$
    باشد به دست آورید.
    \SOLUTION{
        \p
        معادله مشخصه این رابطه‌
        $r^3-2r^2-r+2=(r-1)(r+1)(r-2)=0$
        بوده که ریشه آن
        $r_1=1$،
        $r_2=-1$
        و
        $r_3=2$
        می‌باشد.
        طبق قضیه بالا، جواب
        $a_n$
        به فرم زیر است:
        \[a_n=\alpha_1 1^n+\alpha_2 (-1)^n+\alpha_3 (2)^n\]
        که
        $\alpha_1$،
        $\alpha_2$
        و
        $\alpha_3$
        اعداد ثابت هستند.
        حال از جایگذاری شرایط اولیه داریم:
        \[a_0=\alpha_1+\alpha_2+\alpha_3=3\]
        \[a_1=\alpha_1-\alpha_2+2\alpha_3=2\]
        \[a_2=\alpha_1+\alpha_2+4\alpha_3=6\]
        \[\rightarrow \alpha_1=1,\alpha_2=1,\alpha_3=1\]
        پس جواب نهایی برابر است با:
        \[a_n=1+(-1)^n+2^n\]
    }
\end{PROBLEM}


\begin{PROBLEM}
    \p
    جواب رابطه‌ بازگشتی 
    $a_n=a_{n-1}+a_{n-2}-a_{n-3}$
    را در صورتی که
    $a_0=2$
    ,
    $a_1=1$
    و
    $a_2=4$
    باشد به دست آورید.
    \SOLUTION{
        \p
        معادله مشخصه این رابطه‌
        $r^3-r^2-r+1=(r-1)(r+1)(r-1)=0$
        بوده که ریشه آن
        $r_1=1$،
        $r_2=-1$
        می‌باشد.
        طبق قضیه بالا، جواب
        $a_n$
        به فرم زیر است:
        \[a_n=\alpha_1 1^n+n\alpha_2 1^n+\alpha_3 (-1)^n\]
        که
        $\alpha_1$،
        $\alpha_2$
        و
        $\alpha_3$
        اعداد ثابت هستند.
        حال از جایگذاری شرایط اولیه داریم:
        \[a_0=\alpha_1+\alpha_3=2\]
        \[a_1=\alpha_1+\alpha_2-\alpha_3=1\]
        \[a_2=\alpha_1+2\alpha_2+\alpha_3=4\]
        \[\rightarrow \alpha_1=1,\alpha_2=1,\alpha_3=1\]
        پس جواب نهایی برابر است با:
        \[a_n=1+n+(-1)^n\]
    }
\end{PROBLEM}

\SUBSECTION{حل روابط بازگشتی به کمک توابع مولد}
\p
برخی روابط بازگشتی را می‌توان به کمک توابع مولد حل کرد.
مراحل حل را در مثال زیر به ترتیب دنبال کنید.

\subfile{./example1.tex}

\NOTE{
    در روش حل روابط بازگشتی به کمک توابع مولد، گام‌های ارائه شده در حل مثال بالا تقریباً ثابت هستند.
}



\subfile{./example3.tex}
\subfile{./example2.tex}

\p
برای به دست آوردن فرم بسته روابط بازگشتی، گاهی اوقات نیاز است از تابع مولد نمایی استفاده کنیم.
برای این کار، لازم است در گام دوم روش ارائه شده در مثال اول این بخش، 
\textbf{طرفین رابطه‌ی بازگشتی را بجای
$x^n$ در $\frac{x^n}{n!}$ ضرب کنیم.}

\subfile{./ExponentialExample1.tex}
\begin{PROBLEM}
    \p
    رابطه صریح ‌رابطه‌ی بازگشتی 
    $a_n=a_{n-1}+n$
    را در صورتی که
    $a_0=1$
    باشد به دست آورید.
    \SOLUTION{
        \p
       
        تابع مولد $G(x)$ را به شکل زیر تعریف می‌کنیم:
        
        $$G(x) = \sum_{n = 0}^{\infty} a_nx^n$$
        
        طرفین رابطه‌ی بازگشتی را در $x^n$ ضرب می‌کنیم و سیگما می‌گیریم:
        
        \begin{center}
            $a_n = a_{n-1} + n$ 
            \medbreak
            $\sum_{n = 1}^{\infty} a_nx^n = \sum_{n = 1}^{\infty} (a_{n-1}+n)x^n$ و
            $ n \geqslant 1 $ 
            \medbreak
            $G(x) - a_0 = \sum_{n = 1}^{\infty} a_{n-1}x^n + \sum_{n = 1}^{\infty}nx^n$
            \medbreak
            $G(x) - 1 = x\sum_{n = 1}^{\infty} a_{n-1}x^{n-1} + \sum_{n = 1}^{\infty}nx^n$
            \medbreak
            $G(x) - 1 = xG(x) + \sum_{n = 0}^{\infty}nx^n$
            \medbreak
            $G(x) - 1 - xG(x) = \frac{x}{(1 - x)^2}$
            \medbreak
            $G(x)(1 - x) = 1 + \frac{x}{(1 - x)^2}$
            \medbreak
            $G(x) = \frac{1}{1 - x} + \frac{x}{(1 - x)^3}$
            \medbreak
            $G(x) = \sum_{n = 0}^{\infty} x^n + \sum_{n = 0}^{\infty} \binom{n + 1}{2}x^n$
            \medbreak
            $G(x) = \sum_{n = 0}^{\infty} (\binom{n + 1}{2} + 1)x^n$
        \end{center}
        طبق تعریفی که در ابتدای سوال از تابع مولد ارائه دادیم، ضریب $x^n$ همان $a_n$ است
        که در اینجا برابر
        $\binom{n + 1}{2} + 1$
        می‌باشد.
    }
\end{PROBLEM}
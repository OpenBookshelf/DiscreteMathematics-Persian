\begin{PROBLEM}
    \p
     لاک پشتی روی یک 8 ضلعی قرار دارد و در هر مرحله می‌تواند به یکی از رئوس مجاور مکانی که هست حرکت کند. اگر لاک پشت از راس 1 شروع کرده باشد و بخواهد به راس 5 برسد(شماره گذاری به ترتیب ساعتگرد).
	\begin{enumerate}
		\item
      تعداد راه‌های رسیدن لاک پشت به این خانه را به صورت بازگشتی بیابید.

      \SOLUTION{
        \p
        برای حل این قسمت، از هر خانه دو حالت را در نظر می‌گیریم که به صورت ساعتگرد یا پادساعتگرد به خانه‌های مجاور برویم. اگر تعریف کنیم \(A_i\) برابر است با تعداد حالات رفتن از خانه‌ی i ام به خانه ی 5 ام، آنگاه برای هر خانه غیر از خانه ی 5 ام داریم:
        \[ A_i=A_{(i-1)\%8} + A_{(i+1)\%8}\]که در روابط بالا \% نشان دهنده‌ی باقی مانده است. همینطور درنظر می‌گیریم \(A_0=A_8\) و برای خانه‌ی 5ام داریم:
        \[A_5 = 1\]زیرا پس از ورود لاک پشت به خانه 5ام کار تمام است.
    }
     
\item
      تعداد راه‌های رسیدن لاک پشت به این خانه را بعد از
       $2n$ 
       مرحله به صورت بازگشتی بیابید.
    \SOLUTION{
        \p
      تعریف می‌کنیم \(A_n\) برابر است با تعداد راه‌ها برای رسیدن به خانه 5ام بعد از 
      $2n $
      مرحله با شروع از خانه اول و \(B_n\) برابر است با تعداد راه‌ها برای رسیدن به خانه 5 ام بعد از 
      $2n$
       مرحله با شروع از خانه سوم. روابط موجود بین \(A_n\) و \(B_n\) را بدست می‌آوریم. با شروع از خانه 1 و با انجام دو حرکت، چهار حالت برای دنباله حرکات داریم. یا با 2 حرکت به خانه 7 ام برویم که از این خانه به دلیل تقارن، تعداد حالات موجود برای رسیدن به خانه پنجم برابر است با خانه سوم. یا با 2 حرکت به خانه 3ام برویم. یا با دو حرکت به خانه دوم یا هشتم برویم و برگردیم به خانه اول. این چهار حالت را می‌توان در معادله زیر نشان داد.
\[ A_n=2B_{n-1}+2A_{n-1} \]از طرف دیگر برای خانه های سوم و یا هفتم از بین چهار حالت حرکت یک حالت منجر به اتمام مساله می‌شود که با دو حرکت به خانه پنجم رفتن است منتها این کار تا انتهای
$2n$
 حرکت خواسته شده در سوال مجاز نیست. بنابراین 3 حالت حرکت باقی می‌ماند. دو حالت این که لاک پشت با دو حرکت به یکی از خانه‌های 2 و یا 4 برود و برگردد به خانه 3 که حال باید \(B_{n-1}\) را طی کند. و حالت سوم این که با دو حرکت به خانه اول برگردد که باید \(A_{n-1}\) را در آن جا طی کند. بنابراین به معادله زیر می‌رسیم:
\[ B_n = 2 B_{n-1}+A_{n-1} \]از حذف متغیر بازگشتی 
$B$
به رابطه بازگشتی زیر برای \(A_n\) می‌رسیم:
\[ A_n=4A_{n-1} - 2A_{n-2} \]
برای شرایط اولیه مورد نیاز هم داریم:
\[ A_1 = 0, A_2 = 2 \]که قابل شمارش است.
} 
	\end{enumerate}

\end{PROBLEM}
\begin{PROBLEM}
  \p
  شخصی از نقطه $( 0 , 0)$ شروع به حرکت می‌کند و در هر مرحله می‌تواند 
  از محل کنونی خود یعنی نقطه
  $( x , y )$
  به
  نقطه 
  $( x + 1, y + 1 )$
  یا نقطه 
  $( x + 1, y - 1 )$
 برود با این شرط که
 $y$
 هیچگاه منفی نمی‌شود.
 او به چند طریق می‌تواند به خانه 
 $( 2n , 0 ) $
   برسد؟  
   \SOLUTION{
    \p
با حالت‌بندی روی اولین نقطه برخورد فرد با محور
$x$
, می‌توانیم به یک رابطه بازگشتی برسیم. تعریف می‌کنیم 
\(C_n\) 
حل مساله برای مسیر به طول 
$n$
است.
مدل حرکت در شکل زیر آمده است.
شخص از خانه زرد سمت چپ حرکتش را آغاز می‌کند و در هر حرکت به خانه بالا راست یا پایین راست خانه فعلی در صورت وجود می‌رود.
\centerimage{0.2}{./DMsoal3.png}
\p
حال اگر فرد دوباره برای اولین بار در نقطه با طول 
$2i$
به محور 
$x$
برگردد قبل آن قطعا در خانه‌ی بالا چپ آن قرار داشته است. فرض کنید اولین برخورد فرد به سطح زمین در شکل خانه ی زرد سمت راست باشد. در این صورت فرد حتما یک حرکت قبل در خانه سبز سمت راست قرار داشته است. همینطور فرد در ابتدای مسیر، حتما از خانه سبز در سمت چپ شکل عبور کرده است و در مسیر میان دو خانه سبز دیگر از سطح اول عبور نکرده است زیرا می‌دانیم خانه زرد سمت راست، اولین برخورد فرد با محور 
$x$
در طی مسیر است، بنابراین می‌توان شرایط حرکت فرد بین دو خانه سبز را حل همین مساله برای حالت
 \(C_{i-1}\)
 دانست که حل مساله برای مثلث کوچکتر که از ارتفاع خانه‌های سبز شروع می‌شود است که با توجه به این که حالات حرکت بین دو خانه زرد با حالات حرکت بین دو خانه سبز یکی است،
\(C_{i-1}\) 
برابر با تعداد حالات حرکت فرد تا اولین برخوردش با محور
$x$
نیز هست. بنابراین به رابطه زیر می‌رسیم که بر اساس اولین برخورد با محور
$x$
، حرکت فرد را حالت بندی می‌کند:
\[ C_n = \sum\limits_{i=1}^{n} C_{i-1}C_{n-i} \]
طبق چیزی که در بالا دیدیم، از مسیر حرکت فرد،
 \lr{2n-2i} 
 خانه پس از برخورد اول با محور
 $x$
 در خانه 
 $i$ 
 ام باقی می‌ماند که جمله 
 \(C_{n-i}\) 
 را نتیجه می‌دهد. بنابراین به رابطه اعداد کاتالان رسیدیم و طبق فرمول بسته برای جواب این مساله داریم:
\[ C_n = \binom{2n}{n}-\binom{2n}{n+1} \]
  }
  \end{PROBLEM}
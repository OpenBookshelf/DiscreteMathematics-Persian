\SECTION{اعداد کاتالان}
\p
اعداد کاتالان دنباله‌ای از اعداد طبیعی هستند که در پاسخ به برخی سوالات شمارش مطرح شده‌اند.
یکی از این مسائل را در ادامه می‌بینیم.

\subfile{./examples/1.tex}
\begin{DEFINITION}
    \p
      به اعدادی که در رابطه بازگشتی
    \[\begin{cases}
        C_{n+1}=\sum_{i=0}^n C_iC_{n-i} & n\geq 0 \\
        
        C_0=1
    \end{cases}
    \]
    صدق می‌کنند،
    \FOCUSEDON{ اعداد کاتالان}
      می‌گوییم.
\end{DEFINITION}

%در ادامه فرم صریح این دنباله را می‌یابیم.
\subfile{./examples/2.tex}
\begin{THEOREM}
  \p
    رابطه صریح اعداد کاتالان به صورت
    \[C_n=\binom{2n}{n}-\binom{2n}{n+1}=\dfrac{1}{n+1}\binom{2n}{n}\]
    است
    .
\end{THEOREM}

 کاربرد این دنباله در حل مسائل شمارش است. عموما در این سوالات لازم است که با یک مدل سازی مناسب، مسئله را به مسئله رشته‌های پرانتزی تبدیل کنیم یا رابطه بازگشتی اعداد کاتالان را در مسئله خود پیدا کنیم. در ادامه تعدادی از این مسائل را می‌بینیم.
\subfile{./examples/3.tex}
\subfile{./examples/4.tex}
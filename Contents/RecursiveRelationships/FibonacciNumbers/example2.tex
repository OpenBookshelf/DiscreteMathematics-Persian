\begin{PROBLEM}
    \p
    نشان دهید اگر
    $ n \geq 3 , \alpha = \frac{1+ \sqrt{5}}{2} $
    آنگاه
    $ f_n > \alpha ^ {n-2} $
    .

    \SOLUTION{
    \p
    از استقرا قوی استفاده می‌کنیم.

    پایه استقرا:
    برای
    $n = 3 $
    داریم
    $ f_3 = 2 < \alpha ^ {3-2} $
    و برای
    $n = 4 $
    داریم

    $ f_4 = 3 < \alpha ^ {4-2} = \alpha ^ 2 = \frac{3 + \sqrt{5}}{2} $
    که هر دو گزاره‌های درست هستند.

    فرض استقرا:
    فرض می‌کنیم برای 
    $n = j , 3 \leq j \leq k , k \geq 4 $
    گزاره
    $ f_j = \alpha ^{j-2}$
    ‌ درست است.


    حکم استقرا:
    می‌خواهیم ثابت کنیم برای
    $ n = k+1 $
    گزاره
    $ f_{k+1} = \alpha ^{k-1}$
    درست است.

    می‌دانیم 
    $\alpha$
    ریشه معادله
    $ x^2 - x -1 = 0 $
    است پس
    $ \alpha^2 = \alpha + 1$.

    $$ \alpha^{k-1} = \alpha^2 \alpha^{k-3} = (\alpha +1) \alpha^{k-3} = \alpha^{k-2} + \alpha^{k-3} $$

    با استفاده از فرض استقرا
    $f_{k-1} > \alpha^ {k-3} , f_{k} > \alpha^ {k-2}$
    است.

    بنابراین
    $$ f_{k+1} = f_k + f_{k-1} > \alpha^{k-2} + \alpha^{k-3} = \alpha^{k-1} $$

    }
\end{PROBLEM}
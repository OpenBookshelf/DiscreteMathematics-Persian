\SECTION{دنباله و اعداد فیبوناچی}
فیبوناچی، ریاضیدان ایتالیایی قرن ۱۳ ‌مسئله‌ی معروف رشد جمعیت خرگوش‌ها را در کتاب خود مطرح کرده است.
این مسئله را در مثال زیر بیان می‌کنیم.

\subfile{./rabbitsProblem.tex}

\begin{DEFINITION}
    به اعدادی که در رابطه بازگشتی
  \[\begin{cases}
      f_{n}=f_{n-1} + f_{n-2} & n\geq 2 \\
      
      f_0=1 ,
      f_1 = 1
  \end{cases}
  \]
  صدق می کنند,
  \FOCUSEDON{  دنباله‌ فیبوناچی}
    می‌گوییم.
\end{DEFINITION}


\subfile{./example1.tex}

\begin{THEOREM}
    \p
    \FOCUSEDON{تابع مولد دنباله فیبوناچی}
    برابر است با:
    $$F(x) = \frac{1}{1 - x - x^2}$$
    \FOCUSEDON{فرمول صریح فیبوناچی}
    برابر است با:
    $$f_n = \frac{1}{\sqrt{5}}((\frac{-1 + \sqrt{5}}{2})^{n} - (\frac{1 - \sqrt{5}}{2})^{n})$$
\end{THEOREM}

\subfile{./example2.tex}